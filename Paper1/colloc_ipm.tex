\section{Collocation accelerated IPM for steady state problems}
\label{sec:collIPM}

In the following, we look at steady state problems, i.e. we have
\begin{align}\label{eq:hyperbolicProblemSteady}
\nabla\cdot\bm{f}(\bm{u}(t,\bm{x},\bm{\xi})) = \bm{0} \enskip \text{ in } D
\end{align}
with adequate boundary conditions. A general strategy for computing the steady state solution to \eqref{eq:hyperbolicProblemSteady} is to introduce a pseudo-time and numerically treat \eqref{eq:hyperbolicProblemSteady} as an unsteady problem. A steady state solution is then obtained by iterating in pseudo-time untill the solution remains constant.
When computing steady state moments for this problem, the run time of different UQ methods crucially depends on the chosen initial condition and its distance to the steady state solution.