\subsection{Collocation accelerated IPM}
\label{sec:collIPM}

In the following, we look at steady state problems, i.e. we have
\begin{align}\label{eq:hyperbolicProblemSteady}
\nabla\cdot\bm{f}(\bm{u}(\bm{x},\bm{\xi})) = \bm{0} \enskip \text{ in } D
\end{align}
with adequate boundary conditions. A general strategy for computing the steady state solution to \eqref{eq:hyperbolicProblemSteady} is to introduce a pseudo-time and numerically treat \eqref{eq:hyperbolicProblemSteady} as an unsteady problem. A steady state solution is then obtained by iterating in pseudo-time until the solution remains constant.
Similar to the unsteady case \eqref{eq:hyperbolicProblem}, we can again derive a moment system 
\begin{align}\label{eq:MomentSystemSteady}
\nabla\cdot\langle\bm{f}(\bm{u}(\bm{x},\bm{\xi}))\bm{\varphi}^T\rangle^T = \bm{0} \enskip \text{ in } D
\end{align}
which is again needed for the construction of intrusive methdos. By adding a pseudo-time and using the IPM closure, we obtain the same system as in \eqref{eq:IPMmomentSystem}, i.e. Algorithm \ref{alg:IPM} can be used to iterate to a steady state solution. Note that now, the time iteration is not performed for a fixed number of time steps $N_t$, but until the residual
\begin{align*}
\sum_{j = 1}^{N_x} \Delta x_j \Vert \bm{\hat{u}}_j^n - \bm{\hat{u}}_j^{n-1} \Vert \leq \varepsilon
\end{align*}
is fulfilled. Note that since one is generally interested in low order moments such as the expectation value, this residual can be modified by only accounting for the zero order moments.

When computing steady state moments for this problem, the run time of different UQ methods crucially depends on the chosen initial condition and its distance to the steady state solution. However, it is not always clear how to choose a satisfactory initial guess for the moments. We tackle this problem with the following method: For a bad initial guess, we iterate to the steady state solution with the help of a cheap method. Note that one expects to need a lot of iteration steps to arrive at a steady state solution, however the overall computational time is acceptable due to the choice of a cheap method. The obtained inaccurate moments are then used to start a more accurate, but at the same time more expensive method. Due to the choice of the initial condition, a smaller number of iterations is needed to arrive at the steady state solution of the expensive method, i.e. the numerical costs are again acceptable.

I our case, the cheap method will be stochastic-Collocation with a small number of samples, whereas the expensive method will be IPM with a high number of moments. However different variants of the main idea can be applied: