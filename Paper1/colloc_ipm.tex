In the following, we look at steady state problems, i.e. we have
\begin{align}\label{eq:hyperbolicProblemSteady}
\nabla\cdot\bm{f}(\bm{u}(\bm{x},\bm{\xi})) = \bm{0} \enskip \text{ in } D
\end{align}
with adequate boundary conditions. A general strategy for computing the steady state solution to \eqref{eq:hyperbolicProblemSteady} is to introduce a pseudo-time and numerically treat \eqref{eq:hyperbolicProblemSteady} as an unsteady problem. A steady state solution is then obtained by iterating in pseudo-time until the solution remains constant. It is important to point out that the time it takes to converge to a steady state solution is crucially affected by the chosen initial condition and its distance to the steady state solution.
Similar to the unsteady case \eqref{eq:hyperbolicProblem}, we can again derive a moment system for \eqref{eq:hyperbolicProblemSteady} given by
\begin{align}\label{eq:MomentSystemSteady}
\nabla\cdot\langle\bm{f}(\bm{u}(\bm{x},\bm{\xi}))\bm{\varphi}^T\rangle^T = \bm{0} \enskip \text{ in } D
\end{align}
which is again needed for the construction of intrusive methods. By adding a pseudo-time and using the IPM closure, we obtain the same system as in \eqref{eq:IPMmomentSystem}, i.e. Algorithm \ref{alg:IPM} can be used to iterate to a steady state solution. Note that now, the time iteration is not performed for a fixed number of time steps $N_t$, but until the condition
\begin{align}\label{eq:residualUnsteady}
\sum_{j = 1}^{N_x} \Delta x_j \Vert \bm{\hat{u}}_j^n - \bm{\hat{u}}_j^{n-1} \Vert \leq \varepsilon
\end{align}
is fulfilled. Since one is generally interested in low order moments such as the expectation value, this residual can be modified by only accounting for the zero order moments.

\subsection{Collocation accelerated IPM}
\label{sec:collIPM}
Commonly, a great amount of iterations in pseudo-time are needed to converge to a steady state solution. Consequently, the IPM method which requires solving the dual problem \eqref{eq:dualProblem} in every spatial cell in each iteration becomes prohibitively expensive. We tackle this problem by using IPM only as a postprocesssing step for the steady solution obtained by a cheap method. (Or vice-versa, we use a cheap method as a preprocessing step for IPM). In our case, we perform the preprocessing step with stochastic-Collocation, i.e. we converge the moments to a steady state solution by applying collocation steps. The obtained moments are then used as initial condition for the IPM moment system (for which the moments are no longer a steady state solution). After applying a significantly reduced number of IPM iterations, we obtain a steady state IPM solution. In our numerical experiments presented in section \ref{sec:results}, we can show that the overall costs are dominated by the large number of cheap collocation steps and not by the small number of expensive IPM steps, while the solution shows the expected desirable properties of the IPM solution.

Different variants of this method are possible:
\begin{itemize}
\item Since the IPM iterations will again modify the steady state Collocation solution, it is not necessary to converge Collocation to the exact steady state solution before starting IPM. Here, one needs to determine an indicator to choose at which residual the collocation iteration is sufficiently accurate and can therefore be switched to IPM.
\item The main idea is to use a cheap but inexact method as a preconditioner for an expensive but accurate method. Here, one is not limited in choosing SC and IPM, but one can for example choose Monte Carlo methods as an accelerator or SC with an increased number of quadrature points as the expensive method. Note that in the latter case, a map from the moments to the solution at the increased quadrature set is required, for which the IPM reconstruction \eqref{eq:ansatz} would be a suitable choice.
\end{itemize}