\section{Results}
\label{sec:results}

\subsection{2D Euler equations}
We start by quantifying the effects of an uncertain angle of attack $\phi\sim U(0.75,1.75)$ for a NACA0012 profile computed with different methods. The stochastic Euler equations in two dimensions are given by
\begin{align*}
\partial_t
\begin{pmatrix}
\rho \\ \rho v_1 \\ \rho v_2 \\ \rho e
\end{pmatrix}
+\partial_{x_1}
\begin{pmatrix}
\rho v_1 \\ \rho v_1^2 +p \\ \rho v_1 v_2 \\  v_1 (\rho e+p)
\end{pmatrix}
+\partial_{x_2}
\begin{pmatrix}
\rho v_2 \\ \rho v_1 v_2 \\ \rho v_2^2+p \\ v_2 (\rho e+p)
\end{pmatrix}
=\bm{0}.
\end{align*}
These equations determine the time evolution of the conserved variables $(\rho,rho \bm v, rho e)$, i.e. density, momentum and energy. A closure for the pressure $p$ is given by
\begin{align*}
p = (\gamma-1)\rho\left(e-\frac12(v_1^2+v_2^2)\right).
\end{align*}
Since the the fluid of the following test cases is air, we choose the heat capacity ratio $\gamma$ to be $1.4$. The spatial mesh discretizes the flow domain around the airfoil. At the airfoil boundary $\Gamma_{0}$, we use the Euler slip condition $\bm v^T\bm n = 0$, where $\bm n$ denotes the surface normal. At a sufficiently large distance away from the airfoil, we assume a far field flow with a given Mach number $Ma = 0.8$, pressure $p = 101,325$ Pa and a temperature of $273.15$ K. Now the angle of attack $\phi$ is uniformly distributed in the interval of $[0.75,1.75]$ degrees. I.e. we choose $\phi(\xi) = 1.25 + 0.5\xi$ where $\xi\sim U(-1,1)$.
The aim is to quantify the effects arising from the one-dimensional uncertainty $\xi$ with different methods. The IPM methods makes use of the acceleration strategies proposed in this work. To be able to measure the quality of the obtained solutions, we compute a reference solution using stochastic-Collocation with $100$ Gauss-Lobatto quadrature points.

In the following, we compare stochastic-Collocation with stochastic-Galerkin and IPM as well as its proposed acceleration techniques. Note that since IPM generalizes SG, all proposed techniques can be used for this method as well. For more information on the chosen entropy and the resulting solution ansatz for IPM, see \ref{app:IPM2DEuler}.

 and compare against Collocation with $5$ collocation points as well as SG and IPM with $5$ moments and $10$ quadrature points. Furthermore, we use convergence accelerated (caIPM) as well as convergence accelerated one-shot IPM (caosIPM), which we introduced in Sections \ref{sec:collIPM} and \ref{sec:OneShotIPM} with $10$ moments and $15$ quadrature points to converge the collocation solution with $5$ quadrature points to an entropy solution with increased accuracy.

\begin{figure}[h!]
\centering
		\centering
		\includegraphics[scale=0.7]{figs/{L2_error_E[rho]osIPMIPMCol}.pdf}
		\label{fig:sub1}
	\caption{L2 error at airfoil with 5 MPI and 2 OMP threads.}
	\label{fig:Obstacles2D}
\end{figure}

\begin{figure}[h!]
\centering
		\centering
		\includegraphics[scale=0.7]{figs/{L2_error_E[rho]}.pdf}
		\label{fig:sub1}
	\caption{L2 error at airfoil with 5 MPI and 2 OMP threads.}
	\label{fig:Obstacles2D}
\end{figure}

\begin{figure}[h!]
\centering
		\centering
		\includegraphics[scale=0.7]{figs/{convergence_runtime_residual}.pdf}
		\label{fig:sub1}
	\caption{Convergence to steady state with 5 MPI and 2 OMP threads.}
	\label{fig:Obstacles2D}
\end{figure}

\newgeometry{top=1.5cm, left=1.0cm}
\begin{figure}[h!]
\centering
	\begin{subfigure}{0.33\linewidth}
		\centering
		\includegraphics[scale=0.2]{figs/expectedRho/{euler2D_nacaCoarse_sc_n5_s05_aoa}.png}
		\label{fig:sub1}
	\end{subfigure}%
	\begin{subfigure}{0.33\linewidth}
		\centering
		\includegraphics[scale=0.2]{figs/expectedRho/{euler2D_nacaCoarse_sg_n5_nq10_s05_aoa}.png}
		\label{fig:sub2}
	\end{subfigure}%
	\begin{subfigure}{0.33\linewidth}
		\centering
		\includegraphics[scale=0.2]{figs/expectedRho/{euler2D_nacaCoarse_ipm_n5_nq10_s05_aoa}.png}
		\label{fig:sub1}
	\end{subfigure}
	
	\begin{subfigure}{0.33\linewidth}
		\centering
		\includegraphics[scale=0.2]{figs/expectedRho/{euler2D_nacaCoarse_caipm_n10_nq15_s05_aoa}.png}
		\label{fig:sub2}
	\end{subfigure}%
	\begin{subfigure}{0.33\linewidth}
		\centering
		\includegraphics[scale=0.2]{figs/expectedRho/{euler2D_nacaCoarse_caosipm_n10_nq15_s05_aoa}.png}
		\label{fig:sub1}
	\end{subfigure}%
	\begin{subfigure}{0.33\linewidth}
		\centering
		\includegraphics[scale=0.2]{figs/expectedRho/{euler2D_nacaCoarse_sc_n50_s05_aoa_legend}.png}
		\label{fig:sub2}
	\end{subfigure}%
	\caption{E$[\rho]$ (from top left to bottom right) SC, SG, IPM, caIPM, caosIPM, reference solution.}
	\label{fig:Obstacles2D}
\end{figure}

\begin{figure}[h!]
\centering
	\begin{subfigure}{0.33\linewidth}
		\centering
		\includegraphics[scale=0.2]{figs/varRho/{euler2D_nacaCoarse_sc_n5_s05_aoa_var}.png}
		\label{fig:sub1}
	\end{subfigure}%
	\begin{subfigure}{0.33\linewidth}
		\centering
		\includegraphics[scale=0.2]{figs/varRho/{euler2D_nacaCoarse_sg_n5_nq10_s05_aoa_var}.png}
		\label{fig:sub2}
	\end{subfigure}%
	\begin{subfigure}{0.33\linewidth}
		\centering
		\includegraphics[scale=0.2]{figs/varRho/{euler2D_nacaCoarse_ipm_n5_nq10_s05_aoa_var}.png}
		\label{fig:sub1}
	\end{subfigure}
	
	\begin{subfigure}{0.33\linewidth}
		\centering
		\includegraphics[scale=0.2]{figs/varRho/{euler2D_nacaCoarse_caipm_n10_nq15_s05_aoa_var}.png}
		\label{fig:sub2}
	\end{subfigure}%
	\begin{subfigure}{0.33\linewidth}
		\centering
		\includegraphics[scale=0.2]{figs/varRho/{euler2D_nacaCoarse_caosipm_n10_nq15_s05_aoa_var}.png}
		\label{fig:sub1}
	\end{subfigure}%
	\begin{subfigure}{0.33\linewidth}
		\centering
		\includegraphics[scale=0.2]{figs/varRho/{euler2D_nacaCoarse_sc_n50_s05_aoa_var_legend}.png}
		\label{fig:sub2}
	\end{subfigure}%
	\caption{Var$[\rho]$ (from top left to bottom right) SC, SG, IPM, caIPM, caosIPM, reference solution.}
	\label{fig:Obstacles2D}
\end{figure}

%%% error plots %%%
\begin{figure}[h!]
\centering
	\begin{subfigure}{0.33\linewidth}
		\centering
		\includegraphics[scale=0.2]{figs/expectedRho/errors/{euler2D_nacaCoarse_sc_n5_s05_aoa_error_legend}.png}
		\label{fig:sub1}
	\end{subfigure}%
	\begin{subfigure}{0.33\linewidth}
		\centering
		\includegraphics[scale=0.2]{figs/expectedRho/errors/{euler2D_nacaCoarse_sg_n5_nq10_s05_aoa_error_legend}.png}
		\label{fig:sub2}
	\end{subfigure}%
	\begin{subfigure}{0.33\linewidth}
		\centering
		\includegraphics[scale=0.2]{figs/expectedRho/errors/{euler2D_nacaCoarse_ipm_n5_nq10_s05_aoa_error_legend}.png}
		\label{fig:sub1}
	\end{subfigure}
	
	\begin{subfigure}{0.33\linewidth}
		\centering
		\includegraphics[scale=0.2]{figs/expectedRho/errors/{euler2D_nacaCoarse_caipm_n10_nq15_s05_aoa_error_legend}.png}
		\label{fig:sub2}
	\end{subfigure}%
	\begin{subfigure}{0.33\linewidth}
		\centering
		\includegraphics[scale=0.2]{figs/expectedRho/errors/{euler2D_nacaCoarse_caosipm_n10_nq15_s05_aoa_error_legend}.png}
		\label{fig:sub1}
	\end{subfigure}
	\caption{E$[\rho]$ distance to reference solution (from top left to bottom right) SC, SG, IPM, caIPM, caosIPM.}
	\label{fig:Obstacles2D}
\end{figure}

\begin{figure}[h!]
\centering
	\begin{subfigure}{0.33\linewidth}
		\centering
		\includegraphics[scale=0.2]{figs/varRho/errors/{euler2D_nacaCoarse_sc_n5_s05_aoa_var_error_legend}.png}
		\label{fig:sub1}
	\end{subfigure}%
	\begin{subfigure}{0.33\linewidth}
		\centering
		\includegraphics[scale=0.2]{figs/varRho/errors/{euler2D_nacaCoarse_sg_n5_nq10_s05_aoa_var_error_legend}.png}
		\label{fig:sub2}
	\end{subfigure}%
	\begin{subfigure}{0.33\linewidth}
		\centering
		\includegraphics[scale=0.2]{figs/varRho/errors/{euler2D_nacaCoarse_ipm_n5_nq10_s05_aoa_var_error_legend}.png}
		\label{fig:sub1}
	\end{subfigure}
	
	\begin{subfigure}{0.33\linewidth}
		\centering
		\includegraphics[scale=0.2]{figs/varRho/errors/{euler2D_nacaCoarse_caipm_n10_nq15_s05_aoa_var_error_legend}.png}
		\label{fig:sub2}
	\end{subfigure}%
	\begin{subfigure}{0.33\linewidth}
		\centering
		\includegraphics[scale=0.2]{figs/varRho/errors/{euler2D_nacaCoarse_caosipm_n10_nq15_s05_aoa_var_error_legend}.png}
		\label{fig:sub1}
	\end{subfigure}
	\caption{Var$[\rho]$ distance to reference solution (from top left to bottom right) SC, SG, IPM, caIPM, caosIPM.}
	\label{fig:Obstacles2D}
\end{figure}
\newgeometry{top=2.5cm, bottom=2.5cm}