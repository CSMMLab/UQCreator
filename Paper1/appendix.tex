\appendix
\section{Costs of evaluating the numerical flux}
\label{app:costNumFlux}
In the following, we briefly discuss the number of operations needed when precomputing integrals versus the use of a kinetic flux for Burgers' equation. The stochastic Burgers' equation reads
\begin{align*}
\partial_t u + \partial_x \frac{u^2}{2} &= 0,\\
u(t=0,x,\xi) &= u_{IC}(x,\xi).
\end{align*}
The scalar random variable $\xi$ is uniformly distributed in the interval $[-1,1]$, hence the gPC basis functions $\bm\varphi=(\varphi_0,\cdots,\varphi_M)^T$ are the Legendre polynomials. Choosing the SG ansatz \eqref{eq:SGClosure} and testing with the gPC basis functions yields the SG moment system
\begin{align*}
\partial_t \hat u_i + \partial_x \frac12\sum_{n,m = 0}^M \hat u_n \hat u_m \langle \varphi_n\varphi_m\varphi_i \rangle = 0.
\end{align*}
Defining the matrices $\bm C_i := \langle \bm\varphi\bm\varphi^T\varphi_i\rangle\in\mathbb{R}^{N\times N}$ gives
\begin{align*}
\partial_t \bm{\hat u} + \partial_x \bm F(\bm{\hat u}) = 0
\end{align*}
with $F_i(\bm{\hat u}) = \frac12\bm{\hat u}^T\bm C_i\bm{\hat u}$. Note that $\bm{C}_i$ can be computed analytically, hence choosing a Lax-Friedrichs flux
\begin{align}\label{eq:numFluxAnalytic}
G_i^{(LF)}(\bm{\hat u}_{\ell},\bm{\hat u}_{r}) =\frac{1}{4}\left(\bm{\hat u}_{\ell}^T \bm{C}_i \bm{\hat u}_{\ell}+\bm{\hat u}_{r}^T \bm{C}_i \bm{\hat u}_{r}\right) - \frac{\Delta x}{2\Delta t}(\bm{\hat u}_{r}-\bm{\hat u}_{\ell})_i
\end{align}
requires no integral evaluations. Recall, that the numerical flux choice made in this work gives
\begin{align}\label{eq:numericalFluxIPMBurgers}
 \bm{G}(\bm{\hat u}_{\ell},\bm{\hat u}_{r}) = \sum_{k=1}^Q w_k g(\mathcal{U}(\bm{\hat u}_{\ell};\xi_k),\mathcal{U}(\bm{\hat u}_{r};\xi_k))\bm{\varphi}(\xi_k)f_{\Xi}(\xi_k),
\end{align}
where $\mathcal{U}$ is the SG ansatz \eqref{eq:SGClosure}. When the chosen deterministic flux $g$ is Lax-Friedrichs, the order of the polynomials inside the sum is $3M=3(N-1)$. Choosing a Gauss-Lobatto quadrature rule, $Q = \frac32 N -1$ quadrature points suffice for an exact computation of the numerical flux. Indeed, with this choice of quadrature points, the numerical fluxes \eqref{eq:numFluxAnalytic} and \eqref{eq:numericalFluxIPMBurgers} are equivalent. 
Counting the number of operations, one observes that our choice of the numerical flux \eqref{eq:numericalFluxIPMBurgers} is significantly cheaper: When computing and storing the values in a matrix $\bm A\in\mathbb{R}^{Q\times N}$ with entries $a_{ki} = \varphi_i(\xi_k)$ before running the program, the numerical flux \eqref{eq:numericalFluxIPMBurgers} can be split into two parts. First, we determine the SG solution at all quadrature points, i.e. we compute $\bm{u}^{(\ell)} := \bm A \bm{\hat u}_{\ell}$ and $\bm{u}^{(r)} := \bm A \bm{\hat u}_{r}$ which requires $O(N\cdot Q)$ operations. These solution values are then used to compute the numerical flux
\begin{align*}
G_i(\bm{\hat u}_{\ell},\bm{\hat u}_{r}) &= \sum_{k=1}^Q w_k g(u^{(\ell)}_k,u^{(r)}_k)a_{ki}f_{\Xi}(\xi_k),
\end{align*}
which again requires $O(N\cdot Q)$ operations, i.e. the costs are $O(N^2)$. The evaluation of \eqref{eq:numFluxAnalytic} however requires $O(N^3)$ operations.

\section{IPM for the 2D Euler equations}
\label{app:IPM2DEuler}

In the following, we provide details on the implementation of IPM for the 2D Euler equations. We for clarity, we denote the momentum by $m_1 := \rho v_1$ and $m_2:=\rho v_2$ and the energy by $E:=\rho e$. Then, the vector of conserved variables is $\bm u = (\rho,m_1,m_2,E)^T$. The entropy used is
\begin{align*}
s(\bm u) = -\rho \ln \left(\rho^{-\gamma} \left(E - \frac{m_1^2 + m_2^2}{2
\rho}\right)\right).
\end{align*}
Now the gradient of the entropy $\nabla_{\bm u} s$ has the components
\begin{align*}
\frac{\partial s}{\partial \rho} &= -\ln \left(\rho^{-\gamma}\left(E-\frac{m_1^2+m_2^2}{2 \rho }\right)\right)+\frac{m_1^2+m_2^2}{-2 \rho  E+m_1^2+m_2^2}+\gamma, \\
\frac{\partial s}{\partial m_i} &= -\frac{2\rho  m_i}{-2 \rho  E+m_1^2+m_2^2}, \\
\frac{\partial s}{\partial E} &=-\frac1\rho\left(E-\frac{m_1^2+m_2^2}{2 \rho }\right).
\end{align*}
To compute $\bm u_s(\bm\Lambda) = (\nabla_{\bm u}s)^{-1}(\bm \Lambda)$, we set $\bm \Lambda = \nabla_{\bm u}s(\bm u)$ and rearrange with respect to $\bm u$. Let us define
%\begin{align*}
%\alpha(\bm\Lambda) := \text{exp}\left(-\frac{0.5 \left(-10 \Lambda_1 \Lambda_4+5 \Lambda_2^2+5 \Lambda_3^2+14 \Lambda_4\right)}{\Lambda_4}\right) \cdot (-\Lambda_4)^{\frac{1}{1+\gamma}}.
%\end{align*}
\begin{align*}
\alpha(\bm\Lambda) := \text{exp}\left(\frac{ \Lambda_2^2 + \Lambda_3^2 - 2\Lambda_1  \Lambda_4 - 2 \Lambda_4  \gamma }{ 2 \Lambda_4(1-\gamma) } \right) \cdot (-\Lambda_4)^{\frac{1}{1-\gamma}}
\end{align*}
Then the solution ansatz $\bm u_s$ is given by
\begin{align*}
\rho(\bm\Lambda) &= \alpha(\bm\Lambda),\qquad m_1(\bm\Lambda) = -\frac{\Lambda_2 \alpha(\bm{\Lambda})}{\Lambda_4},\qquad m_2(\bm\Lambda) = -\frac{\Lambda_3 \alpha(\bm{\Lambda})}{\Lambda_4}, \\
E(\bm\Lambda) &= -\frac{  \alpha(\bm{\Lambda}) ( -\Lambda_2^2 - \Lambda_3^2 + 2\Lambda_4 ) }{ 2 \Lambda_4^2}.
\end{align*}