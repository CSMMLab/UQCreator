\documentclass[3p]{elsarticle}


% Packages and macros go here
\usepackage{bm}
\usepackage[parfill]{parskip}
\usepackage[utf8]{inputenc}
\usepackage{url}
\usepackage{amsmath,amssymb,amsfonts,amsthm}
\usepackage{color}
\usepackage{bm}
\usepackage{graphicx}
%\usepackage{subfig}
\usepackage{subcaption}
%\usepackage[]{algorithm2e}
%\usepackage{algorithmic}
%\usepackage{algorithmicx}
\usepackage[linesnumbered,ruled]{algorithm2e}

\usepackage{lipsum}
\usepackage{amsfonts}
\usepackage{graphicx}
\usepackage{epstopdf}
\usepackage{algorithmic}
\ifpdf
  \DeclareGraphicsExtensions{.eps,.pdf,.png,.jpg}
\else
  \DeclareGraphicsExtensions{.eps}
\fi

\newtheorem{theorem}{Theorem}
\newtheorem{proposition}{Proposition}
\newtheorem{remark}[theorem]{Remark}

\DeclareMathOperator*{\argmin}{arg\,min}

\makeatletter
\newsavebox\myboxA
\newsavebox\myboxB
\newlength\mylenA

\newcommand*\xoverline[2][0.75]{%
    \sbox{\myboxA}{$\m@th#2$}%
    \setbox\myboxB\null% Phantom box
    \ht\myboxB=\ht\myboxA%
    \dp\myboxB=\dp\myboxA%
    \wd\myboxB=#1\wd\myboxA% Scale phantom
    \sbox\myboxB{$\m@th\overline{\copy\myboxB}$}%  Overlined phantom
    \setlength\mylenA{\the\wd\myboxA}%   calc width diff
    \addtolength\mylenA{-\the\wd\myboxB}%
    \ifdim\wd\myboxB<\wd\myboxA%
       \rlap{\hskip 0.5\mylenA\usebox\myboxB}{\usebox\myboxA}%
    \else
        \hskip -0.5\mylenA\rlap{\usebox\myboxA}{\hskip 0.5\mylenA\usebox\myboxB}%
    \fi}
\makeatother

\journal{arXiv.org}

%strongly recommended
%\numberwithin{theorem}{section}

% Declare title and authors, without \thanks
\newcommand{\TheTitle}{A semi-intrusive Code Framework for Uncertainty Quantification} 

\date{\today}

% Sets running headers as well as PDF title and authors
%\headers{\RunningTitle}{\TheAuthors}


\usepackage{amsopn}
\DeclareMathOperator{\diag}{diag}
%\DeclareMathOperator{\argmin}{arg\min}

% FundRef data to be entered by SIAM
%<funding-group>
%<award-group>
%<funding-source>
%<named-content content-type="funder-name"> 
%</named-content> 
%<named-content content-type="funder-identifier"> 
%</named-content>
%</funding-source>
%<award-id> </award-id>
%</award-group>
%</funding-group>

\def\lambdabar{\xoverline{\bm{\lambda}}}
\def\lambdabarjn{\lambdabar_j^n}

\newcommand{\commentRyan}[1]{{\Large{\textcolor{red}{\newline[RYAN: #1]\newline}}}}
\newcommand{\commentJonas}[1]{{\Large{\textcolor{blue}{\newline[JONAS: #1]\newline}}}}

\begin{document}

\begin{frontmatter}

\title{\TheTitle}


\author[adressJonas]{Jonas Kusch}
\author[adressJannick]{Jannick Wolters}
\author[adressMartin]{Martin Frank}

\address[adressJonas]{Karlsruhe Institute of Technology, Karlsruhe,
    jonas.kusch@kit.edu}
\address[adressJannick]{Karlsruhe Institute of Technology, Karlsruhe, jannick.wolters@kit.edu}
\address[adressMartin]{Karlsruhe Institute of Technology, Karlsruhe, martin.frank@kit.edu}


\begin{abstract}
Methods for quantifying the effects of uncertainties in hyperbolic problems can be divided into intrusive and non-intrusive techniques. A main drawbacks of intrusive methods is the need to implement new code, whereas collocation methods recycle a given deterministic solver. Furthermore, intrusive methods come with increased computational costs, making it hard to compete with non-intrusive methods.
In this work, we present a code framework, which facilitates the implementation of intrusive methods by recycling parts of deterministic codes. Additionally, we introduce methods to decrease numerical costs of intrusive methods for steady problems. We demonstrate the effectiveness of the proposed strategies by comparing results of the uncertain NACA0012 testcase as well as the shallow water equations.
\end{abstract}

\begin{keyword}
uncertainty quantification, conservation laws, hyperbolic, intrusive, stochastic-Galerkin, Collocation
\end{keyword}

\end{frontmatter}

\section{Introduction}
Hyperbolic equations play an important role in various research areas such as fluid dynamics or plasma physics. Efficient numerical methods combined with robust implementations are widely available for these problems, however they do not account for uncertainties which can arise in measurement data or modeling assumptions. Including the effects of uncertainties in differential equations has become an important topic in the last decades. %Examples include computational fluid dynamics \cite{bijl2013uncertainty}. 

One strategy to represent the solution's dependence on uncertainties is to use a polynomial chaos (PC) expansion \cite{wiener1938homogeneous,xiu2002wiener}, i.e. the uncertainty space is spanned by polynomial basis functions. The remaining task is then to determine adequate expansion coefficients, often called moments or PC coefficients. Numerical methods for approximating these coefficients can be divided into intrusive and non-intrusive techniques. A popular non-intrusive method is the stochastic-Collocation (SC) method, see e.g. \cite{xiu2005high,babuvska2007stochastic,loeven2008probabilistic}, which computes the moments with the help of a numerical quadrature rule. Commonly, SC uses sparse grids, since they posses a reduced number of collocation points for multi-dimensional problems. Since the solution at a fixed quadrature point can be computed by a standard deterministic solver, the SC method does not require a significant implementation effort. Furthermore, SC is embarrassingly parallel, since the required computations decouple and the workload can easily be distributed across several processors. 

The main idea of intrusive methods is to derive a system of equations describing the time evolution of the moments which can then be solved with a deterministic numerical scheme. A popular approach to describe the moment system is the stochastic-Galerkin (SG) method \cite{ghanem2003stochastic}, which chooses a polynomial basis ansatz of the solution and performs a Galerkin projection to derive a closed system of equations. One significant drawback of SG is, that its moment system is not necessarily hyperbolic \cite{poette2009uncertainty}. A generalization of stochastic-Galerkin, which ensures hyperbolicity is the Intrusive Polynomial Moment (IPM) method \cite{poette2009uncertainty}. Instead of performing the PC expansion on the solution, the IPM method represents the entropy variables with polynomials. Besides yielding a hyperbolic moment system, the IPM method has several advantages: Choosing a quadratic entropy yields the stochastic-Galerkin moment system, i.e. IPM generalizes different intrusive methods. Furthermore, at least for scalar problems, IPM is significantly less oscillatory compared to SG \cite{kusch2017maximum}. Also, as discussed in \cite{poette2009uncertainty}, when choosing a physically correct entropy of the deterministic problem, the IPM solution dissipates the expectation value of the entropy, i.e. the IPM method yields a physically correct entropy solution. Unfortunately, the desirable properties of IPM come along with significantly increased numerical costs, since IPM requires the repeated computation of the entropic expansion coefficients from the moment vector, which involves solving a convex optimization problem. However, IPM and minimal entropy methods in general are well suited for modern HPC architectures \cite{garrett2015optimization}. 

When studying hyperbolic equations, the moment approximations of various methods such as Stochastic Galerkin \cite{le2004uncertainty,kusch2018filtered}, IPM \cite{kusch2018filtered,poette2019contribution} and stochastic-Collocation \cite{barth2013non,dwight2013adaptive} tend to show incorrect discontinuities in certain regions of the physical space. These non-physical structures dissolve when the number of basis functions is increased \cite{pettersson2009numerical,offner2017stability} or when artificial diffusion is added through either the spatial numerical method \cite{offner2017stability} or by filters \cite{kusch2018filtered}. Also, a multi-element approach which divides the uncertain domain into cells and uses piece-wise polynomial basis functions to represent the solution has proven to mitigate non-physical discontinuities \cite{wan2006multi,durrwachter2018hyperbolicity}. Non-intrusive Monte-Carlo methods \cite{mishra2012multi,mishra2012sparse,mishra2016numerical}, which randomly sample input unceratinties to compute quantities of interest are robust, but suffer from a slow rate of convergence while again lacking the ability to use adaptivity to their full extend. Discontinuous structures commonly arise on a small portion of the space-time domain. Therefore, intrusive methods seem to be an adequate choice since they are well suited for adaptive strategies. By locally increasing the polynomial order \cite{tryoen2012adaptive,kroker2012finite,giesselmann2017posteriori} or adding artificial viscosity \cite{kusch2018filtered} at certain spatial positions and time steps in which complex structures such as discontinuities occur, a given accuracy can be reached with significantly reduced numerical costs. In addition to that, the number of moments needed to obtain a certain order with intrusive methods is smaller than the number of quadrature points for SC. An additional downside of collocation methods are aliasing effects, which stem from the inexact approximation of integrals. Consequently, collocation methods typically require a higher number of unknowns than intrusive methods to reach a given accuracy \cite{xiu2009fast,alekseev2011estimation}. Therefore, one aim should be to accelerate intrusive methods, since they can potentially outperform non-intrusive methods in complex and high-dimensional settings. \\

In this paper, we propose acceleration techniques for intrusive methods and compare them against stochastic-Collocation. For steady and unsteady problems, we use adaptivity, for which intrusive methods provide a convenient framework: %Since complex structures in the uncertain domain tend to arise in small portions of the spatial mesh, our aim is to locally increase the accuracy of the stochastic discretization in region that show a complex structure in the random domain, while choosing a low order method in the remainder:
\begin{itemize}
\item Since complex structures in the uncertain domain tend to arise in small portions of the spatial mesh, our aim is to locally increase the accuracy of the stochastic discretization in region that show a complex structure in the random domain, while choosing a low order method in the remainder. Such an adaptive treatment cannot be realized with non--intrusive methods, since one needs to break up the black-box approach. To guarantee an efficient implementation, we propose an adaptive discretization strategy for IPM.
\end{itemize}
A steady problem provides different opportunities to take advantage of features of intrusive methods: 
\begin{itemize}
\item When using adaptivity, one can perform a large number of iterations to the steady state solution on a low number of moments and increase the maximal truncation order when the distance to the steady state has reached a specified barrier. Consequently, a large number of iterations will be performed by a cheap, low order method, i.e. we can reduce numerical costs. 
\item Perform an inexact map from the moments to the entropic expansion coefficients for IPM: Since the moments during the iteration process are inaccurate, i.e. they are not the correct steady state solution, we propose to not fully converge the dual iteration, which solves the IPM optimization problem. Consequently, the entropic expansion coefficients and the moments are converged simultaneously to their steady state, which is similar to the idea of One-Shot optimization in shape optimization \cite{hazra2005aerodynamic}.
\end{itemize}

The effectiveness of these acceleration ideas are tested by comparing results with stochastic-Collocation for the uncertain NACA test case as well as a bent shock tube problem. Our numerical studies show the following main results:
\begin{itemize}
\item In our test cases, the need to solve an optimization problem when using the IPM method leads to a significantly higher run time than SC and SG. However when using the discussed acceleration techniques, IPM requires the shortest time to reach a given accuracy.
\item Comparing SG with IPM, one observes that for the same number of unknowns, SG yields more accurate expectation values, whereas IPM shows improved variance approximations.
\item By studying aliasing effects, we show that SC requires a higher number of unknowns than intrusive methods (even for a one-dimensional uncertainty) to reach the same accuracy level.
\item Using sparse grids for the IPM discretization when the space of uncertainty is multi-dimensional, the number of quadrature points needed to guarantee sufficient regularity of the Hessian matrix is significantly increased.
\end{itemize}
The IPM and SG calculations use a semi-intrusive numerical method, meaning that the discretization allows recycling a given deterministic code to generate the IPM solver. While facilitating the task of implementing general intrusive methods, this framework reduces the number of operations required to compute numerical fluxes. Also, it provides the ability to base the intrusive method on the same deterministic solver as used in the implementation of a black-box fashion stochastic-Collocation code, which allows for, what we believe, a fair comparison between intrusive and non-intrusive methods. The code is publicly available to allow reproducibility \cite{uqcreator}.

The paper is structured as follows: After the introduction, we present the discussed methods in more detail in section~\ref{sec:background}. The numerical discretization as well as the implementation and structure of the semi-intrusive method is introduced in section \ref{sec:framework}. In section~\ref{sec:OneShotIPM}, we discuss the idea of not converging the dual iteration. Section~\ref{sec:adaptivity} extends the presented numerical framework to an algorithm making use of adaptivity. Implementation and parallelization details are given in section~\ref{sec:parallel}. A comparison of results computed with the presented methods is then given in section \ref{sec:results}, followed by a summary and outlook in section \ref{sec:summary_outlook}.

\section{Background}
\label{sec:background}
In the following, we briefly introduce the notation and methods used in this work. A general hyperbolic set of equations with random initial data can be written as
\begin{subequations}\label{eq:hyperbolicProblem}
\begin{align}
\label{eq:fulleq}\partial_t \bm{u}(t,\bm{x},\bm{\xi}) + \nabla&\cdot\bm{f}(\bm{u}(t,\bm{x},\bm{\xi})) = \bm{0} \enskip \text{ in } D, \\ \label{eq:ic}
\bm{u}(t=0,\bm{x},&\bm{\xi}) = \bm{u}_{\text{IC}}(\bm{x},\bm{\xi}),
\end{align}
\end{subequations}
where the solution $\bm u\in\mathbb{R}^m$ depends on time $t\in\mathbb{R}^+$, spatial position $\bm{x}\in D\subseteq \mathbb{R}^d$ as well as a vector of random variables $\bm{\xi}\in\Theta\subseteq\mathbb{R}^p$ with given probability density functions $f_{\Xi,i}(\xi_i)$ for $i = 1,\cdots,p$. Hence, the probability density function of $\bm{\xi}$ is $f_{\Xi}(\bm\xi):=\prod_{i=1}^p f_{\Xi,i}(\xi_i)$. The physical flux is given by $\bm{f}:\mathbb{R}^m\to\mathbb{R}^{d\times m}$. To simplify notation, we assume that only the initial condition is random, i.e. $\bm{\xi}$ enters through the definition of $\bm{u}_{IC}$. Equations \eqref{eq:hyperbolicProblem} are usually supplemented with boundary conditions, which we will specify later for the individual problems.

Due to the randomness of the solution, one is interested in determining the expectation value or the variance, i.e.
\begin{align*}
\text{E}[\bm{u}] = \langle \bm{u} \rangle,\qquad \text{Var}[\bm{u}] = \langle \left( \bm{u}-\text{E}[\bm{u}]\right)^2\rangle,
\end{align*}
where we use the bracket operator $\langle \cdot \rangle := \int_{\Theta} \cdot f_{\Xi}(\bm\xi)d\xi_1 \cdots d\xi_p$. To approximate quantities of interest (such as expectation value, variance or higher order moments), the solution is spanned with a set of polynomial basis functions $\varphi_{i}:\Theta\to\mathbb{R}$ such that for the multi-index $i = (i_1,\cdots,i_p)$ we have $|i| \leq M$. Note that this yields
\begin{align}\label{eq:numberBasisFcts}
N:=\begin{pmatrix}
M+p \\ p
\end{pmatrix}
\end{align}
basis functions when defining $|i|:=\sum_{n = 1}^p |i_n|$. Commonly, these functions are chosen to be orthonormal polynomials \cite{wiener1938homogeneous} with respect to the probability function, i.e. $\langle \varphi_i \varphi_j \rangle =\prod_{n=1}^p\delta_{i_nj_n}$. The generalized polynomial chaos (gPC) expansion \cite{xiu2002wiener} approximates the solution by
\begin{align}\label{eq:SGClosure}
\mathcal{U}(\bm{\hat u};\bm\xi):= \sum_{|i|\leq M} \bm{\hat{u}}_i\varphi_i(\bm{\xi}) = \hat{\bm u}^T\bm{\varphi}(\bm\xi),
\end{align}
where the deterministic expansion coefficients $\bm{\hat{u}}_i\in\mathbb{R}^m$ are called moments. To allow a more compact notation, we collect the $N$ moments for which $\vert i \vert \leq M$ holds in the moment matrix $\hat{\bm u}:=(\bm{\hat{u}}_i)_{|i|\leq M}\in\mathbb{R}^{N\times m}$ and the corresponding basis functions in $\bm{\varphi}:=(\varphi_i)_{|i|\leq M}\in\mathbb{R}^{N}$. In the following, the dependency of $\mathcal{U}$ on $\bm \xi$ will occasionally be omitted for sake of readability. The solution ansatz \eqref{eq:SGClosure} is $L^2$-optimal, if the moments are chosen to be the Fourier coefficients $\bm{\hat u}_i \equiv \langle \bm{u}\varphi_i \rangle\in\mathbb{R}^m$. One can also use the ansatz \eqref{eq:SGClosure} to compute the quantities of interest as
\begin{align*}
\text{E}[\mathcal{U}(\bm{\hat u})] = \bm{\hat u}_0,\quad \text{Var}[\mathcal{U}(\bm{\hat u})] = \text{E}[\mathcal{U}(\bm{\hat u})^2] - \text{E}[\mathcal{U}(\bm{\hat u})]^2 = \left(\sum_{i = 1}^N \hat{u}_{\ell i}^2\right)_{\ell = 1,\cdots,m}.
\end{align*}

The core idea of the stochastic-Collocation method is to compute the moments in the gPC expansion with a quadrature rule. Given a set of $Q$ quadrature weights $w_k$ and quadrature points $\bm{\xi}_k$, the moments are approximated by
\begin{align*}
\bm{\hat u}_i = \langle \bm{u}\varphi_i \rangle \approx \sum_{k = 1}^{Q}w_k \bm{u}({t,\bm{x},\bm{\xi}_k})\varphi_i(\bm{\xi}_k)f_{\Xi}(\bm{\xi}_k).
\end{align*} 
Hence, the moments can be computed by running a given deterministic solver for the original problem at each quadrature point. To reduce numerical costs in multi-dimensional settings, SC commonly uses sparse grids as quadrature rule: While tensorized quadrature sets require $O(M^p)$ quadrature points to integrate polynomials of maximal degree $M$ exactly, sparse grids are designed to integrate polynomials of total degree $M$, for which they only require $O(M(\log_2(M)^{p-1}))$ quadrature points. \remja{Brauchen wir hier noch ne Quelle?}

Intrusive methods derive a system which directly describes the time evolution of the moments: Plugging the solution ansatz \eqref{eq:SGClosure} into the set of equations \eqref{eq:hyperbolicProblem} and projecting the resulting residual to zero yields the stochastic-Galerkin moment system
\begin{subequations}\label{eq:SGMomentSystem}
\begin{align}
\partial_t \bm{\hat u}_i(t,\bm{x}) + \nabla&\cdot\langle\bm{f}(\mathcal{U}(\bm{\hat u})) \varphi_i\rangle = \bm{0}, \\
\bm{\hat u}_i(t=0,\bm{x}&) = \langle\bm{u}_{\text{IC}}(\bm{x},\cdot)\varphi_i\rangle,
\end{align}
\end{subequations}
with $|i|\leq M$. As already mentioned, the SG moment system is not necessarily hyperbolic. To ensure hyperbolicity, the IPM method uses a solution ansatz which minimizes a given entropy under a moment constraint instead of a polynomial expansion \eqref{eq:SGClosure}. For a given convex entropy $s:\mathbb{R}^m\to\mathbb{R}$ for the original problem \eqref{eq:hyperbolicProblem}, the IPM solution ansatz is given by
\begin{align}\label{eq:primalProblem}
\mathcal{U}(\bm{\hat u}) = \argmin_{\bm u} \langle s(\bm u) \rangle \enskip \text{ subject to } \bm{\hat u}_i = \langle \bm u \varphi_i \rangle \text{ for } |i| \leq M.
\end{align}
Rewritten in its dual form, \eqref{eq:primalProblem} is transformed into an unconstrained optimization problem. Defining the variables $\bm{\lambda}_i\in\mathbb{R}^m$, where $i$ is again a multi index, gives the unconstrained dual problem
\begin{align}\label{eq:dualProblem}
 \bm{\hat \lambda}(\bm{\hat u}) := \argmin_{\bm{\lambda} \in \mathbb{R}^{N \times m}}
  \left\{\langle s_*(\bm{\lambda}^T \bm\varphi)\rangle - \sum_{|i|\leq M}\bm{\lambda}_i^T \bm{\hat u}_i\right\},
\end{align}
where $s_*:\mathbb{R}^m\to\mathbb{R}$ is the Legendre transformation of $s$, and $\bm{ \hat\lambda}:=(\bm{\hat{\lambda}}_i)_{|i|\leq M}\in \mathbb{R}^{N \times m}$ are called the dual variables. The solution to \eqref{eq:primalProblem} is then given by
\begin{align}\label{eq:ansatz}
 \mathcal{U}(\bm{\hat u}) = \left( \nabla_{\bm{u}} s \right)^{-1}(\bm{\hat{\lambda}}(\bm{\hat u})^T \bm{\varphi}).
\end{align}
When plugging this ansatz into the original equations \eqref{eq:hyperbolicProblem} and projecting the resulting residual to zero again yields the moment system \eqref{eq:SGMomentSystem}, but with the ansatz \eqref{eq:ansatz} instead of \eqref{eq:SGClosure}.




\section{Discretization of the IPM system}
\label{sec:framework}
\subsection{Finite Volume Discretization}
In the following, we discretize the moment system in space and time according to \cite{kusch2017maximum}. Due to the fact, that stochastic-Galerkin can be interpreted as IPM with a quadratic entropy, it suffices to only derive a discretization of the IPM moment system. Hence, we discretize the system \eqref{eq:SGMomentSystem} with the more general IPM solution ansatz \eqref{eq:ansatz}.  
Omitting initial conditions and assuming a one-dimensional spatial domain, we can write the IPM system  as
\begin{align*}
\partial_t \bm{\hat u}+\partial_x \bm{F}(\bm{\hat u}) = \bm{0}
\end{align*}
with the flux $\bm{F}:\mathbb{R}^{N\times m}\to\mathbb{R}^{N\times m}$, $\bm{F}(\bm{\hat u})=\langle \bm f(\mathcal{U}(\bm{\hat u}))\bm{\varphi}^T \rangle^T$. Note that the inner transpose represents a dyadic product and therefore the outer transpose is applied to a matrix. Due to hyperbolicity of the IPM moment system, one can use a finite-volume method to approximate the time evolution of the IPM moments. We choose the discrete unknowns which represent the solution to be the spatial averages over each cell at time $t_n$, given by
\begin{align*}
\bm{\hat u}_{ij}^n \simeq \frac{1}{\Delta x}\int_{x_{j-1/ 2}}^{x_{j+1/ 2}}\bm{\hat u}_i(t_n,x) dx.
\end{align*}
If a moment vector in cell $j$ at time $t_n$ is denoted as $\bm{\hat u}_j^n = (\bm{\hat u}_{0j}^n,\cdots,\bm{\hat u}_{Nj}^n)^T\in\mathbb{R}^{N\times m}$ \remja{0...N -> N+1}, the finite-volume scheme can be written in conservative form with the numerical flux $\bm{G}:\mathbb{R}^{N\times m}\times\mathbb{R}^{N\times m}\to\mathbb{R}^{N\times m}$ as
\begin{align}\label{eq:IPMDiscretization}
\bm{\hat u}_{j}^{n+1} = \bm{\hat u}_{j}^{n}  - \frac{\Delta t}{\Delta x}\left( \bm{G}(\bm{\hat u}_{j}^{n},\bm{\hat u}_{j+1}^{n})- \bm{G}(\bm{\hat u}_{j-1}^{n},\bm{\hat u}_{j}^{n})\right)
\end{align}
for $j = 1,\cdots,N_x$ and $n = 0,\cdots,N_t$. Here, the number of spatial cells is denoted by $N_x$ and the number of time steps by $N_t$.
The numerical flux is assumed to be consistent, i.e. $\bm{G}(\bm{\hat{u}},\bm{\hat{u}})=\bm{F}(\bm{\hat{u}})$.

When a consistent numerical flux $\bm g:\mathbb{R}^m\times\mathbb{R}^m\to\mathbb{R}^m$, $\bm g = \bm g(\bm u_\ell, \bm u_r)$ is available for the original problem \eqref{eq:hyperbolicProblem}, then for the IPM system we can simply take the numerical flux
\begin{align*}
 \bm{\tilde G}(\bm{\hat u}_{j}^n,\bm{\hat u}_{j+1}^{n}) = \langle \bm g(\mathcal{U}(\bm{\hat u}_j^n),\mathcal{U}(\bm{\hat u}_{j+1}^n))\bm{\varphi}^T\rangle^T
\end{align*}
in \eqref{eq:IPMDiscretization}. Commonly, this integral cannot be evaluated analytically and therefore needs to be approximated by a quadrature rule
\begin{align*}
\langle h \rangle \approx \langle h \rangle_{Q} := \sum_{k=1}^Q w_k h(\bm{\xi}_k)f_{\Xi}(\bm{\xi}_k).
\end{align*}
The approximated numerical flux then becomes
\begin{align}\label{eq:numericalFluxIPM}
 \bm{G}(\bm{\hat u}_{j}^n,\bm{\hat u}_{j+1}^{n}) = \langle \bm g(\mathcal{U}(\bm{\hat u}_j^n),\mathcal{U}(\bm{\hat u}_{j+1}^n))\bm{\varphi}^T\rangle^T_Q.
\end{align}
Note that the numerical flux requires evaluating the ansatz $\mathcal{U}(\bm{\hat u}_j^n)$. To simplify notation, we define $\bm{u}_{s}:\mathbb{R}^p \to \mathbb{R}^p$,
\begin{align*}
\bm{u}_{s}(\bm\Lambda):=\left( \nabla_{\bm{u}} s \right)^{-1}(\bm\Lambda),
\end{align*}
meaning that the IPM ansatz \eqref{eq:ansatz} at cell $j$ in timestep $n$ can be written as
\begin{align*}
\mathcal{U}(\bm{\hat u}_j^n) = \bm{u}_{s}(\bm{\hat{\lambda}}(\bm{\hat u}_j^n)^T \bm{\varphi}).
\end{align*}
The computation of the dual variables $\bm{\hat\lambda}_j^n:=\bm{\hat\lambda}(\bm{\hat u}_j^n)$ requires solving the dual problem \eqref{eq:dualProblem} for the moment vector $\bm{\hat u}_{j}^{n}$. Hence, to determine the dual variables for a given moment vector $\bm{\hat{u}}$, the cost function
\begin{align}\label{eq:L}
L(\bm{\lambda};\bm{\hat{u}}) := \langle s_*(\bm{\lambda}^T \bm\varphi)\rangle_Q - \sum_{i\leq M}\bm{\lambda}_i^T \bm{\hat u}_i
\end{align}
needs to be minimized. Hence, one needs to find the root of
\begin{align*}
\nabla_{\bm{\lambda_i}}L(\bm{\lambda};\bm{\hat{u}}) = \langle \nabla s_*(\bm{\lambda}^T \bm\varphi)\bm\varphi^T\rangle_Q^T - \bm{\hat u}_i = \langle \bm u_s(\bm{\lambda}^T \bm\varphi)\bm\varphi^T\rangle_Q^T - \bm{\hat u}_i,
\end{align*}
where we used $\nabla s_* \equiv \bm u_s$. The root is usually determined by using Newton's method. For simplicity, let us define the full gradient of the Lagrangian to be $\nabla_{\bm{\lambda}}L(\bm{\lambda};\bm{\hat{u}})\in\mathbb{R}^{N\cdot m}$, i.e. we store all entries in a vector. Newton's method uses the iteration function $\bm{d}:\mathbb{R}^{N\times m}\times\mathbb{R}^{N\times m}\to\mathbb{R}^{N\times m}$,
\begin{align}\label{eq:dualIterationFunction}
\bm{d}(\bm{\lambda},\bm{\hat{u}}):= \bm{\lambda}-\bm{H}(\bm{\lambda})^{-1}\cdot\nabla_{\bm{\lambda}}L(\bm{\lambda};\bm{\hat{u}}),
\end{align}
where $\bm H\in\mathbb{R}^{N \cdot n\times N\cdot m}$ is the Hessian of \eqref{eq:L}, given by
\begin{align*}
\bm{H}(\bm{\lambda}) := \langle \nabla \bm{u}_{s} (\bm{\lambda}^T\bm{\varphi})\otimes\bm{\varphi}\bm{\varphi}^T\rangle_Q^{T}.
\end{align*}
%Inside the Newton update \eqref{eq:dualIterationFunction}, we abuse notation for better readability by making use of
%\begin{align*}
%\left(\bm{H}^{-1}\cdot\nabla_{\bm{\lambda}}L\right)_{ij} := \sum_{i' = 1}^N\sum_{j' = 1}^m \left(\bm{H}^{-1}\right)_{m(j-1) + i,m(j'-1) + i'}\cdot\nabla_{\bm{\lambda}}L_{i'j'}.
%\end{align*}
The function $\bm d$ will in the following be called dual iteration function. Now, the Newton iteration $l$ \remja{l ist später auch level für adIPM und left flux - habs erstmal überall auf zeta geändert} for spatial cell $j$ is given by
\begin{align}\label{eq:dualIteration1}
\bm{\lambda}^{(l+1)}_j = \bm{d}(\bm{\lambda}_j^{(l)},\bm{\hat{u}_j}).
\end{align}
The exact dual state is then obtained by computing the fixed point of $\bm{d}$, meaning that one converges the iteration \eqref{eq:dualIteration1}, i.e. $\bm{\hat\lambda}_j^n:=\bm{\hat\lambda}(\bm{\hat u_j^n})=\lim_{l\rightarrow\infty}\bm{d}(\bm{\lambda}_j^{(l)},\bm{\hat{u}_j^n})$.
To obtain a finite number of iterations for the iteration in cell $j$, the stopping criterion 
\begin{align}\label{eq:tauCrit}
\sum_{i=0}^m\left\Vert \nabla_{\bm{\lambda_i}}L(\bm{\lambda}_j^{(l)};\bm{\hat{u}}_j^n) \right\Vert < \tau
\end{align}
is used.

We now write down the entire scheme: To obtain a more compact notation, we define
\begin{align}\label{eq:momentIterationFunction}
\bm{c}\left(\bm{\lambda}_{\ell},\bm{\lambda}_c,\bm{\lambda}_r\right):= \langle \bm u_{s}(\bm{\lambda}_c^T\bm{\varphi})\bm{\varphi}^T\rangle_Q^T - \frac{\Delta t}{\Delta x}\left(\langle \bm g(\bm u_{s}(\bm{\lambda}_c^T\bm{\varphi}),\bm u_{s}(\bm{\lambda}_r^T\bm{\varphi}))\bm{\varphi}^T\rangle_Q^T-\langle \bm g(\bm u_{s}(\bm{\lambda}_{\ell}^T\bm{\varphi}),\bm u_{s}(\bm{\lambda}_c^T\bm{\varphi}))\bm{\varphi}^T\rangle_Q^T\right).
\end{align}
The moment iteration is then given by
\begin{align}\label{eq:momentIteration}
\bm{\hat u}_j^{n+1} = \bm{c}\left(\bm{\hat\lambda}(\bm{\hat u}_{j-1}^n),\bm{\hat\lambda}(\bm{\hat u}_{j}^n),\bm{\hat\lambda}(\bm{\hat u}_{j+1}^n)\right),
\end{align}
where the map from the moment vector to the dual variables, i.e. $\bm{\lambda}(\bm{\hat u}_{j}^n)$, is obtained by iterating
\begin{align}\label{eq:dualIteration}
\bm{\lambda}_j^{(l+1)} = \bm{d}(\bm{\lambda}_{j}^{(l)};\bm{\hat u}_j^{n}).
\end{align}
until condition \eqref{eq:tauCrit} is fulfilled. This gives Algorithm \ref{alg:IPM}.

\begin{algorithm}[H]
\begin{algorithmic}[1]
\For{$j=0$ to $N_x+1$}
\State $\bm{u}_j^0 \leftarrow \frac{1}{\Delta x} \int_{x_{j-1/ 2}}^{x_{j+1/ 2}} \langle u_{\text{IC}}(x, \cdot) \bm{\varphi} \rangle_Q dx$
\EndFor
\For{$n=0$ to $N_t$}
\For{$j=0$ to $N_x+1$}
\State $\bm{\lambda}_j^{(0)} \leftarrow \bm{\hat \lambda}_j^{n}$
\While{\eqref{eq:tauCrit} is violated}
\State $\bm{\lambda}_j^{(l+1)} \leftarrow \bm{d}(\bm{\lambda}_{j}^{(l)};\bm{\hat u}_j^{n})$
\State $l \leftarrow l+1$
\EndWhile
\State $\bm{\hat \lambda}_j^{n+1} \leftarrow \bm{\lambda}_j^{(l)}$
\EndFor
\For{$j=1$ to $N_x$}
\State $\bm{\hat u}_j^{n+1} \leftarrow \bm{c}(\bm{\hat \lambda}_{j-1}^{n+1},\bm{\hat \lambda}_j^{n+1},\bm{\hat \lambda}_{j+1}^{n+1})$
\EndFor
\EndFor
\end{algorithmic}
\caption{IPM algorithm}
\label{alg:IPM}
\end{algorithm}

\subsection{Properties of the kinetic flux}
\label{sec:costNumFlux}

A straight-forward implementation is ensured by the choice of the numerical flux \eqref{eq:numericalFluxIPM}. This choice of the numerical flux is common in the field of transport theory, where it is called the \textit{kinetic flux}. By simply taking moments of a given numerical flux for the deterministic problem, the method can easily be applied to various physical problems whenever an implementation of $\bm g = \bm g(\bm u_\ell, \bm u_r)$ is available. Therefore, we call the proposed numerical method \textit{semi-intrusive}.

Intrusive numerical methods which compute arising integrals analytically and therefore directly depend on the moments (i.e. they do not necessitate the evaluation of the gPC expansion on quadrature points) can be constructed by performing a gPC expansion on the system flux directly \cite{debusschere2004numerical}. Examples can be found in \cite{hu2015stochastic,hu2016stochastic,tryoen2010instrusive,durrwachterahigh} for the computation of numerical fluxes and sources. While the analytic computation of arising integrals is not always more efficient \cite[Section 6]{ghanem1998stochastic}, it can also complicate recycling a deterministic solver. See \ref{app:costNumFlux} for a comparison of numerical costs when using Burgers' equation. However, when not using a quadratic entropy in the IPM method or when the physical flux of the deterministic problem is not a polynomial, it is not clear how many quadrature points the numerical quadrature rule requires to guarantee a sufficiently small quadrature error. We will study the approximation properties of IPM with different quadrature orders in Section~\ref{sec:resultsNACA1D}.
\section{Collocation accelerated IPM for steady state problems}
\label{sec:collIPM}
\section{One-Shot IPM}
\label{sec:OneShotIPM}

In the following section we only consider steady state problems, i.e. we have
\begin{align}\label{eq:hyperbolicProblemSteady}
\nabla\cdot\bm{f}(\bm{u}(\bm{x},\bm{\xi})) = \bm{0} \enskip \text{ in } D
\end{align}
with adequate boundary conditions. A general strategy for computing the steady state solution to \eqref{eq:hyperbolicProblemSteady} is to introduce a pseudo-time and numerically treat \eqref{eq:hyperbolicProblemSteady} as an unsteady problem. A steady state solution is then obtained by iterating in pseudo-time until the solution remains constant. It is important to point out that the time it takes to converge to a steady state solution is crucially affected by the chosen initial condition and its distance to the steady state solution.
Similar to the unsteady case \eqref{eq:hyperbolicProblem}, we can again derive a moment system for \eqref{eq:hyperbolicProblemSteady} given by
\begin{align}\label{eq:MomentSystemSteady}
\nabla\cdot\langle\bm{f}(\bm{u}(\bm{x},\bm{\xi}))\bm{\varphi}^T\rangle^T = \bm{0} \enskip \text{ in } D
\end{align}
which is again needed for the construction of intrusive methods. By introducing a pseudo-time and using the IPM closure, we obtain the same system as in \eqref{eq:SGMomentSystem}, i.e. Algorithm \ref{alg:IPM} can be used to iterate to a steady state solution. Note that now, the time iteration is not performed for a fixed number of time steps $N_t$, but until the condition
\begin{align}\label{eq:residualSteady}
\sum_{j = 1}^{N_x} \Delta x_j \Vert \bm{\hat{u}}_j^n - \bm{\hat{u}}_j^{n-1} \Vert \leq \varepsilon
\end{align}
is fulfilled. Since one is generally interested in low order moments such as the expectation value, this residual can be modified by only accounting for the zero order moments.

In this section we aim at breaking up the inner loop in the IPM Algorithm \ref{alg:IPM}, i.e. to just perform one iteration of the dual problem in each time step. Consequently, the IPM reconstruction given by \eqref{eq:primalProblem} is not done exactly, meaning that the reconstructed solution does not minimize the entropy while not fulfilling the moment constraint. However, the fact that the moment vectors are not yet converged to the steady solution seems to permit such an inexact reconstruction. Hence, we aim at iterating the moments to steady state and the dual variables to the exact solution of the IPM optimization problem \eqref{eq:primalProblem} simultaneously.
By successively performing one update of the moment iteration and one update of the dual iteration, we obtain 
\begin{subequations}\label{eq:oneshotIPM}
\begin{align}
&\bm{\lambda}_{j}^{n+1} =  \bm{d}(\bm{\lambda}_j^{n},\bm{u}_j^{n}) \enskip \text{ for all j} \label{eq:oneshotIPMdual}\\
&\bm{u}_j^{n+1} =  \bm{c}\left(\bm{\lambda}_{j-1}^{n+1},\bm{\lambda}_{j}^{n+1},\bm{\lambda}_{j+1}^{n+1}\right)\label{eq:oneshotIPMmoment}.
\end{align}
\end{subequations}
This yields Algotihm \ref{alg:osIPM}.
\begin{algorithm}[H]
\begin{algorithmic}[1]
\For{$j=0$ to $N_x+1$}
\State $\bm{u}_j^0 \leftarrow \frac{1}{\Delta x} \int_{x_{j-1/ 2}}^{x_{j+1/ 2}} \langle u_{\text{IC}}(x, \cdot) \bm{\varphi} \rangle_Q dx$
\EndFor
\While{\eqref{eq:residualSteady} is violated}
\For{$j=1$ to $N_x$}
\State $\bm{\lambda}_j^{n+1} \leftarrow \bm{d}(\bm{\lambda}_{j}^{n};\bm{\hat u}_j^{n})$
\State $\bm{\hat u}_j^{n+1} \leftarrow \bm{c}(\bm{\lambda}_{j-1}^{n+1},\bm{\lambda}_j^{n+1},\bm{\lambda}_{j+1}^{n+1})$
\EndFor
\State $n \leftarrow n+1$
\EndWhile
\end{algorithmic}
\caption{One-Shot IPM implementation}
\label{alg:osIPM}
\end{algorithm}
We call this method One-Shot IPM, since it is inspired by One-Shot optimization, see for example \cite{hazra2005aerodynamic}, which uses only a single iteration of the primal and dual step in order to update the design variables. Note that the dual variables from the One-Shot iteration are written without a hat to indicate that they are not the exact solution of the dual problem.

In the following, we will show that this iteration converges, if the chosen initial condition is sufficiently close to the steady state solution. For this we take an approach commonly chosen to prove local convergence properties of Newton's method: In Theorem \ref{th:Contractive}, we show that the iteration function is contractive at its fixed point and conclude in Theorem \ref{th:localConvergence} that this yields local convergence: \frama{Nico meinte wir sollten hier in die Voraussetzung packen, dass wir nah genug an der richtigen Loesung sind (Einzugsbereich des Newton) um das Wort 'local' zu vermeiden. Manchmal stören sich die Reviewer daran.}
\begin{theorem}\label{th:Contractive}
Assume that the classical IPM iteration is contractive at its fixed point $\bm{\hat u}^*$. Then the Jacobi matrix $\bm{J}$ of the One-Shot IPM iteration \eqref{eq:oneshotIPM} has a spectral radius $\rho(\bm{J})<1$ at the fixed point $(\bm{\lambda}^*,\bm{\hat u}^*)$.
\end{theorem}
\begin{proof}
First, to understand what contraction of the classical IPM iteration implies, we rewrite the moment iteration \eqref{eq:momentIteration} of the classical IPM scheme: When defining the update function
\begin{align*}
\bm{\tilde c}\left(\bm{\hat{u}}_{\ell},\bm{\hat{u}}_{c},\bm{\hat{u}}_{r}\right):=\bm{c}\left(\bm{\hat{\lambda}}(\bm{\hat{u}}_{\ell}),\bm{\hat{\lambda}}(\bm{\hat{u}}_{c}),\bm{\hat{\lambda}}(\bm{\hat{u}}_{r})\right)
\end{align*}
we can rewrite the classical moment iteration as
\begin{align}\label{eq:shortIPMIt}
\bm{\hat u}_j^{n+1} = \bm{\tilde c}\left(\bm{\hat u}_{j-1}^n,\bm{\hat u}_{j}^n,\bm{\hat u}_{j+1}^n\right).
\end{align}
Since we assume that the classical IPM scheme is contractive at its fixed point, we have $\rho (\nabla_{\bm{\hat u}}\bm{\tilde c}(\bm{\hat u}^*))<1$ with $\nabla_{\bm{\hat u}}\bm{\tilde c}\in\mathbb{R}^{N\cdot N_x\times N\cdot N_x}$ defined by
\begin{align*}
\nabla_{\bm{\hat u}}\bm{\tilde c} = 
\begin{pmatrix} 
    \partial_{\bm{\hat u}_c}\bm{\tilde c}_{1} & \partial_{\bm{\hat u}_r}\bm{\tilde c}_{1}& 0 & 0 & \dots \\
    \partial_{\bm{\hat u}_{\ell}}\bm{\tilde c}_{2} & \partial_{\bm{\hat u}_c}\bm{\tilde c}_{2} & \partial_{\bm{\hat u}_r}\bm{\tilde c}_{2}& 0 & \dots \\
    0 & \partial_{\bm{\hat u}_{\ell}}\bm{\tilde c}_{3} & \partial_{\bm{\hat u}_c}\bm{\tilde c}_{3} & \partial_{\bm{\hat u}_r}\bm{\tilde c}_{3}\\
    \vdots & & & \ddots & \\
    0 &\cdots &  0 & \partial_{\bm{\hat u}_{\ell}}\bm{\tilde c}_{N_x} & \partial_{\bm{\hat u}_c}\bm{\tilde c}_{N_x}
    \end{pmatrix},
\end{align*}
where we define $\bm{\tilde c}_{j}:=\bm{\tilde c}\left(\bm{\hat u}_{j-1}^*,\bm{\hat u}_{j}^*,\bm{\hat u}_{j+1}^*\right)$ for all $j$. Now for each term inside the matrix $\nabla_{\bm{\hat u}}\bm{\tilde c}$ we have 
\begin{align}\label{eq:cTildeDer}
\partial_{\bm{\hat u}_{\ell}}\bm{\tilde c}_{j} = \frac{\partial \bm{c}_j}{\partial \bm{\hat \lambda}_{\ell}}\frac{\partial \bm{\hat \lambda}(\bm{\hat u}_{j-1}^*)}{\partial \bm{\hat u}},\enskip\partial_{\bm{\hat u}_c}\bm{\tilde c}_{j} = \frac{\partial \bm{c}_j}{\partial \bm{\hat \lambda}_c}\frac{\partial \bm{\hat \lambda}(\bm{\hat u}_j^*)}{\partial \bm{\hat u}},\enskip\partial_{\bm{\hat u}_r}\bm{\tilde c}_{j} = \frac{\partial \bm{c}_j}{\partial \bm{\hat \lambda}_r}\frac{\partial \bm{\hat \lambda}(\bm{\hat u}_{j+1}^*)}{\partial \bm{\hat u}}.
\end{align}
We first wish to understand the structure of the terms $\partial_{\bm{\hat u}} \bm{\hat \lambda}(\bm{\hat u})$. For this, we note that the exact dual variables fulfill
\begin{align}\label{eq:ulambda}
\bm{\hat u} = \langle \bm{u}_s(\bm{\hat \lambda}^T\bm{\varphi})\bm{\varphi}^T\rangle =: \bm{h}(\bm{\hat \lambda}),
\end{align}
which is why we have the mapping $\bm{\hat u}:\mathbb{R}^{N\times m}\to\mathbb{R}^{N\times m}$, $\bm{\hat u}(\bm{\hat \lambda}) = \bm{h}(\bm{\hat \lambda})$. Since the solution of the dual problem for a given moment vector is unique, this mapping is bijective and therefore we have an inverse function
\begin{align}\label{eq:lambdau}
\bm{\hat \lambda} = \bm{h}^{-1}(\bm{\hat u}(\bm{\hat \lambda})).
\end{align}
Now we differentiate both sides w.r.t. $\bm{\hat \lambda}$ to get
\begin{align*}
\bm{I}_{d} = \frac{\partial \bm{h}^{-1}(\bm{\hat u}(\bm{\hat \lambda}))}{\partial \bm{\hat u}}\frac{\partial \bm{\hat u}(\bm{\hat \lambda})}{\partial \bm{\hat \lambda}}.
\end{align*}
We multiply with the matrix inverse of $\frac{\partial \bm{\hat u}(\bm{\hat \lambda})}{\partial \bm{\hat \lambda}}$ to get
\begin{align*}
\left(\frac{\partial \bm{\hat u}(\bm{\hat \lambda})}{\partial \bm{\hat \lambda}}\right)^{-1} = \frac{\partial \bm{h}^{-1}(\bm{\hat u}(\bm{\hat \lambda}))}{\partial \bm{\hat u}}.
\end{align*}
Note that on the left-hand-side we have the inverse of a matrix and on the right-hand-side, we have the inverse of a multi-dimensional function. By rewriting $\bm{h}^{-1}(\bm{\hat u}(\bm{\hat \lambda}))$ as $\bm{\hat \lambda}(\bm{\hat u})$ and simply computing the term $\frac{\partial \bm{\hat u}(\bm{\hat \lambda})}{\partial \bm{\hat \lambda}}$ by differentiating \eqref{eq:ulambda} w.r.t. $\bm{\hat \lambda}$, one obtains
\begin{align}\label{eq:dudlambdaex}
\partial_{\bm{\hat u}} \bm{\hat \lambda}(\bm{\hat u}) = \langle \nabla\bm{u}_s(\bm{\hat \lambda}^T\bm{\varphi})\bm{\varphi}\bm{\varphi}^T\rangle^{-T}.
\end{align}
Now we begin to derive the spectrum of the \textit{One-Shot IPM} iteration \eqref{eq:oneshotIPM}. Note that in its current form this iteration is not really a fixed point iteration, since it uses the time updated dual variables in \eqref{eq:oneshotIPMmoment}. To obtain a fixed point iteration, we plug the dual iteration step \eqref{eq:oneshotIPMdual} into the moment iteration \eqref{eq:oneshotIPMmoment} to obtain
\begin{align*}
&\bm{\lambda}_j^{n+1} = \bm{d}(\bm{\lambda}_j^{n},\bm{\hat u}_j^{n}) \enskip \text{ for all j} \\
&\bm{\hat u}_j^{n+1} =  \bm{c}\left(\bm{d}(\bm{\lambda}_{j-1}^{n},\bm{\hat u}_{j-1}^{n}),\bm{d}(\bm{\lambda}_{j}^{n},\bm{\hat u}_{j}^{n}),\bm{d}(\bm{\lambda}_{j+1}^{n},\bm{\hat u}_{j+1}^{n})\right).
\end{align*}
The Jacobian $\bm{J}\in\mathbb{R}^{2N\cdot N_x \times 2N\cdot N_x}$ has the form
\begin{align}\label{eq:Jacobian}
\bm{J} = 
\begin{pmatrix}
 \partial_{\bm{\lambda}} \bm{d} & \partial_{\bm{\hat u}} \bm{d}  \\
\partial_{\bm{\lambda}} \bm{c} & \partial_{\bm{\hat u}} \bm{c}
\end{pmatrix},
\end{align}
where each block has entries for all spatial cells. We start by looking at $\partial_{\bm{\lambda}} \bm{d}$. For the columns belonging to cell $j$, we have
\begin{align*}
\partial_{\bm{\lambda}} \bm{d}(\bm{\lambda}_j^n,\bm{\hat u}_j^n) &= \bm{I}_d - \bm{H}(\bm\lambda)^{-1} \cdot \langle \nabla\bm{u}_s(\bm{\varphi}^T\bm{\lambda}_j^n)\bm{\varphi}\bm{\varphi}^T \rangle^T - \partial_{\bm{\lambda}}\bm{H}(\bm\lambda)^{-1} \cdot \left( \langle \bm{u}_s(\bm{\varphi}^T\bm{\lambda}_j^n)\bm{\varphi}^T \rangle^T - \bm{\hat u}\right) \\
&=- \partial_{\bm{\lambda}}\bm{H}(\bm\lambda)^{-1} \cdot \left( \langle \bm{u}_s(\bm{\varphi}^T\bm{\lambda}_j^n)\bm{\varphi}^T \rangle^T - \bm{\hat u}\right).
\end{align*}
Recall that at the fixed point $(\bm{\lambda}^*,\bm{\hat u}^*)$, we have $\langle \bm{u}_s(\bm{\varphi}^T\bm{\lambda}_j^n)\bm{\varphi}^T \rangle^T = \bm{\hat u}$, hence one obtains $\partial_{\bm{\lambda}} \bm{d}=\bm{0}$. For the block $\partial_{\bm{\hat u}} \bm{d}$, we get 
\begin{align*}
\partial_{\bm{\hat u}} \bm{d}(\bm{\lambda}_j^n,\bm{\hat u}_j^n) = \bm{H}(\bm\lambda)^{-1},
\end{align*}
hence $\partial_{\bm{\hat u}} \bm{d}$ is a block diagonal matrix. Let us now look at $\partial_{\bm{\lambda}} \bm{c}$ at a fixed spatial cell $j$:
\begin{align*}
\frac{\partial \bm{c}}{\partial \bm{\lambda}_{\ell}}\frac{\partial \bm{d}(\bm{\lambda}_{j-1}^{n},\bm{\hat u}_{j-1}^{n})}{\partial \bm{\lambda}} = \bm{0},
\end{align*}
since we already showed that by the choice of $\bm{H}(\bm\lambda)^{-1}$ the term $\partial_{\bm{\lambda}} \bm{d}$ is zero. We can show the same result for all spatial cells and all inputs of $\bm{c}$ analogously, hence $\partial_{\bm{\lambda}} \bm{c} = \bm{0}$. For the last block, we have that 
\begin{align*}
\frac{\partial \bm{c}}{\partial \bm{\lambda}_{\ell}}\frac{\partial \bm{d}(\bm{\lambda}_{j-1}^{n},\bm{\hat u}_{j-1}^{n})}{\partial \bm{\hat u}} = \frac{\partial \bm{c}}{\partial \bm{\lambda}_{\ell}} \bm{H}(\bm\lambda)^{-1} = \frac{\partial \bm{c}}{\partial \bm{\lambda}_{\ell}} \langle \nabla\bm{u}_s(\bm{\varphi}^T\bm{\lambda}_{j-1}^n)\bm{\varphi}\bm{\varphi}^T \rangle^{-T} = \partial_{\bm{\hat u}_{\ell}}\bm{\tilde c}_j
\end{align*}
by the choice of $\bm{H}(\bm\lambda)^{-1}$ as well as \eqref{eq:cTildeDer} and \eqref{eq:dudlambdaex}. We obtain an analogous result for the second and third input. Hence, we have that $\partial_{\bm{\hat u}} \bm{c} = \nabla_{\bm{\hat u}}\bm{\tilde c}$, which only has eigenvalues between $-1$ and $1$ by the assumption that the classical IPM iteration is contractive. Since $\bm{J}$ is an upper triangluar block matrix, the eigenvalues are given by $\lambda\left(\partial_{\bm{\lambda}} \bm{d}\right) = 0$ and $\lambda\left(\partial_{\bm{\hat u}} \bm{c}\right)\in(-1,1)$, hence the One-Shot IPM is contractive around its fixed point.
\end{proof}
\begin{theorem}\label{th:localConvergence}
With the assumptions from Theorem \ref{th:Contractive}, the One-Shot IPM converges locally, i.e. there exists a $\delta>0$ s.t. for all starting points $(\bm{\lambda}^0,\bm{\hat u}^0)\in B_{\delta}(\bm{\lambda}^*,\bm{\hat u}^*)$ we have
\begin{align*}
\Vert (\bm{\lambda}^n,\bm{\hat u}^n) - (\bm{\lambda}^*,\bm{\hat u}^*)\Vert \rightarrow 0 \qquad \text{ for } n \rightarrow \infty.
\end{align*}
\end{theorem}
\begin{proof}
By Theorem \ref{th:Contractive}, the One-Shot scheme is contractive at its fixed point. Since we assumed convergence of the classical IPM scheme, we can conclude that all entries in the Jacobian $\bm{J}$ are continuous functions. Furthermore, the determinant of $\bm{\tilde{J}}:=\bm{J}-\lambda \bm{I}_d$ is a polynomial of continuous functions, since
\begin{align*}
\text{det}(\bm{\tilde J}) = \sum_{\sigma} \text{sgn}(\sigma)\prod_{i = 1}^{2 N_x N} \tilde J_{\sigma(i),i}.
\end{align*}
Since the roots of a polynomial vary continuously with its coefficients, the eigenvalues of $\bm{J}$ are continuous w.r.t $(\bm{\lambda},\bm{\hat u})$. Hence there exists an open ball with radius $\delta$ around the fixed point in which the eigenvalues remain in the interval $(-1,1)$.
\end{proof}
%\begin{remark}
%Since the preconditioning step of the Collocation-accelerated IPM method generates initial conditions which are close to the steady state solution, using One-Shot IPM instead of classical IPM is well suited. However, our numerical calculations show that One-Shot IPM converges even if the solution is far away from its steady state. 
%\end{remark}
\section{Results}
\label{sec:results}

\subsection{2D Euler equations with a one dimensional uncertainty}
We start by quantifying the effects of an uncertain angle of attack $\phi\sim U(0.75,1.75)$ for a NACA0012 profile computed with different methods. The stochastic Euler equations in two dimensions are given by
\begin{align*}
\partial_t
\begin{pmatrix}
\rho \\ \rho v_1 \\ \rho v_2 \\ \rho e
\end{pmatrix}
+\partial_{x_1}
\begin{pmatrix}
\rho v_1 \\ \rho v_1^2 +p \\ \rho v_1 v_2 \\  v_1 (\rho e+p)
\end{pmatrix}
+\partial_{x_2}
\begin{pmatrix}
\rho v_2 \\ \rho v_1 v_2 \\ \rho v_2^2+p \\ v_2 (\rho e+p)
\end{pmatrix}
=\bm{0}.
\end{align*}
These equations determine the time evolution of the conserved variables $(\rho,rho \bm v, rho e)$, i.e. density, momentum and energy. A closure for the pressure $p$ is given by
\begin{align*}
p = (\gamma-1)\rho\left(e-\frac12(v_1^2+v_2^2)\right).
\end{align*}
Since the the fluid of the following test cases is air, we choose the heat capacity ratio $\gamma$ to be $1.4$. The spatial mesh discretizes the flow domain around the airfoil. At the airfoil boundary $\Gamma_{0}$, we use the Euler slip condition $\bm v^T\bm n = 0$, where $\bm n$ denotes the surface normal. At a sufficiently large distance away from the airfoil, we assume a far field flow with a given Mach number $Ma = 0.8$, pressure $p = 101,325$ Pa and a temperature of $273.15$ K. Now the angle of attack $\phi$ is uniformly distributed in the interval of $[0.75,1.75]$ degrees. I.e. we choose $\phi(\xi) = 1.25 + 0.5\xi$ where $\xi\sim U(-1,1)$. As commonly done, the initial condition is equal to the far field boundary values. Consequently, the wall condition at the airfoil is violated and will correct the flow solution. The computational domain is a circle with a diameter of $40$ meters. In the center, the NACA0012 airfoil with a length of one meter is located. The discretization is composed of a coarsely refined far field and a finely resolved region around the airfoil, since we are interested in the flow solution at the airfoil. Altogether, the mesh consists of 22361 triangular elements.

The aim is to quantify the effects arising from the one-dimensional uncertainty $\xi$ with different methods. The IPM methods makes use of the acceleration strategies proposed in this work. To be able to measure the quality of the obtained solutions, we compute a reference solution using stochastic-Collocation with $100$ Gauss-Lobatto quadrature points. In the following, we investigate the L$^2$ error of the variance and the expectation value. Hence, when denoting the reference solution by $\bm u$, we have
\begin{align*}
\Vert E[\bm u] - E[\mathcal{U}] \Vert := \sqrt{\sum_{j=0}^{N_x} \Delta x_j \left( E[\bm u] - E[\mathcal{U}] \right)^2}, \enskip \Vert Var[\bm u] - Var[\mathcal{U}] \Vert := \sqrt{\sum_{j=0}^{N_x} \Delta x_j \left( Var[\bm u] - Var[\mathcal{U}] \right)^2}.
\end{align*}
The error is computed inside a box of one meter height and 1.1 meters length around the airfoil to prevent that small fluctuations in the coarsely refined far field influence the error.

In the following, we compare stochastic-Collocation with stochastic-Galerkin and IPM as well as its proposed acceleration techniques. Note that since IPM generalizes SG, all proposed techniques can be used for this method as well. For more information on the chosen entropy and the resulting solution ansatz for IPM, see \ref{app:IPM2DEuler}.

Recall that the numerical flux \eqref{eq:numericalFluxIPM} uses a quadrature rule to approximate integrals. We start by investigating the effects of this quadrature has on the solution accuracy. For this, we run the IPM method with a moment order ranging from three to seven \comment{[use IPM instead of osIPM, add accuracy of SC$^3$]} using a Clenshaw-Curtis quadrature rule with level three (i.e. 9 quadrature points) and level four (i.e. 17 quadrature points). The results are given in Figure TODO. It can be seen that the error stagnates when the chosen quadrature is not sufficiently accurate. This behavior results from aliasing effects, which dominate the accuracy level. Note that SC can be interpreted as an intrusive method which uses as many moments as quadrature points to compute the numerical flux. Therefore, we can conclude that the error of SC is dominated by aliasing effects and will not reach the same accuracy level as an intrusive method with an sufficiently accurate quadrature rule. Especially for high dimensional problems

 and compare against Collocation with $5$ collocation points as well as SG and IPM with $5$ moments and $10$ quadrature points. Furthermore, we use convergence accelerated (caIPM) as well as convergence accelerated one-shot IPM (caosIPM), which we introduced in Sections \ref{sec:collIPM} and \ref{sec:OneShotIPM} with $10$ moments and $15$ quadrature points to converge the collocation solution with $5$ quadrature points to an entropy solution with increased accuracy.

\begin{figure}[h!]
\centering
		\centering
		\includegraphics[scale=0.7]{figs/{L2_error_E[rho]osIPMIPMCol}.pdf}
		\label{fig:sub1}
	\caption{L2 error at airfoil with 5 MPI and 2 OMP threads.}
	\label{fig:Obstacles2D}
\end{figure}

\begin{figure}[h!]
\centering
		\centering
		\includegraphics[scale=0.7]{figs/{L2_error_E[rho]}.pdf}
		\label{fig:sub1}
	\caption{L2 error at airfoil with 5 MPI and 2 OMP threads.}
	\label{fig:Obstacles2D}
\end{figure}

\begin{figure}[h!]
\centering
		\centering
		\includegraphics[scale=0.7]{figs/{convergence_runtime_residual}.pdf}
		\label{fig:sub1}
	\caption{Convergence to steady state with 5 MPI and 2 OMP threads.}
	\label{fig:Obstacles2D}
\end{figure}

\newgeometry{top=1.5cm, left=1.0cm}
\begin{figure}[h!]
\centering
	\begin{subfigure}{0.33\linewidth}
		\centering
		\includegraphics[scale=0.2]{figs/expectedRho/{euler2D_nacaCoarse_sc_n5_s05_aoa}.png}
		\label{fig:sub1}
	\end{subfigure}%
	\begin{subfigure}{0.33\linewidth}
		\centering
		\includegraphics[scale=0.2]{figs/expectedRho/{euler2D_nacaCoarse_sg_n5_nq10_s05_aoa}.png}
		\label{fig:sub2}
	\end{subfigure}%
	\begin{subfigure}{0.33\linewidth}
		\centering
		\includegraphics[scale=0.2]{figs/expectedRho/{euler2D_nacaCoarse_ipm_n5_nq10_s05_aoa}.png}
		\label{fig:sub1}
	\end{subfigure}
	
	\begin{subfigure}{0.33\linewidth}
		\centering
		\includegraphics[scale=0.2]{figs/expectedRho/{euler2D_nacaCoarse_caipm_n10_nq15_s05_aoa}.png}
		\label{fig:sub2}
	\end{subfigure}%
	\begin{subfigure}{0.33\linewidth}
		\centering
		\includegraphics[scale=0.2]{figs/expectedRho/{euler2D_nacaCoarse_caosipm_n10_nq15_s05_aoa}.png}
		\label{fig:sub1}
	\end{subfigure}%
	\begin{subfigure}{0.33\linewidth}
		\centering
		\includegraphics[scale=0.2]{figs/expectedRho/{euler2D_nacaCoarse_sc_n50_s05_aoa_legend}.png}
		\label{fig:sub2}
	\end{subfigure}%
	\caption{E$[\rho]$ (from top left to bottom right) SC, SG, IPM, caIPM, caosIPM, reference solution.}
	\label{fig:Obstacles2D}
\end{figure}

\begin{figure}[h!]
\centering
	\begin{subfigure}{0.33\linewidth}
		\centering
		\includegraphics[scale=0.2]{figs/varRho/{euler2D_nacaCoarse_sc_n5_s05_aoa_var}.png}
		\label{fig:sub1}
	\end{subfigure}%
	\begin{subfigure}{0.33\linewidth}
		\centering
		\includegraphics[scale=0.2]{figs/varRho/{euler2D_nacaCoarse_sg_n5_nq10_s05_aoa_var}.png}
		\label{fig:sub2}
	\end{subfigure}%
	\begin{subfigure}{0.33\linewidth}
		\centering
		\includegraphics[scale=0.2]{figs/varRho/{euler2D_nacaCoarse_ipm_n5_nq10_s05_aoa_var}.png}
		\label{fig:sub1}
	\end{subfigure}
	
	\begin{subfigure}{0.33\linewidth}
		\centering
		\includegraphics[scale=0.2]{figs/varRho/{euler2D_nacaCoarse_caipm_n10_nq15_s05_aoa_var}.png}
		\label{fig:sub2}
	\end{subfigure}%
	\begin{subfigure}{0.33\linewidth}
		\centering
		\includegraphics[scale=0.2]{figs/varRho/{euler2D_nacaCoarse_caosipm_n10_nq15_s05_aoa_var}.png}
		\label{fig:sub1}
	\end{subfigure}%
	\begin{subfigure}{0.33\linewidth}
		\centering
		\includegraphics[scale=0.2]{figs/varRho/{euler2D_nacaCoarse_sc_n50_s05_aoa_var_legend}.png}
		\label{fig:sub2}
	\end{subfigure}%
	\caption{Var$[\rho]$ (from top left to bottom right) SC, SG, IPM, caIPM, caosIPM, reference solution.}
	\label{fig:Obstacles2D}
\end{figure}

%%% error plots %%%
\begin{figure}[h!]
\centering
	\begin{subfigure}{0.33\linewidth}
		\centering
		\includegraphics[scale=0.2]{figs/expectedRho/errors/{euler2D_nacaCoarse_sc_n5_s05_aoa_error_legend}.png}
		\label{fig:sub1}
	\end{subfigure}%
	\begin{subfigure}{0.33\linewidth}
		\centering
		\includegraphics[scale=0.2]{figs/expectedRho/errors/{euler2D_nacaCoarse_sg_n5_nq10_s05_aoa_error_legend}.png}
		\label{fig:sub2}
	\end{subfigure}%
	\begin{subfigure}{0.33\linewidth}
		\centering
		\includegraphics[scale=0.2]{figs/expectedRho/errors/{euler2D_nacaCoarse_ipm_n5_nq10_s05_aoa_error_legend}.png}
		\label{fig:sub1}
	\end{subfigure}
	
	\begin{subfigure}{0.33\linewidth}
		\centering
		\includegraphics[scale=0.2]{figs/expectedRho/errors/{euler2D_nacaCoarse_caipm_n10_nq15_s05_aoa_error_legend}.png}
		\label{fig:sub2}
	\end{subfigure}%
	\begin{subfigure}{0.33\linewidth}
		\centering
		\includegraphics[scale=0.2]{figs/expectedRho/errors/{euler2D_nacaCoarse_caosipm_n10_nq15_s05_aoa_error_legend}.png}
		\label{fig:sub1}
	\end{subfigure}
	\caption{E$[\rho]$ distance to reference solution (from top left to bottom right) SC, SG, IPM, caIPM, caosIPM.}
	\label{fig:Obstacles2D}
\end{figure}

\begin{figure}[h!]
\centering
	\begin{subfigure}{0.33\linewidth}
		\centering
		\includegraphics[scale=0.2]{figs/varRho/errors/{euler2D_nacaCoarse_sc_n5_s05_aoa_var_error_legend}.png}
		\label{fig:sub1}
	\end{subfigure}%
	\begin{subfigure}{0.33\linewidth}
		\centering
		\includegraphics[scale=0.2]{figs/varRho/errors/{euler2D_nacaCoarse_sg_n5_nq10_s05_aoa_var_error_legend}.png}
		\label{fig:sub2}
	\end{subfigure}%
	\begin{subfigure}{0.33\linewidth}
		\centering
		\includegraphics[scale=0.2]{figs/varRho/errors/{euler2D_nacaCoarse_ipm_n5_nq10_s05_aoa_var_error_legend}.png}
		\label{fig:sub1}
	\end{subfigure}
	
	\begin{subfigure}{0.33\linewidth}
		\centering
		\includegraphics[scale=0.2]{figs/varRho/errors/{euler2D_nacaCoarse_caipm_n10_nq15_s05_aoa_var_error_legend}.png}
		\label{fig:sub2}
	\end{subfigure}%
	\begin{subfigure}{0.33\linewidth}
		\centering
		\includegraphics[scale=0.2]{figs/varRho/errors/{euler2D_nacaCoarse_caosipm_n10_nq15_s05_aoa_var_error_legend}.png}
		\label{fig:sub1}
	\end{subfigure}
	\caption{Var$[\rho]$ distance to reference solution (from top left to bottom right) SC, SG, IPM, caIPM, caosIPM.}
	\label{fig:Obstacles2D}
\end{figure}
\newgeometry{top=2.5cm, bottom=2.5cm}
\section{Summary and outlook}
\label{sec:summary_outlook}

In future work, we aim at further accelerating the IPM method by using non-exact Hessian approximations. Similar to the one-shot idea of not fully converging the dual problem, it seems to be plausible to not spend too much time on computing the Hessian when the moments are not close to steady state. Hessian approximations that can be interesting are BFGS and sparse BFGS, which construct the Hessian from previously computed gradients. Note that this strategy will increase the used memory, since old Hessians or gradients from a certain number of old time steps need to be saved in every spacial cell.
Even though the non-intrusive nature of stochastic-Collocation or Monte Carlo methods allows an easy implementation, it can be important to intrusively modify the code in order to fully exploit all acceleration potentials. Synchronizing the time updates of the solution at different quadrature points yields an increased control over the solution during the computation, which can for example be uses to employ adaptive methods. In this case one can switch to a fine quadrature level in a certain spatial cell by for example computing moments with the given coarse set of collocation points. From these moments one can compute an IPM reconstruction, which one can then evaluate at a finer quadrature set. Another example of breaking up the non-intrusive nature of Monte Carlo methods can be found in \cite{poette2019gpc}, where the generation of random samples is combined with the sampling after collisions to increase efficiency.

%\section{Methods of Uncertainty Quantification}
%\subsection{Collocation}
%\subsection{Stochastic Galerkin}
%\subsection{Intrusive Polynomial Moment Method}
%\section{Method discussion}
%\begin{itemize}
%	\item kleinere Fehler [11]
%	\item Programmieraufwand [31]
%	\item zweiabhängig
%	\item adaptiv (breack up black box approach )
%	\item grobe Auflösung für Momente 
%\end{itemize}
%\section{UQCreator}
%\subsection{Adaptivity}
%\subsection{restart IPM} %relay(-strategy) IPM?
%\subsection{one-shot IPM}
%\subsection{adaptive IPM}
%\section{Results}
%\subsection{2D Euler flow over NACA0012}
%\begin{itemize}
%	\item AoA study
%	\item AoA + Ma number + mehr ?
%\end{itemize}
%\subsection{2D shallow water dam break}

\bibliographystyle{unsrt}  
\bibliography{references}
\end{document}
