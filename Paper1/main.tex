\documentclass[3p]{elsarticle}


% Packages and macros go here
\usepackage{bm}
\usepackage[parfill]{parskip}
\usepackage[utf8]{inputenc}
\usepackage{url}
\usepackage{amsmath,amssymb,amsfonts,amsthm}
\usepackage{color}
\usepackage{bm}
\usepackage{graphicx}
%\usepackage{subfig}
\usepackage{subcaption}
%\usepackage[]{algorithm2e}
%\usepackage{algorithmic}
%\usepackage{algorithmicx}
\usepackage[linesnumbered,ruled]{algorithm2e}

\usepackage{lipsum}
\usepackage{amsfonts}
\usepackage{graphicx}
\usepackage{epstopdf}
\usepackage{algorithmic}
\ifpdf
  \DeclareGraphicsExtensions{.eps,.pdf,.png,.jpg}
\else
  \DeclareGraphicsExtensions{.eps}
\fi

\newtheorem{theorem}{Theorem}
\newtheorem{proposition}{Proposition}

\DeclareMathOperator*{\argmin}{arg\,min}

\makeatletter
\newsavebox\myboxA
\newsavebox\myboxB
\newlength\mylenA

\newcommand*\xoverline[2][0.75]{%
    \sbox{\myboxA}{$\m@th#2$}%
    \setbox\myboxB\null% Phantom box
    \ht\myboxB=\ht\myboxA%
    \dp\myboxB=\dp\myboxA%
    \wd\myboxB=#1\wd\myboxA% Scale phantom
    \sbox\myboxB{$\m@th\overline{\copy\myboxB}$}%  Overlined phantom
    \setlength\mylenA{\the\wd\myboxA}%   calc width diff
    \addtolength\mylenA{-\the\wd\myboxB}%
    \ifdim\wd\myboxB<\wd\myboxA%
       \rlap{\hskip 0.5\mylenA\usebox\myboxB}{\usebox\myboxA}%
    \else
        \hskip -0.5\mylenA\rlap{\usebox\myboxA}{\hskip 0.5\mylenA\usebox\myboxB}%
    \fi}
\makeatother

\journal{arXiv.org}

%strongly recommended
%\numberwithin{theorem}{section}

% Declare title and authors, without \thanks
\newcommand{\TheTitle}{A semi-intrusive Code Framework for Uncertainty Quantification} 

\date{\today}

% Sets running headers as well as PDF title and authors
%\headers{\RunningTitle}{\TheAuthors}


\usepackage{amsopn}
\DeclareMathOperator{\diag}{diag}
%\DeclareMathOperator{\argmin}{arg\min}

% FundRef data to be entered by SIAM
%<funding-group>
%<award-group>
%<funding-source>
%<named-content content-type="funder-name"> 
%</named-content> 
%<named-content content-type="funder-identifier"> 
%</named-content>
%</funding-source>
%<award-id> </award-id>
%</award-group>
%</funding-group>

\def\lambdabar{\xoverline{\bm{\lambda}}}
\def\lambdabarjn{\lambdabar_j^n}

\newcommand{\commentRyan}[1]{{\Large{\textcolor{red}{\newline[RYAN: #1]\newline}}}}
\newcommand{\commentJonas}[1]{{\Large{\textcolor{blue}{\newline[JONAS: #1]\newline}}}}

\begin{document}

\begin{frontmatter}

\title{\TheTitle}


\author[adressJonas]{Jonas Kusch}
\author[adressJannick]{Jannick Wolters}
\author[adressMartin]{Martin Frank}

\address[adressJonas]{Karlsruhe Institute of Technology, Karlsruhe,
    jonas.kusch@kit.edu}
\address[adressJannick]{Karlsruhe Institute of Technology, Karlsruhe, jannick.wolters@kit.edu}
\address[adressMartin]{Karlsruhe Institute of Technology, Karlsruhe, martin.frank@kit.edu}


\begin{abstract}
Methods for quantifying the effects of uncertainties in hyperbolic problems can be divided into intrusive and non-intrusive techniques. Intrusive methods require the implementation of a new code and yield 


\end{abstract}

\begin{keyword}
conservation laws, hyperbolic, intrusive, UQ, IPM, SG, Collocation
\end{keyword}

\end{frontmatter}

\section{Introduction}
Hyperbolic equations play an important role in various research areas such as fluid dynamics or plasma physics. Efficient numerical methods combined with robust implementations are available for these problems, however they do not account for uncertainties which can arise in measurement data or modeling assumptions. Including the effects of uncertainties in differential equations has become an important topic in the last decades. %Examples include computational fluid dynamics \cite{bijl2013uncertainty}. 

A general hyperbolic set of equations with random initial data can be written as
\begin{subequations}\label{eq:hyperbolicProblem}
\begin{align}
\partial_t \bm{u}(t,\bm{x},\bm{\xi}) + \nabla&\cdot\bm{f}(\bm{u}(t,\bm{x},\bm{\xi})) = \bm{0} \enskip \text{ in } D, \\ \label{eq:ic}
\bm{u}(t=0,\bm{x},&\bm{\xi}) = \bm{u}_{\text{IC}}(\bm{x},\bm{\xi}),
\end{align}
\end{subequations}
where the solution $\bm u\in\mathbb{R}^p$ depends on time $t\in\mathbb{R}^+$, spatial position $\bm{x}\in D\subseteq \mathbb{R}^d$ as well as a vector of random variables $\bm{\xi}\in\Theta\subseteq\mathbb{R}^s$ with given probability density functions $f_{\Xi,i}(\xi_i)$ for $i = 1,\cdots,s$. Hence, the probability density function of $\bm{\xi}$ is then given by $f_{\Xi}(\bm\xi):=\prod_{i=1}^s f_{\Xi,i}(\xi_i)$. The physical flux is given by $\bm{f}:\mathbb{R}^p\to\mathbb{R}^{d\times p}$. To simplify notation, we assume that only the initial condition is random, i.e. $\bm{\xi}$ enters through the definition of $\bm{u}_{IC}$. Equations \eqref{eq:hyperbolicProblem} are usually supplemented with boundary conditions, which we will specify later for the individual problems.

Due to the randomness of the solution, one is interested in determining the expectation value or the variance, i.e.
\begin{align*}
\text{E}[\bm{u}] = \langle \bm{u} \rangle,\qquad \text{Var}[\bm{u}] = \langle \left( \bm{u}-\text{E}[\bm{u}]\right)^2\rangle,
\end{align*}
where we use the bracket operator $\langle \cdot \rangle := \int_{\Theta} \cdot f_{\Xi}(\bm\xi)d\xi_1 \cdots d\xi_s$. More generally, one is interested in determining the moments of the solution for a given set of basis functions $\varphi_{i}:\Theta\to\mathbb{R}$ such that for the multi-index $i = (i_1,\cdots,i_s)$ we have $|i| \leq M$. Note that this yields
\begin{align*}
N:=\begin{pmatrix}
M+s \\ s
\end{pmatrix}
\end{align*}
basis polynomials when defining $|i|:=\sum_{n = 1}^s |i_n|$. Commonly one chooses orthonormal polynomials as basis functions \cite{xiu2002wiener}, i.e. $\langle \varphi_i \varphi_j \rangle =\prod_{n=1}^s\delta_{i_nj_n}$.  The corresponding moments are then given by $\bm{\hat u}_i := \langle \bm{u}\varphi_i \rangle\in\mathbb{R}^p$. Besides facilitating the computation of the expectation value and variance, the moments can be used to span the solution in $\bm\xi$ \cite{wiener1938homogeneous}. \\

Numerical methods for approximating the moment $\bm{\hat u}_i$ can be divided into intrusive and non-intrusive techniques. A popular non-intrusive method is the stochastic-Collocation (SC) method, see e.g. \cite{xiu2005high,babuvska2007stochastic,loeven2008probabilistic}, which computes the moments with the help of a numerical quadrature rule: For a given set of $Q$ quadrature weights $w_k$ and quadrature points $\bm{\xi}_k$, the moments are approximated by
\begin{align*}
\bm{\hat u}_i = \langle \bm{u}\varphi_i \rangle \approx \sum_{k = 1}^{Q}w_k \bm{u}({t,\bm{x},\bm{\xi}_k})\varphi_i(\bm{\xi}_k)f_{\Xi}(\bm{\xi}_k).
\end{align*} 
Since the solution at a fixed quadrature point can be computed by a standard deterministic solver, the SC method does not require a significant implementation effort. Furthermore, SC is embarrassingly parallel, since the required computations can be carried out in parallel on different cores. A downside of collocation methods are aliasing effects, which stem from the inexact approximation of integrals. Furthermore, collocation methods require more runs of the deterministic solver than intrusive methods \cite{xiu2009fast,alekseev2011estimation}. Despite their easy implementation, collocation methods face challenges when examining unsteady problems, since in this case, the solution needs to be written out, i.e. stored in an external file, at all quadrature points in every time step to compute the time evolution of the moments. This contradicts modern HPC paradigms, which aim at reducing the amount of data produced by numerical methods.

Intrusive methods do not suffer from this problem, since the computation is directly carried out on the moments, i.e. their time evolution can be recorded during the computation. Also for steady problems, the fact that the moments are known in each iteration enables the computation of the stochastic residual, which indicates when to stop the iteration towards the steady state solution. However, intrusive methods are in general more difficult to implement and come along with higher numerical costs. The main idea of these methods is to derive a system of equations for the moments and then implementing a numerical solver for this system: Testing the initial problem \eqref{eq:hyperbolicProblem} with $\varphi_i$ for $|i|\leq M$ yields
\begin{subequations}\label{eq:nonClosedMomentSystem}
\begin{align}
\partial_t \bm{\hat u}_i(t,\bm{x}) + \nabla&\cdot\langle\bm{f}(\bm{u}(t,\bm{x},\cdot)) \varphi_i\rangle = \bm{0}, \\
\bm{\hat u_i}(t=0,\bm{x}&) = \langle\bm{u}_{\text{IC}}(\bm{x},\cdot)\varphi_i\rangle.
\end{align}
\end{subequations}
To obtain a closed set of equations, one needs to derive a closure $\mathcal{U}$ such that 
\begin{align*}
\bm{u}(t,\bm x,\bm \xi) \approx \mathcal{U}(\bm{\hat u};\bm\xi).
\end{align*}
Here, we collect the $N$ moments for which $\vert i \vert \leq M$ holds in the moment matrix $\bm{\hat{u}}:=(\bm{\hat{u}}_i)_{|i|\leq M}\in\mathbb{R}^{N \times p}$. The dependency of $\mathcal{U}$ on $\bm \xi$ will occasionally be omitted in the following for sake of readability. A commonly used closure is given by stochastic-Galerkin (SG) \cite{ghanem2003stochastic}, which represents the solution by a polynomial:
\begin{align}\label{eq:SGClosure}
\mathcal{U}_{\text{SG}}(\bm{\hat u};\bm\xi):= \sum_{|i|\leq M} \bm{\hat{u}}_i\varphi_i(\bm{\xi}) = \hat{\bm u}^T\bm{\varphi}(\bm\xi),
\end{align}
Corresponding to the definition of $\bm{\hat u}$, we collect the $N$ basis function for which $\vert i \vert \leq M$ holds in $\bm{\varphi}:=(\varphi_i)_{|i|\leq M}\in\mathbb{R}^{N+1}$. When using the stochastic-Galerkin method to close \eqref{eq:nonClosedMomentSystem}, the resulting moment system is not necessarily hyperbolic \cite{poette2009uncertainty} and the solution needs to be manipulated \cite{schlachter2018hyperbolicity} in order to prevent a failure of the method. A generalization of stochastic-Galerkin, which ensures hyperbolicity is the Intrusive Polynomial Moment (IPM) closure \cite{poette2009uncertainty}: The closure is given by a constraint optimization problem. For a given convex entropy $s:\mathbb{R}^p\to\mathbb{R}$ for the original problem \eqref{eq:hyperbolicProblem}, this optimization problem is given by
\begin{align}\label{eq:primalProblem}
\mathcal{U}(\bm{\hat u}) = \argmin_{\bm u} \langle s(\bm u) \rangle \enskip \text{ subject to } \bm{\hat u}_i = \langle \bm u \varphi_i \rangle \text{ for } |i| \leq M.
\end{align}
Rewritten in its dual form, \eqref{eq:primalProblem} is transformed into an unconstraint optimization problem. Defining the variables $\bm{\lambda}_i\in\mathbb{R}^p$ where $i$ is again a multi index, gives the unconstrained dual problem
\begin{align}\label{eq:dualProblem}
 \bm{\hat \lambda}(\bm{\hat u}) := \argmin_{\bm{\lambda} \in \mathbb{R}^{N \times p}}
  \left\{\langle s_*(\bm{\lambda}^T \bm\varphi)\rangle - \sum_{|i|\leq M}\bm{\lambda}_i^T \bm{\hat u}_i\right\},
\end{align}
where $s_*:\mathbb{R}^p\to\mathbb{R}$ is the Legendre transformation of $s$, and $\bm{ \hat\lambda}:=(\bm{\hat{\lambda}}_i)_{|i|\leq M}\in \mathbb{R}^{N \times p}$ are called the dual variables. The solution to \eqref{eq:primalProblem} is then given by
\begin{align}\label{eq:ansatz}
 \mathcal{U}(\bm{\hat u}) = \left( \nabla_{\bm{u}} s \right)^{-1}(\bm{\hat{\lambda}}(\bm{\hat u})^T \bm{\varphi}).
\end{align}
Plugging the derived closure into the moment system \eqref{eq:nonClosedMomentSystem}, one obtains the IPM moment system
\begin{subequations}\label{eq:IPMmomentSystem}
\begin{align}
\partial_t \bm{\hat u}_i(t,\bm{x}) + \nabla&\cdot\langle\bm{f}(\mathcal{U}(\bm{\hat u})) \varphi_i\rangle = \bm{0}, \\
\bm{\hat u}_i(t=0,\bm{x}&) = \langle\bm{u}_{\text{IC}}(\bm{x},\cdot)\varphi_i\rangle.
\end{align}
\end{subequations}
with $|i|\leq M$. The IPM method has several advantages: Choosing the entropy $s(\bm{u}) = \frac{1}{2}\bm{u}^T\bm{u}$ yields the stochastic-Galerkin closure, i.e. IPM generalizes different intrusive methods. Furthermore, at least for scalar problems, IPM is significantly less oscillatory compared to SG \cite{kusch2017maximum}. Also, as discussed in \cite{poette2009uncertainty}, when choosing $s(\bm u)$ to be a physically correct entropy of the deterministic problem, the IPM solution dissipates the expectation value of the entropy, which is
\begin{align*}
S(\bm{\hat u}) := \langle s( \mathcal{U}(\bm{\hat u}))\rangle,
\end{align*}
i.e. the IPM method yields a physically correct entropy solution. This again underlines a weakness of stochastic-Galerkin: If $s(\bm{u}) = \frac{1}{2}\bm{u}^T\bm{u}$ is not a correct entropy of the original problem, the SG method can lead to non-physical solution values, which can then cause a failure of the method. The main weakness of the IPM method is its run time, since it requires the repeated evaluation of \eqref{eq:ansatz}, which involves solving the optimization problem \eqref{eq:dualProblem}. Hence, the desirable properties of IPM come along with significantly increased numerical costs. However, IPM and minimal entropy methods in general are well suited for modern HPC architecture, which can be used to reduce the run time \cite{garrett2015optimization}. 

When studying hyperbolic equations, the moment approximations of various methods such as Stochastic Galerkin \cite{le2004uncertainty}, IPM \cite{kusch2018filtered} and stochastic-Collocation \cite{barth2013non,dwight2013adaptive} tend to show incorrect discontinuities in certain regions of the physical space. These non-physical structures dissolve when the number of basis functions is increased \cite{pettersson2009numerical,offner2017stability} or when artificial diffusion is added through the spatial numerical method \cite{offner2017stability} or filters \cite{kusch2018filtered}. Also, a multi-element approach which divides the uncertain domain into cells and uses piece-wise polynomial basis functions to represent the solution has proven to mitigate non-physical discontinuities \cite{durrwachter2018hyperbolicity}. These structures commonly arise on a small portion of the space-time domain. Therefore, intrusive methods seem to be an adequate choice since they are well suited for adaptive strategies. By locally increasing the polynomial order \cite{tryoen2012adaptive,kroker2012finite,giesselmann2017posteriori} or adding artificial viscosity \cite{kusch2018filtered} at certain spatial positions and time steps in which complex structures such as discontinuities occur, a given accuracy can be reached with significantly reduced numerical costs. The main task here is to find an adequate refinement indicator. \\

In this paper, we present a semi-intrusive code framework, which facilitates the task of implementing general intrusive methods. The framework only requires the numerical flux of the deterministic problem (as well as an entropy if IPM is used). We thereby provide the ability to recycle existing implementations of deterministic solvers.
Furthermore, we investigate intrusive methods for steady problems and compare them against collocation methods. The steady setting provides different opportunities to take advantage of features of intrusive methods: 
\begin{itemize}
\item Accelerate the convergence to the IPM steady state solution by applying IPM as a post-processing step for collocation methods: We converge the moments of the solution to a steady state with an inaccurate, but cheap collocation method and then use the resulting collocation moments as starting values for an expensive but accurate intrusive method such as IPM, which we then again converge to steady state. 
\item Compute inexact dual variables \eqref{eq:dualProblem} for IPM: Since the moments during the iteration process are inaccurate, i.e. they are not the correct steady state solution, we propose to not fully converge the dual iteration, which computes \eqref{eq:dualProblem}. 
\end{itemize}
In addition to these two strategies for steady problems, we make use of adaptivity: The intrusive nature of SG and IPM can be used to locally increase the number of moments whenever the solution has a complex structure in the random variable (as well as decrease the number of moments if not). In the case of steady problems, one can perform a large number of iteration to the steady state solution on a low number of moments and increase the maximal refinement level when the distance to the steady state has reached a specified barrier. Similar to the idea of using IPM as a post processing step, a large number of iterations will be performed by a cheap method, i.e. we can reduce numerical costs. The effectiveness of these acceleration ideas are tested by comparing results with stochastic-Collocation for the uncertain NACA test case as well as a bent shock tube problem. 

The paper is structured as follows: After the introduction, we discuss the numerical discretization as well as the implementation and structure of the semi-intrusive framework in section \ref{sec:framework}. Section \ref{sec:collIPM} discusses the IPM acceleration with a non-intrusive method and in section \ref{sec:OneShotIPM}, we discuss the idea of not converging the dual iteration. Section~\ref{sec:adaptivity} extends the presented numerical framework to an algorithm making use of adaptivity. A comparison of results computed with the presented methods is then given in \ref{sec:results}, followed by a summary and outlook in section \ref{sec:summary_outlook}.


\section{Discretization and code framework}
\label{sec:framework}
\subsection{Discretization}
In the following, we discretize the moment system in space and time following according to \cite{kusch2017maximum}. Due to the fact, that stochastic-Galerkin can be interpreted as IPM with a quadratic entropy, it suffices to only derive a discretization of the IPM moment system \eqref{eq:IPMmomentSystem}.  
Omitting initial conditions and assuming a one-dimensional spatial domain, we can write this system  as
\begin{align*}
\partial_t \bm{\hat u}+\partial_x \bm{F}(\bm{\hat u}) = \bm{0}
\end{align*}
with the flux $\bm{F}:\mathbb{R}^{(N+1)\times p}\to\mathbb{R}^{(N+1)\times p}$, $\bm{F}(\bm{u})=\langle f(\mathcal{U}_{ME}(\bm{\hat u}))\bm{\varphi} \rangle$, where we omit the index ME in the following for efficiency of exposition. Due to hyperbolicity of the IPM moment system, one can use a finite-volume method to approximate the time evolution of the IPM moments. First we perform the discretization of the spatial domain: We choose the discrete unknowns to be the spatial averages over each cell at time $t_n$, given by
\begin{align*}
\bm{\hat u}_{ij}^n \simeq \frac{1}{\Delta x}\int_{x_{j-1/ 2}}^{x_{j+1/ 2}}\bm{\hat u}_i(t_n,x) dx.
\end{align*}
If a moment vector in cell $j$ at time $t_n$ is denoted as $\bm{\hat u}_j^n = (\bm{\hat u}_{0j}^n,\cdots,\bm{\hat u}_{Nj}^n)^T\in\mathbb{R}^{N+1}$, the finite-volume scheme can be written in conservative form with the numerical flux $\bm{G}:\mathbb{R}^{(N+1)\times p}\times\mathbb{R}^{(N+1)\times p}\to\mathbb{R}^{(N+1)\times p}$ as
\begin{align}\label{eq:IPMDiscretization}
\bm{\hat u}_{j}^{n+1} = \bm{\hat u}_{j}^{n}  - \frac{\Delta t}{\Delta x}\left( \bm{G}(\bm{\hat u}_{j}^{n},\bm{\hat u}_{j+1}^{n})- \bm{G}(\bm{\hat u}_{j-1}^{n},\bm{\hat u}_{j}^{n})\right)
\end{align}
for $j = 1,\cdots,N_x$ and $n = 0,\cdots,N_t$, where $N_x$ is the number of spatial cells and $N_t$ is the number of time steps.
The numerical flux is assumed to be consistent, i.e., that $\bm{G}(\bm{u},\bm{u})=\bm{F}(\bm{u})$.
To ensure stability, a CFL condition has to be derived by investigating the eigenvalues of $\nabla \bm{F}$.

When a consistent numerical flux $\bm g:\mathbb{R}^p\times\mathbb{R}^p\to\mathbb{R}^p$, $\bm g = \bm g(\bm u_\ell, \bm u_r)$ is available for the deterministic problem \eqref{eq:origProblem}, then for the IPM system we can simply take
\begin{align*}
 \bm{G}(\bm{\hat u}_{j}^n,\bm{\hat u}_{j+1}^{n}) = \langle g(u_{ME}(\hat\Lambda(\bm{\hat u}_{j}^{n})),u_{ME}(\hat\Lambda(\bm{\hat u}_{j+1}^{n})))\bm{\varphi}\rangle.
\end{align*}
This choice of the numerical flux is a common choice in kinetic theory and is called kinetic flux.
The time update of the moment vector now becomes
\begin{align}\label{eq:exactUpdate}
\bm{\hat u}_{j}^{n+1} = \bm{\hat u}_{j}^{n}- \frac{\Delta t}{\Delta x}\left( \langle g(u_{ME}(\hat \Lambda_j^n),u_{ME}(\hat \Lambda_{j+1}^n))\bm{\varphi}\rangle - \langle g(u_{ME}(\hat \Lambda_{j-1}^n),u_{ME}(\hat \Lambda_{j}^n))\bm{\varphi} \rangle\right),
\end{align}
where $\hat\Lambda_{j}^n :=\hat\Lambda(\bm{\hat u}_{j}^{n})$ for all $j$. Note that the computation of $\hat\Lambda_{j}^n$ requires solving the dual problem \eqref{eq:dualProblem} for the moment vector $\bm{\hat u}_{j}^{n}$.

Unfortunately \eqref{eq:exactUpdate} cannot be implemented because the dual problem cannot be solved exactly.%
\footnote{
Equation \eqref{eq:exactUpdate} also includes integral evaluations which cannot be computed in closed form.
Their approximation by numerical quadrature, however, does not play a role in the realizability problems we discuss below.
}
Instead, it must be solved numerically, for example with Newton's method.
The stopping criterion for the numerical optimizer ensures that the approximate multiplier vector it returns, which we denote $\xoverline{\bm{\lambda}}_j^n\in\mathbb{R}^{N+1}$ for the moment vector $\bm{\hat u}_{j}^{n}$, satisfies the stopping criterion
\begin{align}\label{eq:tauCrit}
\left\Vert \bm{\hat u}_{j}^{n}-\left\langle u_{ME}\left(\left(\xoverline{\bm{\lambda}}_j^n\right)^T\bm{\varphi}\right)\bm{\varphi}\right\rangle \right\Vert < \tau.
\end{align}
This is derived from the first-order necessary conditions for the dual problem.
Once the numerical optimizer finds such a $\xoverline{\bm{\lambda}}_j^n$, the corresponding dual state $\xoverline{\Lambda}_j^n := \left(\xoverline{\bm{\lambda}}_j^n\right)^T\bm{\varphi} \in \mathbb{P}(\Theta)$ can be used in \eqref{eq:exactUpdate} for the unknown $\hat \Lambda_j^n$.
This gives Algorithm \ref{alg:seq}.

\begin{algorithm}[H]
\begin{algorithmic}[1]
\FOR{$j=0$ to $N_x+1$}
\STATE $\bm{u}_j^0 = \frac{1}{\Delta x} \int_{x_{j-1/ 2}}^{x_{j+1/ 2}} \langle u_0(x, \cdot) \bm{\varphi} \rangle dx$
\ENDFOR
\FOR{$n=0$ to $N_t$}
\FOR{$j=0$ to $N_x+1$}
\STATE $\xoverline{\bm{\lambda}}_j^n \approx \argmin_{\bm{\lambda}}  \left( \langle s_*(\bm{\lambda}^T \bm{\varphi})\rangle - \bm{\lambda}^T \bm{\hat u}_{j}^{n} \right)$
\hfill such that \eqref{eq:tauCrit} holds
\STATE $\xoverline \Lambda_j^n = \left(\xoverline{\bm{\lambda}}_j^n\right)^T\bm{\varphi}$
\ENDFOR
\FOR{$j=1$ to $N_x$}
\STATE $\bm{\hat u}_{j}^{n+1} = \bm{\hat u}_{j}^{n}- \frac{\Delta t}{\Delta x}\left( \langle g(u_{ME}(\xoverline \Lambda_j^n),u_{ME}(\xoverline \Lambda_{j+1}^n))\bm{\varphi}\rangle - \langle g(u_{ME}(\xoverline \Lambda_{j-1}^n),u_{ME}(\xoverline \Lambda_{j}^n))\bm{\varphi} \rangle\right)$ 
\ENDFOR
\ENDFOR
\end{algorithmic}
\caption{IPM for Uncertainty Quantification}
\label{alg:seq}
\end{algorithm}
%\subsection{Collocation accelerated IPM}
%\label{sec:collIPM}
%Commonly, a great amount of iterations in pseudo-time is needed to converge to a steady state solution. Consequently, the IPM method which requires solving the dual problem \eqref{eq:dualProblem} in every spatial cell in each iteration becomes prohibitively expensive. We tackle this problem by using IPM only as a postprocesssing step for the steady solution obtained by a cheap method. (Or vice-versa, we use a cheap method as a preprocessing step for IPM). In our case, we perform the preprocessing step with stochastic-Collocation, i.e. we converge the moments to a steady state solution by applying collocation steps. The obtained moments are then used as initial condition for the IPM moment system (for which the moments are no longer a steady state solution). After applying a significantly reduced number of IPM iterations, we obtain a steady state IPM solution. In our numerical experiments presented in section \ref{sec:results}, we can show that the overall costs are dominated by the large number of cheap collocation steps and not by the small number of expensive IPM steps, while the solution shows the expected desirable properties of the IPM solution.

%Different variants of this method are possible:
%\begin{itemize}
%\item Since the IPM iterations will again modify the steady state Collocation solution, it is not necessary to converge Collocation to the exact steady state solution before starting IPM. Here, one needs to determine an indicator to choose at which residual the collocation iteration is sufficiently accurate and can therefore be switched to IPM.
%\item The main idea is to use a cheap but inexact method as a preconditioner for an expensive but accurate method. Here, one is not limited in choosing SC and IPM, but one can for example choose Monte Carlo methods as an accelerator or SC with an increased number of quadrature points as the expensive method. Note that in the latter case, a map from the moments to the solution at the increased quadrature set is required, for which the IPM reconstruction \eqref{eq:ansatz} would be a suitable choice.
%\end{itemize}
\section{One-Shot IPM}
\label{sec:OneShotIPM}

In the following section we only consider steady state problems, i.e. equation \eqref{eq:fulleq} reduces to
\begin{align}\label{eq:hyperbolicProblemSteady}
\nabla\cdot\bm{f}(\bm{u}(\bm{x},\bm{\xi})) = \bm{0} \enskip \text{ in } D
\end{align}
with adequate boundary conditions. A general strategy for computing the steady state solution to \eqref{eq:hyperbolicProblemSteady} is to introduce a pseudo-time and numerically treat \eqref{eq:hyperbolicProblemSteady} as an unsteady problem. A steady state solution is then obtained by iterating in pseudo-time until the solution remains constant. It is important to point out that the time it takes to converge to a steady state solution is crucially affected by the chosen initial condition and its distance to the steady state solution.
Similar to the unsteady case \eqref{eq:hyperbolicProblem}, we can again derive a moment system for \eqref{eq:hyperbolicProblemSteady} given by
\begin{align}\label{eq:MomentSystemSteady}
\nabla\cdot\langle\bm{f}(\bm{u}(\bm{x},\bm{\xi}))\bm{\varphi}^T\rangle^T = \bm{0} \enskip \text{ in } D
\end{align}
which is again needed for the construction of intrusive methods. By introducing a pseudo-time and using the IPM closure, we obtain the same system as in \eqref{eq:SGMomentSystem}, i.e. Algorithm \ref{alg:IPM} can be used to iterate to a steady state solution. Note that now, the time iteration is not performed for a fixed number of time steps $N_t$, but until the condition
\begin{align}\label{eq:residualSteady}
\sum_{j = 1}^{N_x} \Delta x_j \Vert \bm{\hat{u}}_j^n - \bm{\hat{u}}_j^{n-1} \Vert \leq \varepsilon
\end{align}
is fulfilled. Condition \eqref{eq:residualSteady}, which is for example being used in the SU2 code framework \cite{economon2015su2}, measures the change of the solution by a single time iteration. Note, that in order to obtain an estimate of the distance to the steady state solution, one has to include the Lipschitz constant of the corresponding fixed point problem. Since one is generally interested in low order moments such as the expectation value, the residual \eqref{eq:residualSteady} can be modified by only accounting for the zero order moments.

In this section we aim at breaking up the inner loop in the IPM Algorithm \ref{alg:IPM}, i.e. to just perform one iteration of the dual problem in each time step. Consequently, the IPM reconstruction given by \eqref{eq:primalProblem} is not done exactly, meaning that the reconstructed solution does not minimize the entropy while not fulfilling the moment constraint. However, the fact that the moment vectors are not yet converged to the steady solution seems to permit such an inexact reconstruction. Hence, we aim at iterating the moments to steady state and the dual variables to the exact solution of the IPM optimization problem \eqref{eq:primalProblem} simultaneously.
By successively performing one update of the moment iteration and one update of the dual iteration, we obtain 
\begin{subequations}\label{eq:oneshotIPM}
\begin{align}
&\bm{\lambda}_{j}^{n+1} =  \bm{d}(\bm{\lambda}_j^{n},\bm{u}_j^{n}) \enskip \text{ for all j} \label{eq:oneshotIPMdual}\\
&\bm{u}_j^{n+1} =  \bm{c}\left(\bm{\lambda}_{j-1}^{n+1},\bm{\lambda}_{j}^{n+1},\bm{\lambda}_{j+1}^{n+1}\right)\label{eq:oneshotIPMmoment}.
\end{align}
\end{subequations}
This yields Algorithm \ref{alg:osIPM}.
\begin{algorithm}[H]
\begin{algorithmic}[1]
\For{$j=0$ to $N_x+1$}
\State $\bm{u}_j^0 \leftarrow \frac{1}{\Delta x} \int_{x_{j-1/ 2}}^{x_{j+1/ 2}} \langle u_{\text{IC}}(x, \cdot) \bm{\varphi} \rangle_Q dx$
\EndFor
\While{\eqref{eq:residualSteady} is violated}
\For{$j=1$ to $N_x$}
\State $\bm{\lambda}_j^{n+1} \leftarrow \bm{d}(\bm{\lambda}_{j}^{n};\bm{\hat u}_j^{n})$
\State $\bm{\hat u}_j^{n+1} \leftarrow \bm{c}(\bm{\lambda}_{j-1}^{n+1},\bm{\lambda}_j^{n+1},\bm{\lambda}_{j+1}^{n+1})$
\EndFor
\State $n \leftarrow n+1$
\EndWhile
\end{algorithmic}
\caption{One-Shot IPM implementation}
\label{alg:osIPM}
\end{algorithm}
We call this method One-Shot IPM, since it is inspired by One-Shot optimization, see for example \cite{hazra2005aerodynamic}, which uses only a single iteration of the primal and dual step in order to update the design variables. Note that the dual variables from the One-Shot iteration are written without a hat to indicate that they are not the exact solution of the dual problem.

In the following, we will show that this iteration converges, if the chosen initial condition is sufficiently close to the steady state solution. For this we take an approach commonly chosen to prove local convergence properties of Newton's method: In Theorem \ref{th:Contractive}, we show that the iteration function is contractive at its fixed point and conclude in Theorem \ref{th:localConvergence} that this yields local convergence. Hence, we preserve the convergence property of the original IPM method, which uses Newton's method and therefore only converges locally as well.
\begin{theorem}\label{th:Contractive}
Assume that the classical IPM iteration is contractive at its fixed point $\bm{\hat u}^*$. Then the Jacobi matrix $\bm{J}$ of the One-Shot IPM iteration \eqref{eq:oneshotIPM} has a spectral radius $\rho(\bm{J})<1$ at the fixed point $(\bm{\lambda}^*,\bm{\hat u}^*)$.
\end{theorem}
\begin{proof}
First, to understand what contraction of the classical IPM iteration implies, we rewrite the moment iteration \eqref{eq:momentIteration} of the classical IPM scheme: When defining the update function
\begin{align*}
\bm{\tilde c}\left(\bm{\hat{u}}_{\ell},\bm{\hat{u}}_{c},\bm{\hat{u}}_{r}\right):=\bm{c}\left(\bm{\hat{\lambda}}(\bm{\hat{u}}_{\ell}),\bm{\hat{\lambda}}(\bm{\hat{u}}_{c}),\bm{\hat{\lambda}}(\bm{\hat{u}}_{r})\right)
\end{align*}
we can rewrite the classical moment iteration as
\begin{align}\label{eq:shortIPMIt}
\bm{\hat u}_j^{n+1} = \bm{\tilde c}\left(\bm{\hat u}_{j-1}^n,\bm{\hat u}_{j}^n,\bm{\hat u}_{j+1}^n\right).
\end{align}
Since we assume that the classical IPM scheme is contractive at its fixed point, we have $\rho (\nabla_{\bm{\hat u}}\bm{\tilde c}(\bm{\hat u}^*))<1$ with $\nabla_{\bm{\hat u}}\bm{\tilde c}\in\mathbb{R}^{N\cdot N_x\times N\cdot N_x}$ defined by
\begin{align*}
\nabla_{\bm{\hat u}}\bm{\tilde c} = 
\begin{pmatrix} 
    \partial_{\bm{\hat u}_c}\bm{\tilde c}_{1} & \partial_{\bm{\hat u}_r}\bm{\tilde c}_{1}& 0 & 0 & \dots \\
    \partial_{\bm{\hat u}_{\ell}}\bm{\tilde c}_{2} & \partial_{\bm{\hat u}_c}\bm{\tilde c}_{2} & \partial_{\bm{\hat u}_r}\bm{\tilde c}_{2}& 0 & \dots \\
    0 & \partial_{\bm{\hat u}_{\ell}}\bm{\tilde c}_{3} & \partial_{\bm{\hat u}_c}\bm{\tilde c}_{3} & \partial_{\bm{\hat u}_r}\bm{\tilde c}_{3}\\
    \vdots & & & \ddots & \\
    0 &\cdots &  0 & \partial_{\bm{\hat u}_{\ell}}\bm{\tilde c}_{N_x} & \partial_{\bm{\hat u}_c}\bm{\tilde c}_{N_x}
    \end{pmatrix},
\end{align*}
where we define $\bm{\tilde c}_{j}:=\bm{\tilde c}\left(\bm{\hat u}_{j-1}^*,\bm{\hat u}_{j}^*,\bm{\hat u}_{j+1}^*\right)$ for all $j$. Now for each term inside the matrix $\nabla_{\bm{\hat u}}\bm{\tilde c}$ we have 
\begin{align}\label{eq:cTildeDer}
\partial_{\bm{\hat u}_{\ell}}\bm{\tilde c}_{j} = \frac{\partial \bm{c}_j}{\partial \bm{\hat \lambda}_{\ell}}\frac{\partial \bm{\hat \lambda}(\bm{\hat u}_{j-1}^*)}{\partial \bm{\hat u}},\enskip\partial_{\bm{\hat u}_c}\bm{\tilde c}_{j} = \frac{\partial \bm{c}_j}{\partial \bm{\hat \lambda}_c}\frac{\partial \bm{\hat \lambda}(\bm{\hat u}_j^*)}{\partial \bm{\hat u}},\enskip\partial_{\bm{\hat u}_r}\bm{\tilde c}_{j} = \frac{\partial \bm{c}_j}{\partial \bm{\hat \lambda}_r}\frac{\partial \bm{\hat \lambda}(\bm{\hat u}_{j+1}^*)}{\partial \bm{\hat u}}.
\end{align}
We first wish to understand the structure of the terms $\partial_{\bm{\hat u}} \bm{\hat \lambda}(\bm{\hat u})$. For this, we note that the exact dual variables fulfill
\begin{align}\label{eq:ulambda}
\bm{\hat u} = \langle \bm{u}_s(\bm{\hat \lambda}^T\bm{\varphi})\bm{\varphi}^T\rangle^T =: \bm{h}(\bm{\hat \lambda}),
\end{align}
which is why we have the mapping $\bm{\hat u}:\mathbb{R}^{N\times m}\to\mathbb{R}^{N\times m}$, $\bm{\hat u}(\bm{\hat \lambda}) = \bm{h}(\bm{\hat \lambda})$. Since the solution of the dual problem for a given moment vector is unique, this mapping is bijective and therefore we have an inverse function
\begin{align}\label{eq:lambdau}
\bm{\hat \lambda} = \bm{h}^{-1}(\bm{\hat u}(\bm{\hat \lambda})).
\end{align}
Now we differentiate both sides w.r.t. $\bm{\hat \lambda}$ to get
\begin{align*}
\bm{I}_{d} = \frac{\partial \bm{h}^{-1}(\bm{\hat u}(\bm{\hat \lambda}))}{\partial \bm{\hat u}}\frac{\partial \bm{\hat u}(\bm{\hat \lambda})}{\partial \bm{\hat \lambda}}.
\end{align*}
We multiply with the matrix inverse of $\frac{\partial \bm{\hat u}(\bm{\hat \lambda})}{\partial \bm{\hat \lambda}}$ to get
\begin{align*}
\left(\frac{\partial \bm{\hat u}(\bm{\hat \lambda})}{\partial \bm{\hat \lambda}}\right)^{-1} = \frac{\partial \bm{h}^{-1}(\bm{\hat u}(\bm{\hat \lambda}))}{\partial \bm{\hat u}}.
\end{align*}
Note that on the left-hand-side we have the inverse of a matrix and on the right-hand-side, we have the inverse of a multi-dimensional function. By rewriting $\bm{h}^{-1}(\bm{\hat u}(\bm{\hat \lambda}))$ as $\bm{\hat \lambda}(\bm{\hat u})$ and simply computing the term $\frac{\partial \bm{\hat u}(\bm{\hat \lambda})}{\partial \bm{\hat \lambda}}$ by differentiating \eqref{eq:ulambda} w.r.t. $\bm{\hat \lambda}$, one obtains
\begin{align}\label{eq:dudlambdaex}
\partial_{\bm{\hat u}} \bm{\hat \lambda}(\bm{\hat u}) = \langle \nabla\bm{u}_s(\bm{\hat \lambda}^T\bm{\varphi})\bm{\varphi}\bm{\varphi}^T\rangle^{-T}.
\end{align}
Now we begin to derive the spectrum of the \textit{One-Shot IPM} iteration \eqref{eq:oneshotIPM}. Note that in its current form this iteration is not really a fixed point iteration, since it uses the time updated dual variables in \eqref{eq:oneshotIPMmoment}. To obtain a fixed point iteration, we plug the dual iteration step \eqref{eq:oneshotIPMdual} into the moment iteration \eqref{eq:oneshotIPMmoment} to obtain
\begin{align*}
&\bm{\lambda}_j^{n+1} = \bm{d}(\bm{\lambda}_j^{n},\bm{\hat u}_j^{n}) \enskip \text{ for all j} \\
&\bm{\hat u}_j^{n+1} =  \bm{c}\left(\bm{d}(\bm{\lambda}_{j-1}^{n},\bm{\hat u}_{j-1}^{n}),\bm{d}(\bm{\lambda}_{j}^{n},\bm{\hat u}_{j}^{n}),\bm{d}(\bm{\lambda}_{j+1}^{n},\bm{\hat u}_{j+1}^{n})\right).
\end{align*}
The Jacobian $\bm{J}\in\mathbb{R}^{2N\cdot N_x \times 2N\cdot N_x}$ has the form
\begin{align}\label{eq:Jacobian}
\bm{J} = 
\begin{pmatrix}
 \partial_{\bm{\lambda}} \bm{d} & \partial_{\bm{\hat u}} \bm{d}  \\
\partial_{\bm{\lambda}} \bm{c} & \partial_{\bm{\hat u}} \bm{c}
\end{pmatrix},
\end{align}
where each block has entries for all spatial cells. We start by looking at $\partial_{\bm{\lambda}} \bm{d}$. For the columns belonging to cell $j$, we have
\begin{align*}
\partial_{\bm{\lambda}} \bm{d}(\bm{\lambda}_j^n,\bm{\hat u}_j^n) &= \bm{I}_d - \bm{H}(\bm\lambda)^{-1} \cdot \langle \nabla\bm{u}_s(\bm{\varphi}^T\bm{\lambda}_j^n)\bm{\varphi}\bm{\varphi}^T \rangle^T - \partial_{\bm{\lambda}}\bm{H}(\bm\lambda)^{-1} \cdot \left( \langle \bm{u}_s(\bm{\varphi}^T\bm{\lambda}_j^n)\bm{\varphi}^T \rangle^T - \bm{\hat u}\right) \\
&=- \partial_{\bm{\lambda}}\bm{H}(\bm\lambda)^{-1} \cdot \left( \langle \bm{u}_s(\bm{\varphi}^T\bm{\lambda}_j^n)\bm{\varphi}^T \rangle^T - \bm{\hat u}\right).
\end{align*}
Recall that at the fixed point $(\bm{\lambda}^*,\bm{\hat u}^*)$, we have $\langle \bm{u}_s(\bm{\varphi}^T\bm{\lambda}_j^n)\bm{\varphi}^T \rangle^T = \bm{\hat u}$, hence one obtains $\partial_{\bm{\lambda}} \bm{d}=\bm{0}$. For the block $\partial_{\bm{\hat u}} \bm{d}$, we get 
\begin{align*}
\partial_{\bm{\hat u}} \bm{d}(\bm{\lambda}_j^n,\bm{\hat u}_j^n) = \bm{H}(\bm\lambda)^{-1},
\end{align*}
hence $\partial_{\bm{\hat u}} \bm{d}$ is a block diagonal matrix. Let us now look at $\partial_{\bm{\lambda}} \bm{c}$ at a fixed spatial cell $j$:
\begin{align*}
\frac{\partial \bm{c}}{\partial \bm{\lambda}_{\ell}}\frac{\partial \bm{d}(\bm{\lambda}_{j-1}^{n},\bm{\hat u}_{j-1}^{n})}{\partial \bm{\lambda}} = \bm{0},
\end{align*}
since we already showed that by the choice of $\bm{H}(\bm\lambda)^{-1}$ the term $\partial_{\bm{\lambda}} \bm{d}$ is zero. We can show the same result for all spatial cells and all inputs of $\bm{c}$ analogously, hence $\partial_{\bm{\lambda}} \bm{c} = \bm{0}$. For the last block, we have that 
\begin{align*}
\frac{\partial \bm{c}}{\partial \bm{\lambda}_{\ell}}\frac{\partial \bm{d}(\bm{\lambda}_{j-1}^{n},\bm{\hat u}_{j-1}^{n})}{\partial \bm{\hat u}} = \frac{\partial \bm{c}}{\partial \bm{\lambda}_{\ell}} \bm{H}(\bm\lambda)^{-1} = \frac{\partial \bm{c}}{\partial \bm{\lambda}_{\ell}} \langle \nabla\bm{u}_s(\bm{\varphi}^T\bm{\lambda}_{j-1}^n)\bm{\varphi}\bm{\varphi}^T \rangle^{-T} = \partial_{\bm{\hat u}_{\ell}}\bm{\tilde c}_j
\end{align*}
by the choice of $\bm{H}(\bm\lambda)^{-1}$ as well as \eqref{eq:cTildeDer} and \eqref{eq:dudlambdaex}. We obtain an analogous result for the second and third input. Hence, we have that $\partial_{\bm{\hat u}} \bm{c} = \nabla_{\bm{\hat u}}\bm{\tilde c}$, which only has eigenvalues between $-1$ and $1$ by the assumption that the classical IPM iteration is contractive. Since $\bm{J}$ is an upper triangular block matrix, the eigenvalues are given by $\lambda\left(\partial_{\bm{\lambda}} \bm{d}\right) = 0$ and $\lambda\left(\partial_{\bm{\hat u}} \bm{c}\right)\in(-1,1)$, hence the One-Shot IPM is contractive around its fixed point.
\end{proof}
\begin{theorem}\label{th:localConvergence}
With the assumptions from Theorem \ref{th:Contractive}, the One-Shot IPM converges locally, i.e. there exists a $\delta>0$ s.t. for all starting points $(\bm{\lambda}^0,\bm{\hat u}^0)\in B_{\delta}(\bm{\lambda}^*,\bm{\hat u}^*)$ we have
\begin{align*}
\Vert (\bm{\lambda}^n,\bm{\hat u}^n) - (\bm{\lambda}^*,\bm{\hat u}^*)\Vert \rightarrow 0 \qquad \text{ for } n \rightarrow \infty.
\end{align*}
\end{theorem}
\begin{proof}
By Theorem \ref{th:Contractive}, the One-Shot scheme is contractive at its fixed point. Since we assumed convergence of the classical IPM scheme, we can conclude that all entries in the Jacobian $\bm{J}$ are continuous functions. Furthermore, the determinant of $\bm{\tilde{J}}:=\bm{J}-\lambda \bm{I}_d$ is a polynomial of continuous functions, since
\begin{align*}
\text{det}(\bm{\tilde J}) = \sum_{\sigma} \text{sgn}(\sigma)\prod_{i = 1}^{2 N_x N} \tilde J_{\sigma(i),i}.
\end{align*}
Since the roots of a polynomial vary continuously with its coefficients, the eigenvalues of $\bm{J}$ are continuous w.r.t $(\bm{\lambda},\bm{\hat u})$. Hence there exists an open ball with radius $\delta$ around the fixed point in which the eigenvalues remain in the interval $(-1,1)$.
\end{proof}
%\begin{remark}
%Since the preconditioning step of the Collocation-accelerated IPM method generates initial conditions which are close to the steady state solution, using One-Shot IPM instead of classical IPM is well suited. However, our numerical calculations show that One-Shot IPM converges even if the solution is far away from its steady state. 
%\end{remark}
\section{Results}
\label{sec:results}

\subsection{2D Euler equations}
We start by quantifying the effects of an uncertain angle of attack $\phi\sim U(0.75,1.75)$ for a NACA0012 profile computed with different methods. The stochastic Euler equations in two dimensions are given by
\begin{align*}
\partial_t
\begin{pmatrix}
\rho \\ \rho v_1 \\ \rho v_2 \\ \rho e
\end{pmatrix}
+\partial_{x_1}
\begin{pmatrix}
\rho v_1 \\ \rho v_1^2 +p \\ \rho v_1 v_2 \\  v_1 (\rho e+p)
\end{pmatrix}
+\partial_{x_2}
\begin{pmatrix}
\rho v_2 \\ \rho v_1 v_2 \\ \rho v_2^2+p \\ v_2 (\rho e+p)
\end{pmatrix}
=\bm{0}.
\end{align*}
These equations determine the time evolution of the conserved variables $(\rho,rho \bm v, rho e)$, i.e. density, momentum and energy. A closure for the pressure $p$ is given by
\begin{align*}
p = (\gamma-1)\rho\left(e-\frac12(v_1^2+v_2^2)\right).
\end{align*}
Since the the fluid of the following test cases is air, we choose the heat capacity ratio $\gamma$ to be $1.4$. The spatial mesh discretizes the flow domain around the airfoil. At the airfoil boundary $\Gamma_{0}$, we use the Euler slip condition $\bm v^T\bm n = 0$, where $\bm n$ denotes the surface normal. At a sufficiently large distance away from the airfoil, we assume a far field flow with a given Mach number $Ma = 0.8$, pressure $p = 101,325$ Pa and a temperature of $273.15$ K. Now the angle of attack $\phi$ is uniformly distributed in the interval of $[0.75,1.75]$ degrees. I.e. we choose $\phi(\xi) = 1.25 + 0.5\xi$ where $\xi\sim U(-1,1)$.
The aim is to quantify the effects arising from the one-dimensional uncertainty $\xi$ with different methods. The IPM methods makes use of the acceleration strategies proposed in this work. To be able to measure the quality of the obtained solutions, we compute a reference solution using stochastic-Collocation with $100$ Gauss-Lobatto quadrature points.

In the following, we compare stochastic-Collocation with stochastic-Galerkin and IPM as well as its proposed acceleration techniques. Note that since IPM generalizes SG, all proposed techniques can be used for this method as well. For more information on the chosen entropy and the resulting solution ansatz for IPM, see \ref{app:IPM2DEuler}.

 and compare against Collocation with $5$ collocation points as well as SG and IPM with $5$ moments and $10$ quadrature points. Furthermore, we use convergence accelerated (caIPM) as well as convergence accelerated one-shot IPM (caosIPM), which we introduced in Sections \ref{sec:collIPM} and \ref{sec:OneShotIPM} with $10$ moments and $15$ quadrature points to converge the collocation solution with $5$ quadrature points to an entropy solution with increased accuracy.

\begin{figure}[h!]
\centering
		\centering
		\includegraphics[scale=0.7]{figs/{L2_error_E[rho]osIPMIPMCol}.pdf}
		\label{fig:sub1}
	\caption{L2 error at airfoil with 5 MPI and 2 OMP threads.}
	\label{fig:Obstacles2D}
\end{figure}

\begin{figure}[h!]
\centering
		\centering
		\includegraphics[scale=0.7]{figs/{L2_error_E[rho]}.pdf}
		\label{fig:sub1}
	\caption{L2 error at airfoil with 5 MPI and 2 OMP threads.}
	\label{fig:Obstacles2D}
\end{figure}

\begin{figure}[h!]
\centering
		\centering
		\includegraphics[scale=0.7]{figs/{convergence_runtime_residual}.pdf}
		\label{fig:sub1}
	\caption{Convergence to steady state with 5 MPI and 2 OMP threads.}
	\label{fig:Obstacles2D}
\end{figure}

\newgeometry{top=1.5cm, left=1.0cm}
\begin{figure}[h!]
\centering
	\begin{subfigure}{0.33\linewidth}
		\centering
		\includegraphics[scale=0.2]{figs/expectedRho/{euler2D_nacaCoarse_sc_n5_s05_aoa}.png}
		\label{fig:sub1}
	\end{subfigure}%
	\begin{subfigure}{0.33\linewidth}
		\centering
		\includegraphics[scale=0.2]{figs/expectedRho/{euler2D_nacaCoarse_sg_n5_nq10_s05_aoa}.png}
		\label{fig:sub2}
	\end{subfigure}%
	\begin{subfigure}{0.33\linewidth}
		\centering
		\includegraphics[scale=0.2]{figs/expectedRho/{euler2D_nacaCoarse_ipm_n5_nq10_s05_aoa}.png}
		\label{fig:sub1}
	\end{subfigure}
	
	\begin{subfigure}{0.33\linewidth}
		\centering
		\includegraphics[scale=0.2]{figs/expectedRho/{euler2D_nacaCoarse_caipm_n10_nq15_s05_aoa}.png}
		\label{fig:sub2}
	\end{subfigure}%
	\begin{subfigure}{0.33\linewidth}
		\centering
		\includegraphics[scale=0.2]{figs/expectedRho/{euler2D_nacaCoarse_caosipm_n10_nq15_s05_aoa}.png}
		\label{fig:sub1}
	\end{subfigure}%
	\begin{subfigure}{0.33\linewidth}
		\centering
		\includegraphics[scale=0.2]{figs/expectedRho/{euler2D_nacaCoarse_sc_n50_s05_aoa_legend}.png}
		\label{fig:sub2}
	\end{subfigure}%
	\caption{E$[\rho]$ (from top left to bottom right) SC, SG, IPM, caIPM, caosIPM, reference solution.}
	\label{fig:Obstacles2D}
\end{figure}

\begin{figure}[h!]
\centering
	\begin{subfigure}{0.33\linewidth}
		\centering
		\includegraphics[scale=0.2]{figs/varRho/{euler2D_nacaCoarse_sc_n5_s05_aoa_var}.png}
		\label{fig:sub1}
	\end{subfigure}%
	\begin{subfigure}{0.33\linewidth}
		\centering
		\includegraphics[scale=0.2]{figs/varRho/{euler2D_nacaCoarse_sg_n5_nq10_s05_aoa_var}.png}
		\label{fig:sub2}
	\end{subfigure}%
	\begin{subfigure}{0.33\linewidth}
		\centering
		\includegraphics[scale=0.2]{figs/varRho/{euler2D_nacaCoarse_ipm_n5_nq10_s05_aoa_var}.png}
		\label{fig:sub1}
	\end{subfigure}
	
	\begin{subfigure}{0.33\linewidth}
		\centering
		\includegraphics[scale=0.2]{figs/varRho/{euler2D_nacaCoarse_caipm_n10_nq15_s05_aoa_var}.png}
		\label{fig:sub2}
	\end{subfigure}%
	\begin{subfigure}{0.33\linewidth}
		\centering
		\includegraphics[scale=0.2]{figs/varRho/{euler2D_nacaCoarse_caosipm_n10_nq15_s05_aoa_var}.png}
		\label{fig:sub1}
	\end{subfigure}%
	\begin{subfigure}{0.33\linewidth}
		\centering
		\includegraphics[scale=0.2]{figs/varRho/{euler2D_nacaCoarse_sc_n50_s05_aoa_var_legend}.png}
		\label{fig:sub2}
	\end{subfigure}%
	\caption{Var$[\rho]$ (from top left to bottom right) SC, SG, IPM, caIPM, caosIPM, reference solution.}
	\label{fig:Obstacles2D}
\end{figure}

%%% error plots %%%
\begin{figure}[h!]
\centering
	\begin{subfigure}{0.33\linewidth}
		\centering
		\includegraphics[scale=0.2]{figs/expectedRho/errors/{euler2D_nacaCoarse_sc_n5_s05_aoa_error_legend}.png}
		\label{fig:sub1}
	\end{subfigure}%
	\begin{subfigure}{0.33\linewidth}
		\centering
		\includegraphics[scale=0.2]{figs/expectedRho/errors/{euler2D_nacaCoarse_sg_n5_nq10_s05_aoa_error_legend}.png}
		\label{fig:sub2}
	\end{subfigure}%
	\begin{subfigure}{0.33\linewidth}
		\centering
		\includegraphics[scale=0.2]{figs/expectedRho/errors/{euler2D_nacaCoarse_ipm_n5_nq10_s05_aoa_error_legend}.png}
		\label{fig:sub1}
	\end{subfigure}
	
	\begin{subfigure}{0.33\linewidth}
		\centering
		\includegraphics[scale=0.2]{figs/expectedRho/errors/{euler2D_nacaCoarse_caipm_n10_nq15_s05_aoa_error_legend}.png}
		\label{fig:sub2}
	\end{subfigure}%
	\begin{subfigure}{0.33\linewidth}
		\centering
		\includegraphics[scale=0.2]{figs/expectedRho/errors/{euler2D_nacaCoarse_caosipm_n10_nq15_s05_aoa_error_legend}.png}
		\label{fig:sub1}
	\end{subfigure}
	\caption{E$[\rho]$ distance to reference solution (from top left to bottom right) SC, SG, IPM, caIPM, caosIPM.}
	\label{fig:Obstacles2D}
\end{figure}

\begin{figure}[h!]
\centering
	\begin{subfigure}{0.33\linewidth}
		\centering
		\includegraphics[scale=0.2]{figs/varRho/errors/{euler2D_nacaCoarse_sc_n5_s05_aoa_var_error_legend}.png}
		\label{fig:sub1}
	\end{subfigure}%
	\begin{subfigure}{0.33\linewidth}
		\centering
		\includegraphics[scale=0.2]{figs/varRho/errors/{euler2D_nacaCoarse_sg_n5_nq10_s05_aoa_var_error_legend}.png}
		\label{fig:sub2}
	\end{subfigure}%
	\begin{subfigure}{0.33\linewidth}
		\centering
		\includegraphics[scale=0.2]{figs/varRho/errors/{euler2D_nacaCoarse_ipm_n5_nq10_s05_aoa_var_error_legend}.png}
		\label{fig:sub1}
	\end{subfigure}
	
	\begin{subfigure}{0.33\linewidth}
		\centering
		\includegraphics[scale=0.2]{figs/varRho/errors/{euler2D_nacaCoarse_caipm_n10_nq15_s05_aoa_var_error_legend}.png}
		\label{fig:sub2}
	\end{subfigure}%
	\begin{subfigure}{0.33\linewidth}
		\centering
		\includegraphics[scale=0.2]{figs/varRho/errors/{euler2D_nacaCoarse_caosipm_n10_nq15_s05_aoa_var_error_legend}.png}
		\label{fig:sub1}
	\end{subfigure}
	\caption{Var$[\rho]$ distance to reference solution (from top left to bottom right) SC, SG, IPM, caIPM, caosIPM.}
	\label{fig:Obstacles2D}
\end{figure}
\newgeometry{top=2.5cm, bottom=2.5cm}

%\section{Methods of Uncertainty Quantification}
%\subsection{Collocation}
%\subsection{Stochastic Galerkin}
%\subsection{Intrusive Polynomial Moment Method}
%\section{Method discussion}
%\begin{itemize}
%	\item kleinere Fehler [11]
%	\item Programmieraufwand [31]
%	\item zweiabhängig
%	\item adaptiv (breack up black box approach )
%	\item grobe Auflösung für Momente 
%\end{itemize}
%\section{UQCreator}
%\subsection{Adaptivity}
%\subsection{restart IPM} %relay(-strategy) IPM?
%\subsection{one-shot IPM}
%\subsection{adaptive IPM}
%\section{Results}
%\subsection{2D Euler flow over NACA0012}
%\begin{itemize}
%	\item AoA study
%	\item AoA + Ma number + mehr ?
%\end{itemize}
%\subsection{2D shallow water dam break}

\bibliographystyle{unsrt}  
\bibliography{references}
\end{document}
