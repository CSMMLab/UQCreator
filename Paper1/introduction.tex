\section{Introduction}
Hyperbolic equations play an important role in various research areas such as fluid dynamics or plasma physics. Efficient numerical methods combined with robust implementations are available for these problems, however they do not account for uncertainties which can arise in measurement data or modeling assumptions. Including the effects of uncertainties in differential equations has become an important topic in the last decades. %Examples include computational fluid dynamics \cite{bijl2013uncertainty}. 

A general hyperbolic set of equations with random initial data can be written as
\begin{subequations}\label{eq:hyperbolicProblem}
\begin{align}
\partial_t \bm{u}(t,\bm{x},\bm{\xi}) + \nabla&\cdot\bm{f}(\bm{u}(t,\bm{x},\bm{\xi})) = \bm{0} \enskip \text{ in } D, \\ \label{eq:ic}
\bm{u}(t=0,\bm{x},&\bm{\xi}) = \bm{u}_{\text{IC}}(\bm{x},\bm{\xi}),
\end{align}
\end{subequations}
where the solution $\bm u\in\mathbb{R}^p$ depends on time $t\in\mathbb{R}^+$, spatial position $\bm{x}\in D\subseteq \mathbb{R}^d$ as well as a vector of random variables $\bm{\xi}\in\Theta\subseteq\mathbb{R}^s$ with given probability density functions $f_{\Xi,i}(\xi_i)$ for $i = 1,\cdots,s$. Hence, the probability density function of $\bm{\xi}$ is then given by $f_{\Xi}(\bm\xi):=\prod_{i=1}^s f_{\Xi,i}(\xi_i)$. The physical flux is given by $\bm{f}:\mathbb{R}^p\to\mathbb{R}^{d\times p}$. To simplify notation, we assume that only the initial condition is random, i.e. $\bm{\xi}$ enters through the definition of $\bm{u}_{IC}$. Equations \eqref{eq:hyperbolicProblem} are usually supplemented with boundary conditions, which we will specify later for the individual problems.

Due to the randomness of the solution, one is interested in determining the expectation value or the variance, i.e.
\begin{align*}
\text{E}[\bm{u}] = \langle \bm{u} \rangle,\qquad \text{Var}[\bm{u}] = \langle \left( \bm{u}-\text{E}[\bm{u}]\right)^2\rangle,
\end{align*}
where we use the bracket operator $\langle \cdot \rangle := \int_{\Theta} \cdot f_{\Xi}(\bm\xi)d\xi_1 \cdots d\xi_s$. To approximate quantities of interest (such as expectation value, variance or higher order moments), the solution is spanned with a set of polynomial basis functions $\varphi_{i}:\Theta\to\mathbb{R}$ such that for the multi-index $i = (i_1,\cdots,i_s)$ we have $|i| \leq M$. Note that this yields
\begin{align}\label{eq:numberBasisFcts}
N:=\begin{pmatrix}
M+s \\ s
\end{pmatrix}
\end{align}
basis functions when defining $|i|:=\sum_{n = 1}^s |i_n|$. Commonly, these functions are chosen to be orthonormal polynomials \cite{wiener1938homogeneous} with respect to the probability function, i.e. $\langle \varphi_i \varphi_j \rangle =\prod_{n=1}^s\delta_{i_nj_n}$. The generalized polynomial chaos (gPC) expansion \cite{xiu2002wiener} approximates the solution by
\begin{align}\label{eq:SGClosure}
\mathcal{U}(\bm{\hat u};\bm\xi):= \sum_{|i|\leq M} \bm{\hat{u}}_i\varphi_i(\bm{\xi}) = \hat{\bm u}^T\bm{\varphi}(\bm\xi),
\end{align}
where the deterministic expansion coefficients $\bm{\hat{u}}_i\in\mathbb{R}^p$ are called moments. To allow a more compact notation, we collect the $N$ moments for which $\vert i \vert \leq M$ holds in the moment matrix $\hat{\bm u}:=(\bm{\hat{u}}_i)_{|i\leq M|}\in\mathbb{R}^{N\times p}$ and the corresponding basis functions in $\bm{\varphi}:=(\varphi_i)_{|i|\leq M}\in\mathbb{R}^{N}$. In the following, the dependency of $\mathcal{U}$ on $\bm \xi$ will occasionally be omitted for sake of readability. The solution ansatz \eqref{eq:SGClosure} is $L^2$ optimal if the moments are chosen to be the Fourier coefficients $\bm{\hat u}_i := \langle \bm{u}\varphi_i \rangle\in\mathbb{R}^p$. One can use the solution ansatz \eqref{eq:SGClosure} to compute the quantities of interest. Indeed, we have that
\begin{align*}
\text{E}[\mathcal{U}(\bm{\hat u})] = \bm{\hat u}_0,\quad \text{Var}[\mathcal{U}(\bm{\hat u})] = \text{E}[\mathcal{U}(\bm{\hat u})^2] - \text{E}[\mathcal{U}(\bm{\hat u})]^2 = \left(\sum_{i = 1}^N \hat{u}_{\ell i}^2\right)_{\ell = 1,\cdots,p}.
\end{align*}

Numerical methods for approximating the moments $\bm{\hat u}$ can be divided into intrusive and non-intrusive techniques. A popular non-intrusive method is the stochastic-Collocation (SC) method, see e.g. \cite{xiu2005high,babuvska2007stochastic,loeven2008probabilistic}, which computes the moments with the help of a numerical quadrature rule: For a given set of $Q$ quadrature weights $w_k$ and quadrature points $\bm{\xi}_k$, the moments are approximated by
\begin{align*}
\bm{\hat u}_i = \langle \bm{u}\varphi_i \rangle \approx \sum_{k = 1}^{Q}w_k \bm{u}({t,\bm{x},\bm{\xi}_k})\varphi_i(\bm{\xi}_k)f_{\Xi}(\bm{\xi}_k).
\end{align*} 
Commonly, SC uses sparse grids as quadrature rule, since they posses a reduced number of collocation points for multi-dimensional problems. Compared to tensorized quadrature sets, which require $O(M^s)$ quadrature points, sparse grids use $O(M(\log_2(M)^{s-1}))$ quadrature points to integrate polynomials of total degree $M$ exactly. 
Since the solution at a fixed quadrature point can be computed by a standard deterministic solver, the SC method does not require a significant implementation effort. Furthermore, SC is embarrassingly parallel, since the required computations can be carried out in parallel on different cores. A downside of collocation methods are aliasing effects, which stem from the inexact approximation of integrals. Therefore, collocation methods typically require more runs of the deterministic solver than intrusive methods \cite{xiu2009fast,alekseev2011estimation} to reach a given accuracy.

Intrusive methods are in general more difficult to implement and come along with higher numerical costs. The main idea of these methods is to derive a system of equations for the moments and then implementing a numerical solver for this system: Plugging the solution ansatz \eqref{eq:SGClosure} into the set of equations \eqref{eq:hyperbolicProblem} and projecting the resulting residual to zero yields the stochastic-Galerkin (SG) \cite{ghanem2003stochastic} moment system
\begin{subequations}\label{eq:SGMomentSystem}
\begin{align}
\partial_t \bm{\hat u}_i(t,\bm{x}) + \nabla&\cdot\langle\bm{f}(\mathcal{U}(\bm{\hat u})) \varphi_i\rangle = \bm{0}, \\
\bm{\hat u_i}(t=0,\bm{x}&) = \langle\bm{u}_{\text{IC}}(\bm{x},\cdot)\varphi_i\rangle,
\end{align}
\end{subequations}
with $|i|\leq M$. This system of $N$ equations can be solved by a deterministic method to determine the time evolution of the moments. One significant drawback of the stochastic-Galerkin method is that the resulting moment system is not necessarily hyperbolic \cite{poette2009uncertainty}. A generalization of stochastic-Galerkin, which ensures hyperbolicity is the Intrusive Polynomial Moment (IPM) method \cite{poette2009uncertainty}. The main difference to SG is that the solution ansatz of IPM is given by an optimization problem instead of a polynomial expansion \eqref{eq:SGClosure}. For a given convex entropy $s:\mathbb{R}^p\to\mathbb{R}$ for the original problem \eqref{eq:hyperbolicProblem}, this optimization problem is given by
\begin{align}\label{eq:primalProblem}
\mathcal{U}(\bm{\hat u}) = \argmin_{\bm u} \langle s(\bm u) \rangle \enskip \text{ subject to } \bm{\hat u}_i = \langle \bm u \varphi_i \rangle \text{ for } |i| \leq M.
\end{align}
Rewritten in its dual form, \eqref{eq:primalProblem} is transformed into an unconstrained optimization problem. Defining the variables $\bm{\lambda}_i\in\mathbb{R}^p$ where $i$ is again a multi index, gives the unconstrained dual problem
\begin{align}\label{eq:dualProblem}
 \bm{\hat \lambda}(\bm{\hat u}) := \argmin_{\bm{\lambda} \in \mathbb{R}^{N \times p}}
  \left\{\langle s_*(\bm{\lambda}^T \bm\varphi)\rangle - \sum_{|i|\leq M}\bm{\lambda}_i^T \bm{\hat u}_i\right\},
\end{align}
where $s_*:\mathbb{R}^p\to\mathbb{R}$ is the Legendre transformation of $s$, and $\bm{ \hat\lambda}:=(\bm{\hat{\lambda}}_i)_{|i|\leq M}\in \mathbb{R}^{N \times p}$ are called the dual variables. The solution to \eqref{eq:primalProblem} is then given by
\begin{align}\label{eq:ansatz}
 \mathcal{U}(\bm{\hat u}) = \left( \nabla_{\bm{u}} s \right)^{-1}(\bm{\hat{\lambda}}(\bm{\hat u})^T \bm{\varphi}).
\end{align}
When plugging this ansatz into the original equations \eqref{eq:hyperbolicProblem} and projecting the resulting residual to zero again yields the moment system \eqref{eq:SGMomentSystem}, but with the ansatz \eqref{eq:ansatz} instead of \eqref{eq:SGClosure}. The IPM method has several advantages: Choosing the entropy $s(\bm{u}) = \frac{1}{2}\bm{u}^T\bm{u}$ yields the stochastic-Galerkin closure, i.e. IPM generalizes different intrusive methods. Furthermore, at least for scalar problems, IPM is significantly less oscillatory compared to SG \cite{kusch2017maximum}. Also, as discussed in \cite{poette2009uncertainty}, when choosing $s(\bm u)$ to be a physically correct entropy of the deterministic problem, the IPM solution dissipates the expectation value of the entropy, which is
\begin{align*}
S(\bm{\hat u}) := \langle s( \mathcal{U}(\bm{\hat u}))\rangle,
\end{align*}
i.e. the IPM method yields a physically correct entropy solution. This again underlines a weakness of stochastic-Galerkin: If $s(\bm{u}) = \frac{1}{2}\bm{u}^T\bm{u}$ is not a correct entropy of the original problem, the SG method can lead to non-physical solution values, which can then cause a failure of the method. The main weakness of the IPM method is its computational effort, since it requires the repeated evaluation of \eqref{eq:ansatz}, which involves solving the optimization problem \eqref{eq:dualProblem}. Hence, the desirable properties of IPM come along with significantly increased numerical costs. However, IPM and minimal entropy methods in general are well suited for modern HPC architecture, which can be used to reduce the run time \cite{garrett2015optimization}. 

When studying hyperbolic equations, the moment approximations of various methods such as Stochastic Galerkin \cite{le2004uncertainty}, IPM \cite{kusch2018filtered} and stochastic-Collocation \cite{barth2013non,dwight2013adaptive} tend to show incorrect discontinuities in certain regions of the physical space. These non-physical structures dissolve when the number of basis functions is increased \cite{pettersson2009numerical,offner2017stability} or when artificial diffusion is added through the spatial numerical method \cite{offner2017stability} or filters \cite{kusch2018filtered}. Also, a multi-element approach which divides the uncertain domain into cells and uses piece-wise polynomial basis functions to represent the solution has proven to mitigate non-physical discontinuities \cite{durrwachter2018hyperbolicity}. These structures commonly arise on a small portion of the space-time domain. Therefore, intrusive methods seem to be an adequate choice since they are well suited for adaptive strategies. By locally increasing the polynomial order \cite{tryoen2012adaptive,kroker2012finite,giesselmann2017posteriori} or adding artificial viscosity \cite{kusch2018filtered} at certain spatial positions and time steps in which complex structures such as discontinuities occur, a given accuracy can be reached with significantly reduced numerical costs. In addition to that, the number of unknowns when using intrusive instead of non-intrusive methods is significantly decreased in high dimensional problems: The number of moments $N$ given by \eqref{eq:numberBasisFcts} is asymptotically smaller than the number of quadrature points, which increases with $O(M(\log_2(M)^{s-1}))$. Therefore, one aim should be to accelerate intrusive methods, since they can potentially outperform non-intrusive methods in complex and high-dimensional settings. \\

In this paper, we investigate intrusive methods for steady problems and compare them against collocation methods. For both steady and unsteady problems, we use adaptivity: 
\begin{itemize}
\item The intrusive nature of SG and IPM can be used to locally increase the number of moments whenever the solution has a complex structure in the random variable (as well as decrease the number of moments if not).
\end{itemize}
The steady setting provides different opportunities to take advantage of features of intrusive methods: 
\begin{itemize}
\item When using adaptivity, one can perform a large number of iteration to the steady state solution on a low number of moments and increase the maximal refinement level when the distance to the steady state has reached a specified barrier. Consequently, a large number of iterations will be performed by a cheap method, i.e. we can reduce numerical costs. 
\item Accelerate the convergence to the steady state IPM solution by applying IPM as a post-processing step for collocation methods: We converge the moments of the solution to a steady state with an inaccurate, but cheap collocation method and then use the resulting collocation moments as starting values for an expensive but accurate intrusive method such as IPM, which we then again converge to steady state. 
\item Compute inexact dual variables \eqref{eq:dualProblem} for IPM: Since the moments during the iteration process are inaccurate, i.e. they are not the correct steady state solution, we propose to not fully converge the dual iteration, which computes \eqref{eq:dualProblem}. Consequently, the dual variables and the moments are converges simultaneously to their steady state, which is similar to the idea of one-shot optimization in shape optimization \cite{hazra2005aerodynamic}.
\end{itemize}

The effectiveness of these acceleration ideas are tested by comparing results with stochastic-Collocation for the uncertain NACA test case as well as a bent shock tube problem. Furthermore, we present a semi-intrusive method, which facilitates the task of implementing general intrusive methods. The method only requires the numerical flux of the deterministic problem (as well as an entropy if IPM is used). We thereby provide the ability to recycle existing implementations of deterministic solvers.

The paper is structured as follows: After the introduction, we discuss the numerical discretization as well as the implementation and structure of the semi-intrusive method in section \ref{sec:framework}. Section \ref{sec:collIPM} discusses the IPM acceleration with a non-intrusive method and in section \ref{sec:OneShotIPM}, we discuss the idea of not converging the dual iteration. Section~\ref{sec:adaptivity} extends the presented numerical framework to an algorithm making use of adaptivity. A comparison of results computed with the presented methods is then given in \ref{sec:results}, followed by a summary and outlook in section \ref{sec:summary_outlook}.

