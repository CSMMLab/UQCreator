\documentclass[10pt, a4paper, titlepage, bibliography=totocnumbered]{article}

\renewcommand*\rmdefault{ppl}

\usepackage[utf8]{inputenc}
\usepackage[T1]{fontenc}
\usepackage[english]{babel}                 % Anpassung f\"ur deutsche Sprache
\usepackage{amsmath}
\usepackage{graphicx}
\usepackage{caption}
\usepackage{subcaption}
\usepackage{listings}
\usepackage{algorithm,algorithmic}

\newtheorem{theorem}{Theorem}[section]
\newtheorem{lemma}[theorem]{Lemma}
\newtheorem{proposition}[theorem]{Proposition}
\newtheorem{corollary}[theorem]{Corollary}

\usepackage{amssymb}

\newenvironment{proof}[1][Proof]{\begin{trivlist}
\item[\hskip \labelsep {\bfseries #1}]}{\end{trivlist}}
\newenvironment{definition}[1][Definition]{\begin{trivlist}
\item[\hskip \labelsep {\bfseries #1}]}{\end{trivlist}}
\newenvironment{example}[1][Example]{\begin{trivlist}
\item[\hskip \labelsep {\bfseries #1}]}{\end{trivlist}}
\newenvironment{remark}[1][Remark]{\begin{trivlist}
\item[\hskip \labelsep {\bfseries #1}]}{\end{trivlist}}

\usepackage{bm}
\usepackage{color}

\newcommand{\qed}{\hfill \ensuremath{\Box}}

\usepackage{geometry}
%\geometry{verbose,a4paper,tmargin=10mm,bmargin=15mm,lmargin=12mm,rmargin=12mm}

\begin{document}

\section*{Proof One-Shot IPM}
For classical IPM, the iteation scheme for the momen vectors is
\begin{align}\label{eq:momentIteration}
\bm{u}_j^{n+1} = \bm{c}\left(\bm{\lambda}(\bm{u}_{j-1}^n),\bm{\lambda}(\bm{u}_{j}^n),\bm{\lambda}(\bm{u}_{j+1}^n)\right),
\end{align}
where $\bm{c}:\mathbb{R}^{N+1}\times\mathbb{R}^{N+1}\times\mathbb{R}^{N+1}\to\mathbb{R}^{N+1}$ is given by
\begin{align*}
\bm{c}\left(\bm{\lambda}_{\ell},\bm{\lambda}_c,\bm{\lambda}_r\right):= \langle u(\bm{\lambda}_c^T\bm{\varphi})\bm{\varphi}\rangle - \frac{\Delta t}{\Delta x}\left(\langle g(u(\bm{\lambda}_c^T\bm{\varphi}),u(\bm{\lambda}_r^T\bm{\varphi}))\bm{\varphi}\rangle-\langle g(u(\bm{\lambda}_{\ell}^T\bm{\varphi}),u(\bm{\lambda}_r^T\bm{\varphi}))\bm{\varphi}\rangle\right).
\end{align*}
The map from the moment vector to the dual variables is given by the exact fix point of $\bm{d}:\mathbb{R}^{N+1}\times\mathbb{R}^{N+1}\to\mathbb{R}^{N+1}$,
\begin{align*}
\bm{d}(\bm{u},\bm{\lambda}):= \bm{\lambda}-\bm{B}\cdot \left(\langle u(\bm{\lambda}^T\bm{\varphi})\bm{\varphi}\rangle)-\bm{u}\right),
\end{align*}
where $\bm{B}$ is a preconditioner, which is used in the numerical scheme to ensure convergence of the fix point iteration (which we call dual iteration in the following)
\begin{align}\label{eq:dualIteration}
\bm{\lambda}_j^{m+1} = \bm{d}(\bm{u}_j^{n},\bm{\lambda}_j^m).
\end{align}
Note that here, $\bm{u}_j^{n}$ is a fixed parameter. By letting the dual iteration converge, i.e. $m\rightarrow\infty$, we obtain $\lim_{m\rightarrow\infty}\bm{d}(\bm{u}_j^{n},\bm{\lambda}_j^m) =:\bm{\lambda}_j^{n} \equiv \bm\lambda(\bm{u}_j^{n})$, which are called the dual variables of $\bm{u}_j^{n}$. Hence, the numerical scheme for IPM converges the dual iteration \eqref{eq:dualIteration} to compute the mapping from the moment vectors to their dual variables and then performs one iteration of \eqref{eq:momentIteration} to obtain the time update of the moment vector. Note that this can become extremely expensive, since in each iteration of the moment vectors, we fully converge the dual iteration for all spatial cells.

To simplify notation, we leave out the dependency on $\bm{\lambda}$ when writing the moment scheme in the following: Hence with 
\begin{align*}
\bm{\tilde c}\left(\bm{u}_{\ell}^n,\bm{u}_{c}^n,\bm{u}_{r}^n\right):=\bm{c}\left(\bm{\lambda}(\bm{u}_{\ell}^n),\bm{\lambda}(\bm{u}_{c}^n),\bm{\lambda}(\bm{u}_{r}^n)\right)
\end{align*}
we can rewrite the moment iteration as
\begin{align}
\bm{u}_j^{n+1} = \bm{\tilde c}\left(\bm{u}_{j-1}^n,\bm{u}_{j}^n,\bm{u}_{j+1}^n\right).
\end{align}

For steady problems, we assume that the IPM scheme converges to a fix point, i.e. we must have that $\rho (\bm{\tilde c}_{\bm{u}})<1$. The main idea of \textit{One-Shot IPM} is to not fully converge the dual iteration, since the moment vectors are not yet converged to the exact steady solution. So if we successively perform one update of the moment iteration and one update of the dual iteration, we obtain 
\begin{subequations}\label{eq:oneshotIPM}
\begin{align}
&\bm{\lambda}_{j}^{n} =  \bm{d}(\bm{u}_j^{n},\bm{\lambda}_j^{n-1}) \enskip \text{ for all j} \label{eq:oneshotIPMdual}\\
&\bm{u}_j^{n+1} =  \bm{c}\left(\bm{\lambda}_{j-1}^n,\bm{\lambda}_{j}^n,\bm{\lambda}_{j+1}^n\right)\label{eq:oneshotIPMmoment}.
\end{align}
\end{subequations}
\textcolor{blue}{[first moments, the duals?]}
In the following, we prove convergence of this scheme:
\begin{theorem}
The \textit{One-Shot IPM} scheme \eqref{eq:oneshotIPM} converges with the same rate as the moment iteration \eqref{eq:momentIteration} if the preconditioner $\bm{B} = \langle u'(\bm{\varphi}^T\bm{\lambda}_j^{n-1})\bm{\varphi}\bm{\varphi}^T\rangle$ is used.
\end{theorem}
\begin{proof}
We first point out that since we assume convergence of the classical IPM scheme, we have $\rho (\bm{\tilde c}_{\bm{u}})<1$ with $\bm{\tilde c}_{\bm{u}}\in\mathbb{R}^{(N+1)\cdot N_x\times (N+1)\cdot N_x}$ defined by\textcolor{blue}{[write for all j]}
\begin{align*}
\bm{\tilde c}_{\bm{u}} = 
\begin{pmatrix} 
    \partial_{\bm{u}_c}\bm{\tilde c}_{1} & \partial_{\bm{u}_r}\bm{\tilde c}_{1}& 0 & 0 & \dots \\
    \partial_{\bm{u}_{\ell}}\bm{\tilde c}_{2} & \partial_{\bm{u}_c}\bm{\tilde c}_{2} & \partial_{\bm{u}_r}\bm{\tilde c}_{2}& 0 & \dots \\
    0 & \partial_{\bm{u}_{\ell}}\bm{\tilde c}_{3} & \partial_{\bm{u}_c}\bm{\tilde c}_{3} & \partial_{\bm{u}_r}\bm{\tilde c}_{3}\\
    \vdots & & & \ddots & \\
    0 &\cdots &  0 & \partial_{\bm{u}_{\ell}}\bm{\tilde c}_{N_x} & \partial_{\bm{u}_c}\bm{\tilde c}_{N_x}
    \end{pmatrix},
\end{align*}
where we define $\bm{\tilde c}_{j}:=\bm{\tilde c}\left(\bm{u}_{j-1}^n,\bm{u}_{j}^n,\bm{u}_{j+1}^n\right)$ for all $j$. Now we have for each term inside the matrix $\bm{\tilde c}_{\bm{u}}$
\begin{align*}
\partial_{\bm{u}_{\ell}}\bm{\tilde c}_{j} = \frac{\partial \bm{c}_j}{\partial \bm{\lambda}_{\ell}}\frac{\partial \bm{\lambda}(\bm{u}_{j-1})}{\partial \bm{u}},\enskip\partial_{\bm{u}_c}\bm{\tilde c}_{j} = \frac{\partial \bm{c}_j}{\partial \bm{\lambda}_c}\frac{\partial \bm{\lambda}(\bm{u}_j)}{\partial \bm{u}},\enskip\partial_{\bm{u}_r}\bm{\tilde c}_{j} = \frac{\partial \bm{c}_j}{\partial \bm{\lambda}_r}\frac{\partial \bm{\lambda}(\bm{u}_{j+1})}{\partial \bm{u}}.
\end{align*}
We now wish to understand the structure of the terms $\partial_{\bm{u}} \bm{\lambda}(\bm{u})$. For this, we note that the exact dual state fulfills
\begin{align}\label{eq:ulambda}
\bm{u} = \langle u(\bm{\lambda}^T\bm{\varphi})\bm{\varphi}\rangle =: \bm{h}(\bm{\lambda}),
\end{align}
which is why we have the mapping $\bm{u}:\mathbb{R}^{N+1}\to\mathbb{R}^{N+1}$, $\bm{u}(\bm{\lambda}) = \bm{h}(\bm{\lambda})$. Since the solution of the dual problem for a given moment vector is unique, this mapping is bijective and therefore we have an inverse function
\begin{align}\label{eq:lambdau}
\bm{\lambda} = \bm{h}^{-1}(\bm{u}(\bm{\lambda}))
\end{align}
Now we differentiate both sides by $\bm{\lambda}$ to get
\begin{align*}
\bm{I}_{d} = \frac{\partial \bm{h}^{-1}(\bm{u}(\bm{\lambda}))}{\partial \bm{u}}\frac{\partial \bm{u}(\bm{\lambda})}{\partial \bm{\lambda}}
\end{align*}
We multiply with the matrix inverse of $\frac{\partial \bm{u}(\bm{\lambda})}{\partial \bm{\lambda}}$ and rewrite $\bm{h}^{-1}(\bm{u}(\bm{\lambda}))$ as $\bm{\lambda}(\bm{u})$ to get
\begin{align*}
\left(\frac{\partial \bm{u}(\bm{\lambda})}{\partial \bm{\lambda}}\right)^{-1} = \bm{h}^{-1}(\bm{u}(\bm{\lambda})).
\end{align*}
Note that on the left-hand-side we have the inverse of a matrix and on the right-hand-side, we have the inverse of a multi-dimensional function. By rewriting $\bm{h}^{-1}(\bm{u}(\bm{\lambda}))$ as $\bm{\lambda}(\bm{u})$ and simply computing the term $\frac{\partial \bm{u}(\bm{\lambda})}{\partial \bm{\lambda}}$ by differentiating \eqref{eq:ulambda} w.r.t. $\bm{\lambda}$, one obtains
\begin{align*}
\partial_{\bm{u}} \bm{\lambda}(\bm{u}) = \langle u'(\bm{\lambda}^T\bm{\varphi})\bm{\varphi}\bm{\varphi}^T\rangle.
\end{align*}
Now we begin to derive the spectrum of the \textit{One-Shot IPM} iteration \eqref{eq:oneshotIPM}. To obtain a single iteration function, we plug the dual iteration step \eqref{eq:oneshotIPMdual} into the moment iteration \eqref{eq:oneshotIPMmoment} to obtain
\begin{align*}
\bm{u}_j^{n+1} =  \bm{c}\left(\bm{d}(\bm{u}_{j-1}^{n},\bm{\lambda}_{j-1}^{n-1}),\bm{d}(\bm{u}_{j}^{n},\bm{\lambda}_{j}^{n-1}),\bm{d}(\bm{u}_{j+1}^{n},\bm{\lambda}_{j+1}^{n-1})\right)
\end{align*}
\end{proof}

\end{document}














