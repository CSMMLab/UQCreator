\section{Discretization and code framework}
\label{sec:framework}
\subsection{Discretization}
In the following, we discretize the moment system in space and time following according to \cite{kusch2017maximum}. Due to the fact, that stochastic-Galerkin can be interpreted as IPM with a quadratic entropy, it suffices to only derive a discretization of the IPM moment system \eqref{eq:IPMmomentSystem}.  
Omitting initial conditions and assuming a one-dimensional spatial domain, we can write this system  as
\begin{align*}
\partial_t \bm{\hat u}+\partial_x \bm{F}(\bm{\hat u}) = \bm{0}
\end{align*}
with the flux $\bm{F}:\mathbb{R}^{(N+1)\times p}\to\mathbb{R}^{(N+1)\times p}$, $\bm{F}(\bm{\hat u})=\langle f(\mathcal{U}_{ME}(\bm{\hat u}))\bm{\varphi} \rangle$, where we omit the index ME in the following for efficiency of exposition. Due to hyperbolicity of the IPM moment system, one can use a finite-volume method to approximate the time evolution of the IPM moments. First we perform the discretization of the spatial domain: We choose the discrete unknowns to be the spatial averages over each cell at time $t_n$, given by
\begin{align*}
\bm{\hat u}_{ij}^n \simeq \frac{1}{\Delta x}\int_{x_{j-1/ 2}}^{x_{j+1/ 2}}\bm{\hat u}_i(t_n,x) dx.
\end{align*}
If a moment vector in cell $j$ at time $t_n$ is denoted as $\bm{\hat u}_j^n = (\bm{\hat u}_{0j}^n,\cdots,\bm{\hat u}_{Nj}^n)^T\in\mathbb{R}^{N+1}$, the finite-volume scheme can be written in conservative form with the numerical flux $\bm{G}:\mathbb{R}^{(N+1)\times p}\times\mathbb{R}^{(N+1)\times p}\to\mathbb{R}^{(N+1)\times p}$ as
\begin{align}\label{eq:IPMDiscretization}
\bm{\hat u}_{j}^{n+1} = \bm{\hat u}_{j}^{n}  - \frac{\Delta t}{\Delta x}\left( \bm{G}(\bm{\hat u}_{j}^{n},\bm{\hat u}_{j+1}^{n})- \bm{G}(\bm{\hat u}_{j-1}^{n},\bm{\hat u}_{j}^{n})\right)
\end{align}
for $j = 1,\cdots,N_x$ and $n = 0,\cdots,N_t$, where $N_x$ is the number of spatial cells and $N_t$ is the number of time steps.
The numerical flux is assumed to be consistent, i.e., that $\bm{G}(\bm{u},\bm{u})=\bm{F}(\bm{u})$.
To ensure stability, a CFL condition has to be derived by investigating the eigenvalues of $\nabla \bm{F}$.

When a consistent numerical flux $\bm g:\mathbb{R}^p\times\mathbb{R}^p\to\mathbb{R}^p$, $\bm g = \bm g(\bm u_\ell, \bm u_r)$ is available for the deterministic problem \eqref{eq:hyperbolicProblem}, then for the IPM system we can simply take
\begin{align*}
 \bm{G}(\bm{\hat u}_{j}^n,\bm{\hat u}_{j+1}^{n}) = \langle \bm g(\mathcal{U}_{ME}(\bm{\hat u}_j^n),\mathcal{U}_{ME}(\bm{\hat u}_{j+1}^n))\bm{\varphi}\rangle.
\end{align*}
This choice of the numerical flux is a common choice in kinetic theory and is called kinetic flux.

The numerical flux requires evaluating the closure $\mathcal{U}_{ME}(\bm{\hat u}_j^n)$. To simplify notation, we define
\begin{align*}
\bm{u}_{ME}(\bm\Lambda):=\left( \nabla_{\bm{\hat u}} s \right)^{-1}(\bm\Lambda),
\end{align*}
meaning that we can write the closure as
\begin{align*}
\mathcal{U}_{ME}(\bm{\hat u}_j^n) = \bm{u}_{ME}(\bm{\hat{\lambda}}(\bm{\hat u}_j^n)^T \bm{\varphi}).
\end{align*}
The computation of the so called entropy variables $\bm{\hat\lambda}_j^n:=\bm{\hat\lambda}(\bm{\hat u}_j^n)$ requires solving the dual problem \eqref{eq:dualProblem} for the moment vector $\bm{\hat u}_{j}^{n}$. Commonly, this optimization problem is solved with Newton's method: We wish to minimize the cost function
\begin{align*}
L(\bm{\lambda}) := \langle s_*(\bm{\lambda}^T \bm\varphi)\rangle - \sum_{i=0}^{N}\bm{\lambda}_i^T \bm{\hat u}_i
\end{align*}
Algorithm 


Unfortunately \eqref{eq:exactUpdate} cannot be implemented because the dual problem cannot be solved exactly.%
\footnote{
Equation \eqref{eq:exactUpdate} also includes integral evaluations which cannot be computed in closed form.
Their approximation by numerical quadrature, however, does not play a role in the realizability problems we discuss below.
}
Instead, it must be solved numerically, for example with Newton's method.
The stopping criterion for the numerical optimizer ensures that the approximate multiplier vector it returns, which we denote $\xoverline{\bm{\lambda}}_j^n\in\mathbb{R}^{N+1}$ for the moment vector $\bm{\hat u}_{j}^{n}$, satisfies the stopping criterion
\begin{align}\label{eq:tauCrit}
\left\Vert \bm{\hat u}_{j}^{n}-\left\langle u_{ME}\left(\left(\xoverline{\bm{\lambda}}_j^n\right)^T\bm{\varphi}\right)\bm{\varphi}\right\rangle \right\Vert < \tau.
\end{align}
This is derived from the first-order necessary conditions for the dual problem.
Once the numerical optimizer finds such a $\xoverline{\bm{\lambda}}_j^n$, the corresponding dual state $\xoverline{\Lambda}_j^n := \left(\xoverline{\bm{\lambda}}_j^n\right)^T\bm{\varphi} \in \mathbb{P}(\Theta)$ can be used in \eqref{eq:exactUpdate} for the unknown $\hat \Lambda_j^n$.
This gives Algorithm \ref{alg:seq}.

\begin{algorithm}[H]
\begin{algorithmic}[1]
\FOR{$j=0$ to $N_x+1$}
\STATE $\bm{u}_j^0 = \frac{1}{\Delta x} \int_{x_{j-1/ 2}}^{x_{j+1/ 2}} \langle u_0(x, \cdot) \bm{\varphi} \rangle dx$
\ENDFOR
\FOR{$n=0$ to $N_t$}
\FOR{$j=0$ to $N_x+1$}
\STATE $\xoverline{\bm{\lambda}}_j^n \approx \argmin_{\bm{\lambda}}  \left( \langle s_*(\bm{\lambda}^T \bm{\varphi})\rangle - \bm{\lambda}^T \bm{\hat u}_{j}^{n} \right)$
\hfill such that \eqref{eq:tauCrit} holds
\STATE $\xoverline \Lambda_j^n = \left(\xoverline{\bm{\lambda}}_j^n\right)^T\bm{\varphi}$
\ENDFOR
\FOR{$j=1$ to $N_x$}
\STATE $\bm{\hat u}_{j}^{n+1} = \bm{\hat u}_{j}^{n}- \frac{\Delta t}{\Delta x}\left( \langle g(u_{ME}(\xoverline \Lambda_j^n),u_{ME}(\xoverline \Lambda_{j+1}^n))\bm{\varphi}\rangle - \langle g(u_{ME}(\xoverline \Lambda_{j-1}^n),u_{ME}(\xoverline \Lambda_{j}^n))\bm{\varphi} \rangle\right)$ 
\ENDFOR
\ENDFOR
\end{algorithmic}
\caption{IPM for Uncertainty Quantification}
\label{alg:seq}
\end{algorithm}