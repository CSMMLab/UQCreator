\section{Discretization and code framework}
\label{sec:framework}
\subsection{Discretization}
Omitting initial conditions, we can write the IPM moment system \eqref{eq:IPMmomentSystem} as
\begin{align*}
\partial_t \bm{u}+\partial_x \bm{F}(\bm{u}) = \bm{0}
\end{align*}
with the flux $\bm{F}:\mathbb{R}^{N+1}\to\mathbb{R}^{N+1}$, $\bm{F}(\bm{u})=\langle f(u_{ME}(\hat \Lambda(\bm{u})))\bm{\varphi} \rangle$ depending on the dual state 
\begin{align*}
\hat \Lambda(\bm{u}) = \bm{\hat{\lambda}}(\bm{u})^T\bm{\varphi}.
\end{align*}
For efficiency of exposition, we sometimes omit the dependence on $\bm{u}$.
The IPM system is hyperbolic, so it is naturally solved by a finite-volume method. First we discretize the spatial domain into cells. The discrete unknowns are chosen to be the spatial averages over each cell at time $t_n$, given by
\begin{align*}
u_{ij}^n \simeq \frac{1}{\Delta x}\int_{x_{j-1/ 2}}^{x_{j+1/ 2}}u_i(t_n,x) dx.
\end{align*}
If a moment vector in cell $j$ at time $t_n$ is denoted as $\bm{u}_j^n = (u_{0j}^n,\cdots,u_{Nj}^n)^T\in\mathbb{R}^{N+1}$, the finite-volume scheme can be written in conservative form with the numerical flux $\bm{G}:\mathbb{R}^{N+1}\times\mathbb{R}^{N+1}\to\mathbb{R}^{N+1}$ as
\begin{align}\label{eq:IPMDiscretization}
\bm{u}_{j}^{n+1} = \bm{u}_{j}^{n}  - \frac{\Delta t}{\Delta x}\left( \bm{G}(\bm{u}_j^n,\bm{u}_{j+1}^n)- \bm{G}(\bm{u}_{j-1}^n,\bm{u}_{j}^n)\right)
\end{align}
for $j = 1,\cdots,N_x$ and $n = 0,\cdots,N_t$, where $N_x$ is the number of spatial cells and $N_t$ is the number of time steps.
%The discretization of the dual state $\Lambda$ at spatial cell $j$ and time step $n$ is given by
%\begin{align*}
%\hat{\Lambda}_j^n = \bm{\hat{\lambda}}(\bm{u}_j^n)^T\bm{\varphi}.
%\end{align*}
The numerical flux is assumed to be consistent, i.e., that $\bm{G}(\bm{u},\bm{u})=\bm{F}(\bm{u})$.
To ensure stability, a CFL condition has to be derived by investigating the eigenvalues of $\nabla \bm{F}$.

When a consistent numerical flux $g:\mathbb{R}\times\mathbb{R}\to\mathbb{R}$, $g = g(u_\ell, u_r)$ is available for the deterministic problem \eqref{eq:origProblem}, then for the IPM system we can simply take
\begin{align*}
 \bm{G}(\bm{u}_{j}^n,\bm{u}_{j+1}^n) = \langle g(u_{ME}(\hat\Lambda(\bm{u}_{j}^n)),u_{ME}(\hat\Lambda(\bm{u}_{j+1}^n)))\bm{\varphi}\rangle.
\end{align*}
This choice of the numerical flux is a common choice in kinetic theory and is called kinetic flux.
The time update of the moment vector now becomes
\begin{align}\label{eq:exactUpdate}
\bm{u}_{j}^{n+1} = \bm{u}_{j}^{n}- \frac{\Delta t}{\Delta x}\left( \langle g(u_{ME}(\hat \Lambda_j^n),u_{ME}(\hat \Lambda_{j+1}^n))\bm{\varphi}\rangle - \langle g(u_{ME}(\hat \Lambda_{j-1}^n),u_{ME}(\hat \Lambda_{j}^n))\bm{\varphi} \rangle\right),
\end{align}
where $\hat\Lambda_{j}^n :=\hat\Lambda(\bm{u}_{j}^n)$ for all $j$. Note that the computation of $\hat\Lambda_{j}^n$ requires solving the dual problem \eqref{eq:dualProblem} for the moment vector $\bm{u}_{j}^n$.

Unfortunately \eqref{eq:exactUpdate} cannot be implemented because the dual problem cannot be solved exactly.%
\footnote{
Equation \eqref{eq:exactUpdate} also includes integral evaluations which cannot be computed in closed form.
Their approximation by numerical quadrature, however, does not play a role in the realizability problems we discuss below.
}
Instead, it must be solved numerically, for example with Newton's method.
The stopping criterion for the numerical optimizer ensures that the approximate multiplier vector it returns, which we denote $\xoverline{\bm{\lambda}}_j^n\in\mathbb{R}^{N+1}$ for the moment vector $\bm{u}_j^n$, satisfies the stopping criterion
\begin{align}\label{eq:tauCrit}
\left\Vert \bm{u}_j^n-\left\langle u_{ME}\left(\left(\xoverline{\bm{\lambda}}_j^n\right)^T\bm{\varphi}\right)\bm{\varphi}\right\rangle \right\Vert < \tau.
\end{align}
This is derived from the first-order necessary conditions for the dual problem.
Once the numerical optimizer finds such a $\xoverline{\bm{\lambda}}_j^n$, the corresponding dual state $\xoverline{\Lambda}_j^n := \left(\xoverline{\bm{\lambda}}_j^n\right)^T\bm{\varphi} \in \mathbb{P}(\Theta)$ can be used in \eqref{eq:exactUpdate} for the unknown $\hat \Lambda_j^n$.
This gives Algorithm \ref{alg:seq}.

\begin{algorithm}[H]
\begin{algorithmic}[1]
\FOR{$j=0$ to $N_x+1$}
\STATE $\bm{u}_j^0 = \frac{1}{\Delta x} \int_{x_{j-1/ 2}}^{x_{j+1/ 2}} \langle u_0(x, \cdot) \bm{\varphi} \rangle dx$
\ENDFOR
\FOR{$n=0$ to $N_t$}
\FOR{$j=0$ to $N_x+1$}
\STATE $\xoverline{\bm{\lambda}}_j^n \approx \argmin_{\bm{\lambda}}  \left( \langle s_*(\bm{\lambda}^T \bm{\varphi})\rangle - \bm{\lambda}^T \bm{u}_j^n \right)$
\hfill such that \eqref{eq:tauCrit} holds
\STATE $\xoverline \Lambda_j^n = \left(\xoverline{\bm{\lambda}}_j^n\right)^T\bm{\varphi}$
\ENDFOR
\FOR{$j=1$ to $N_x$}
\STATE $\bm{u}_{j}^{n+1} = \bm{u}_{j}^{n}- \frac{\Delta t}{\Delta x}\left( \langle g(u_{ME}(\xoverline \Lambda_j^n),u_{ME}(\xoverline \Lambda_{j+1}^n))\bm{\varphi}\rangle - \langle g(u_{ME}(\xoverline \Lambda_{j-1}^n),u_{ME}(\xoverline \Lambda_{j}^n))\bm{\varphi} \rangle\right)$ 
\ENDFOR
\ENDFOR
\end{algorithmic}
\caption{IPM for Uncertainty Quantification}
\label{alg:seq}
\end{algorithm}