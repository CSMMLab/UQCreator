\section{Discretization of the IPM system}
\label{sec:framework}
\subsection{Finite Volume Discretization}
In the following, we discretize the moment system in space and time according to \cite{kusch2017maximum}. Due to the fact, that stochastic-Galerkin can be interpreted as IPM with a quadratic entropy, it suffices to only derive a discretization of the IPM moment system. Hence, we discretize the system \eqref{eq:SGMomentSystem} with the more general IPM solution ansatz \eqref{eq:ansatz}.  
Omitting initial conditions and assuming a one-dimensional spatial domain, we can write the IPM system  as
\begin{align*}
\partial_t \bm{\hat u}+\partial_x \bm{F}(\bm{\hat u}) = \bm{0}
\end{align*}
with the flux $\bm{F}:\mathbb{R}^{N\times m}\to\mathbb{R}^{N\times m}$, $\bm{F}(\bm{\hat u})=\langle \bm f(\mathcal{U}(\bm{\hat u}))\bm{\varphi}^T \rangle^T$. Note that the inner transpose represents a dyadic product and therefore the outer transpose is applied to a matrix. Due to hyperbolicity of the IPM moment system, one can use a finite-volume method to approximate the time evolution of the IPM moments. We choose the discrete unknowns which represent the solution to be the spatial averages over each cell at time $t_n$, given by
\begin{align*}
\bm{\hat u}_{ij}^n \simeq \frac{1}{\Delta x}\int_{x_{j-1/ 2}}^{x_{j+1/ 2}}\bm{\hat u}_i(t_n,x) dx.
\end{align*}
If a moment vector in cell $j$ at time $t_n$ is denoted as $\bm{\hat u}_j^n = (\bm{\hat u}_{0j}^n,\cdots,\bm{\hat u}_{Nj}^n)^T\in\mathbb{R}^{N\times m}$ \remja{0...N -> N+1}, the finite-volume scheme can be written in conservative form with the numerical flux $\bm{G}:\mathbb{R}^{N\times m}\times\mathbb{R}^{N\times m}\to\mathbb{R}^{N\times m}$ as
\begin{align}\label{eq:IPMDiscretization}
\bm{\hat u}_{j}^{n+1} = \bm{\hat u}_{j}^{n}  - \frac{\Delta t}{\Delta x}\left( \bm{G}(\bm{\hat u}_{j}^{n},\bm{\hat u}_{j+1}^{n})- \bm{G}(\bm{\hat u}_{j-1}^{n},\bm{\hat u}_{j}^{n})\right)
\end{align}
for $j = 1,\cdots,N_x$ and $n = 0,\cdots,N_t$. Here, the number of spatial cells is denoted by $N_x$ and the number of time steps by $N_t$.
The numerical flux is assumed to be consistent, i.e. $\bm{G}(\bm{\hat{u}},\bm{\hat{u}})=\bm{F}(\bm{\hat{u}})$.

When a consistent numerical flux $\bm g:\mathbb{R}^m\times\mathbb{R}^m\to\mathbb{R}^m$, $\bm g = \bm g(\bm u_\ell, \bm u_r)$ is available for the original problem \eqref{eq:hyperbolicProblem}, then for the IPM system we can simply take the numerical flux
\begin{align*}
 \bm{\tilde G}(\bm{\hat u}_{j}^n,\bm{\hat u}_{j+1}^{n}) = \langle \bm g(\mathcal{U}(\bm{\hat u}_j^n),\mathcal{U}(\bm{\hat u}_{j+1}^n))\bm{\varphi}^T\rangle^T
\end{align*}
in \eqref{eq:IPMDiscretization}. Commonly, this integral cannot be evaluated analytically and therefore needs to be approximated by a quadrature rule
\begin{align*}
\langle h \rangle \approx \langle h \rangle_{Q} := \sum_{k=1}^Q w_k h(\bm{\xi}_k)f_{\Xi}(\bm{\xi}_k).
\end{align*}
The approximated numerical flux then becomes
\begin{align}\label{eq:numericalFluxIPM}
 \bm{G}(\bm{\hat u}_{j}^n,\bm{\hat u}_{j+1}^{n}) = \langle \bm g(\mathcal{U}(\bm{\hat u}_j^n),\mathcal{U}(\bm{\hat u}_{j+1}^n))\bm{\varphi}^T\rangle^T_Q.
\end{align}
Note that the numerical flux requires evaluating the ansatz $\mathcal{U}(\bm{\hat u}_j^n)$. To simplify notation, we define $\bm{u}_{s}:\mathbb{R}^p \to \mathbb{R}^p$,
\begin{align*}
\bm{u}_{s}(\bm\Lambda):=\left( \nabla_{\bm{u}} s \right)^{-1}(\bm\Lambda),
\end{align*}
meaning that the IPM ansatz \eqref{eq:ansatz} at cell $j$ in timestep $n$ can be written as
\begin{align*}
\mathcal{U}(\bm{\hat u}_j^n) = \bm{u}_{s}(\bm{\hat{\lambda}}(\bm{\hat u}_j^n)^T \bm{\varphi}).
\end{align*}
The computation of the dual variables $\bm{\hat\lambda}_j^n:=\bm{\hat\lambda}(\bm{\hat u}_j^n)$ requires solving the dual problem \eqref{eq:dualProblem} for the moment vector $\bm{\hat u}_{j}^{n}$. Hence, to determine the dual variables for a given moment vector $\bm{\hat{u}}$, the cost function
\begin{align}\label{eq:L}
L(\bm{\lambda};\bm{\hat{u}}) := \langle s_*(\bm{\lambda}^T \bm\varphi)\rangle_Q - \sum_{i\leq M}\bm{\lambda}_i^T \bm{\hat u}_i
\end{align}
needs to be minimized. Hence, one needs to find the root of
\begin{align*}
\nabla_{\bm{\lambda_i}}L(\bm{\lambda};\bm{\hat{u}}) = \langle \nabla s_*(\bm{\lambda}^T \bm\varphi)\bm\varphi^T\rangle_Q^T - \bm{\hat u}_i = \langle \bm u_s(\bm{\lambda}^T \bm\varphi)\bm\varphi^T\rangle_Q^T - \bm{\hat u}_i,
\end{align*}
where we used $\nabla s_* \equiv \bm u_s$. The root is usually determined by using Newton's method. For simplicity, let us define the full gradient of the Lagrangian to be $\nabla_{\bm{\lambda}}L(\bm{\lambda};\bm{\hat{u}})\in\mathbb{R}^{N\cdot m}$, i.e. we store all entries in a vector. Newton's method uses the iteration function $\bm{d}:\mathbb{R}^{N\times m}\times\mathbb{R}^{N\times m}\to\mathbb{R}^{N\times m}$,
\begin{align}\label{eq:dualIterationFunction}
\bm{d}(\bm{\lambda},\bm{\hat{u}}):= \bm{\lambda}-\bm{H}(\bm{\lambda})^{-1}\cdot\nabla_{\bm{\lambda}}L(\bm{\lambda};\bm{\hat{u}}),
\end{align}
where $\bm H\in\mathbb{R}^{N \cdot n\times N\cdot m}$ is the Hessian of \eqref{eq:L}, given by
\begin{align*}
\bm{H}(\bm{\lambda}) := \langle \nabla \bm{u}_{s} (\bm{\lambda}^T\bm{\varphi})\otimes\bm{\varphi}\bm{\varphi}^T\rangle_Q^{T}.
\end{align*}
%Inside the Newton update \eqref{eq:dualIterationFunction}, we abuse notation for better readability by making use of
%\begin{align*}
%\left(\bm{H}^{-1}\cdot\nabla_{\bm{\lambda}}L\right)_{ij} := \sum_{i' = 1}^N\sum_{j' = 1}^m \left(\bm{H}^{-1}\right)_{m(j-1) + i,m(j'-1) + i'}\cdot\nabla_{\bm{\lambda}}L_{i'j'}.
%\end{align*}
The function $\bm d$ will in the following be called dual iteration function. Now, the Newton iteration $\zeta$ \remja{l ist später auch level für adIPM und left flux - habs erstmal überall auf zeta geändert} for spatial cell $j$ is given by
\begin{align}\label{eq:dualIteration1}
\bm{\lambda}^{(\zeta+1)}_j = \bm{d}(\bm{\lambda}_j^{(\zeta)},\bm{\hat{u}_j}).
\end{align}
The exact dual state is then obtained by computing the fixed point of $\bm{d}$, meaning that one converges the iteration \eqref{eq:dualIteration1}, i.e. $\bm{\hat\lambda}_j^n:=\bm{\hat\lambda}(\bm{\hat u_j^n})=\lim_{\zeta\rightarrow\infty}\bm{d}(\bm{\lambda}_j^{(\zeta)},\bm{\hat{u}_j^n})$.
To obtain a finite number of iterations for the iteration in cell $j$, the stopping criterion 
\begin{align}\label{eq:tauCrit}
\sum_{i=0}^m\left\Vert \nabla_{\bm{\lambda_i}}L(\bm{\lambda}_j^{(\zeta)};\bm{\hat{u}}_j^n) \right\Vert < \tau
\end{align}
is used.

We now write down the entire scheme: To obtain a more compact notation, we define
\begin{align}\label{eq:momentIterationFunction}
\bm{c}\left(\bm{\lambda}_{\ell},\bm{\lambda}_c,\bm{\lambda}_r\right):= \langle \bm u_{s}(\bm{\lambda}_c^T\bm{\varphi})\bm{\varphi}^T\rangle_Q^T - \frac{\Delta t}{\Delta x}\left(\langle \bm g(\bm u_{s}(\bm{\lambda}_c^T\bm{\varphi}),\bm u_{s}(\bm{\lambda}_r^T\bm{\varphi}))\bm{\varphi}^T\rangle_Q^T-\langle \bm g(\bm u_{s}(\bm{\lambda}_{\ell}^T\bm{\varphi}),\bm u_{s}(\bm{\lambda}_c^T\bm{\varphi}))\bm{\varphi}^T\rangle_Q^T\right).
\end{align}
The moment iteration is then given by
\begin{align}\label{eq:momentIteration}
\bm{\hat u}_j^{n+1} = \bm{c}\left(\bm{\hat\lambda}(\bm{\hat u}_{j-1}^n),\bm{\hat\lambda}(\bm{\hat u}_{j}^n),\bm{\hat\lambda}(\bm{\hat u}_{j+1}^n)\right),
\end{align}
where the map from the moment vector to the dual variables, i.e. $\bm{\lambda}(\bm{\hat u}_{j}^n)$, is obtained by iterating
\begin{align}\label{eq:dualIteration}
\bm{\lambda}_j^{(\zeta+1)} = \bm{d}(\bm{\lambda}_{j}^{(\zeta)};\bm{\hat u}_j^{n}).
\end{align}
until condition \eqref{eq:tauCrit} is fulfilled. This gives Algorithm \ref{alg:IPM}.

\begin{algorithm}[H]
\begin{algorithmic}[1]
\For{$j=0$ to $N_x+1$}
\State $\bm{u}_j^0 \leftarrow \frac{1}{\Delta x} \int_{x_{j-1/ 2}}^{x_{j+1/ 2}} \langle u_{\text{IC}}(x, \cdot) \bm{\varphi} \rangle_Q dx$
\EndFor
\For{$n=0$ to $N_t$}
\For{$j=0$ to $N_x+1$}
\State $\bm{\lambda}_j^{(0)} \leftarrow \bm{\hat \lambda}_j^{n}$
\While{\eqref{eq:tauCrit} is violated}
\State $\bm{\lambda}_j^{(\zeta+1)} \leftarrow \bm{d}(\bm{\lambda}_{j}^{(\zeta)};\bm{\hat u}_j^{n})$
\State $\zeta \leftarrow \zeta+1$
\EndWhile
\State $\bm{\hat \lambda}_j^{n+1} \leftarrow \bm{\lambda}_j^{(\zeta)}$
\EndFor
\For{$j=1$ to $N_x$}
\State $\bm{\hat u}_j^{n+1} \leftarrow \bm{c}(\bm{\hat \lambda}_{j-1}^{n+1},\bm{\hat \lambda}_j^{n+1},\bm{\hat \lambda}_{j+1}^{n+1})$
\EndFor
\EndFor
\end{algorithmic}
\caption{IPM algorithm}
\label{alg:IPM}
\end{algorithm}

\subsection{Properties of the kinetic flux}
\label{sec:costNumFlux}

A straight-forward implementation is ensured by the choice of the numerical flux \eqref{eq:numericalFluxIPM}. This choice of the numerical flux is common in the field of transport theory, where it is called the \textit{kinetic flux}. By simply taking moments of a given numerical flux for the deterministic problem, the method can easily be applied to various physical problems whenever an implementation of $\bm g = \bm g(\bm u_\ell, \bm u_r)$ is available. Therefore, we call the proposed numerical method \textit{semi-intrusive}.

Intrusive numerical methods which compute arising integrals analytically and therefore directly depend on the moments (i.e. they do not necessitate the evaluation of the gPC expansion on quadrature points) can be constructed by performing a gPC expansion on the system flux directly \cite{debusschere2004numerical}. Examples can be found in \cite{hu2015stochastic,hu2016stochastic,tryoen2010instrusive,durrwachterahigh} for the computation of numerical fluxes and sources. While the analytic computation of arising integrals is not always more efficient \cite[Section 6]{ghanem1998stochastic}, it can also complicate recycling a deterministic solver. See \ref{app:costNumFlux} for a comparison of numerical costs when using Burgers' equation. However, when not using a quadratic entropy in the IPM method or when the physical flux of the deterministic problem is not a polynomial, it is not clear how many quadrature points the numerical quadrature rule requires to guarantee a sufficiently small quadrature error. We will study the approximation properties of IPM with different quadrature orders in Section~\ref{sec:resultsNACA1D}.