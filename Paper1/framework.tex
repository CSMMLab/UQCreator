\section{Discretization of the IPM system}
\label{sec:framework}
\subsection{Finite Volume Discretization}
In the following, we discretize the moment system in space and time according to \cite{kusch2017maximum}. Due to the fact, that stochastic-Galerkin can be interpreted as IPM with a quadratic entropy, it suffices to only derive a discretization of the IPM moment system \eqref{eq:SGMomentSystem}.  
Omitting initial conditions and assuming a one-dimensional spatial domain, we can write this system  as
\begin{align*}
\partial_t \bm{\hat u}+\partial_x \bm{F}(\bm{\hat u}) = \bm{0}
\end{align*}
with the flux $\bm{F}:\mathbb{R}^{N\times p}\to\mathbb{R}^{N\times p}$, $\bm{F}(\bm{\hat u})=\langle \bm f(\mathcal{U}(\bm{\hat u}))\bm{\varphi}^T \rangle^T$. Due to hyperbolicity of the IPM moment system, one can use a finite-volume method to approximate the time evolution of the IPM moments. We choose the discrete unknowns which represent the solution to be the spatial averages over each cell at time $t_n$, given by
\begin{align*}
\bm{\hat u}_{ij}^n \simeq \frac{1}{\Delta x}\int_{x_{j-1/ 2}}^{x_{j+1/ 2}}\bm{\hat u}_i(t_n,x) dx.
\end{align*}
If a moment vector in cell $j$ at time $t_n$ is denoted as $\bm{\hat u}_j^n = (\bm{\hat u}_{0j}^n,\cdots,\bm{\hat u}_{Nj}^n)^T\in\mathbb{R}^{N+1}$, the finite-volume scheme can be written in conservative form with the numerical flux $\bm{G}:\mathbb{R}^{N\times p}\times\mathbb{R}^{N\times p}\to\mathbb{R}^{N\times p}$ as
\begin{align}\label{eq:IPMDiscretization}
\bm{\hat u}_{j}^{n+1} = \bm{\hat u}_{j}^{n}  - \frac{\Delta t}{\Delta x}\left( \bm{G}(\bm{\hat u}_{j}^{n},\bm{\hat u}_{j+1}^{n})- \bm{G}(\bm{\hat u}_{j-1}^{n},\bm{\hat u}_{j}^{n})\right)
\end{align}
for $j = 1,\cdots,N_x$ and $n = 0,\cdots,N_t$. Here, the number of spatial cells is denoted by $N_x$ and the number of time steps by $N_t$.
The numerical flux is assumed to be consistent, i.e. $\bm{G}(\bm{\hat{u}},\bm{\hat{u}})=\bm{F}(\bm{\hat{u}})$.

When a consistent numerical flux $\bm g:\mathbb{R}^p\times\mathbb{R}^p\to\mathbb{R}^p$, $\bm g = \bm g(\bm u_\ell, \bm u_r)$ is available for the original problem \eqref{eq:hyperbolicProblem}, then for the IPM system we can simply take
\begin{align*}
 \bm{\tilde G}(\bm{\hat u}_{j}^n,\bm{\hat u}_{j+1}^{n}) = \langle \bm g(\mathcal{U}(\bm{\hat u}_j^n),\mathcal{U}(\bm{\hat u}_{j+1}^n))\bm{\varphi}^T\rangle^T.
\end{align*}
Commonly, this integral cannot be evaluated analytically and therefore needs to be approximated by a quadrature rule
\begin{align*}
\langle h \rangle \approx \langle h \rangle_{Q} := \sum_{k=1}^Q w_k h(\bm{\xi}_k)f_{\Xi}(\bm{\xi}_k).
\end{align*}
The approximated numerical flux then becomes
\begin{align}\label{eq:numericalFluxIPM}
 \bm{G}(\bm{\hat u}_{j}^n,\bm{\hat u}_{j+1}^{n}) = \langle \bm g(\mathcal{U}(\bm{\hat u}_j^n),\mathcal{U}(\bm{\hat u}_{j+1}^n))\bm{\varphi}^T\rangle^T_Q.
\end{align}
Note that the numerical flux requires evaluating the ansatz $\mathcal{U}(\bm{\hat u}_j^n)$. To simplify notation, we define $\bm{u}_{s}:\mathbb{R}^s \to \mathbb{R}^s$,
\begin{align*}
\bm{u}_{s}(\bm\Lambda):=\left( \nabla_{\bm{u}} s \right)^{-1}(\bm\Lambda),
\end{align*}
meaning that the IPM ansatz \eqref{eq:ansatz} at cell $j$ in timestep $n$ can be written as
\begin{align*}
\mathcal{U}(\bm{\hat u}_j^n) = \bm{u}_{s}(\bm{\hat{\lambda}}(\bm{\hat u}_j^n)^T \bm{\varphi}).
\end{align*}
The computation of the so called entropy variables $\bm{\hat\lambda}_j^n:=\bm{\hat\lambda}(\bm{\hat u}_j^n)$ requires solving the dual problem \eqref{eq:dualProblem} for the moment vector $\bm{\hat u}_{j}^{n}$. Hence, to determine the dual variables for a given moment vector $\bm{\hat{u}}$, the cost function
\begin{align}\label{eq:L}
L(\bm{\lambda};\bm{\hat{u}}) := \langle s_*(\bm{\lambda}^T \bm\varphi)\rangle_Q - \sum_{i\leq M}\bm{\lambda}_i^T \bm{\hat u}_i
\end{align}
needs to be minimized. Hence, one needs to find the root of
\begin{align*}
\nabla_{\bm{\lambda_i}}L(\bm{\lambda};\bm{\hat{u}}) = \langle \nabla s_*(\bm{\lambda}^T \bm\varphi)\bm\varphi^T\rangle_Q^T - \bm{\hat u}_i = \langle \bm u_s(\bm{\lambda}^T \bm\varphi)\bm\varphi^T\rangle_Q^T - \bm{\hat u}_i,
\end{align*}
where we used $\nabla s_* \equiv \bm u_s$. The root is usually determined by using Newton's method. For simplicity, let us define the full gradient of the Lagrangian to be $\nabla_{\bm{\lambda}}L(\bm{\lambda};\bm{\hat{u}})\in\mathbb{R}^{N\cdot p}$, i.e. we store all entries in a vector. Newton's method uses the iteration function $\bm{d}:\mathbb{R}^{N\times p}\times\mathbb{R}^{N\times p}\to\mathbb{R}^{N\times p}$,
\begin{align}\label{eq:dualIterationFunction}
\bm{d}(\bm{\lambda},\bm{\hat{u}}):= \bm{\lambda}-\bm{H}(\bm{\lambda})^{-1}\cdot\nabla_{\bm{\lambda}}L(\bm{\lambda};\bm{\hat{u}}),
\end{align}
where $\bm H\in\mathbb{R}^{p\cdot N\times p\cdot N}$ is the Hessian of \eqref{eq:L}, given by
\begin{align*}
\bm{H}(\bm{\lambda}) := \langle \nabla \bm{u}_{\bm{s}} (\bm{\lambda}^T\bm{\varphi})\otimes\bm{\varphi}\bm{\varphi}^T\rangle_Q^{T}.
\end{align*}
The function $\bm d$ will in the following be called dual iteration function. Now, the Newton iteration for spatial cell $j$ is given by
\begin{align}\label{eq:dualIteration1}
\bm{\lambda}^{(m+1)}_j = \bm{d}(\bm{\lambda}_j^{(m)},\bm{\hat{u}_j}).
\end{align}
The exact dual state is then obtained by computing the fixed point of $\bm{d}$, meaning that one converges the iteration \eqref{eq:dualIteration1}, i.e. $\bm{\hat\lambda}_j^n:=\bm{\hat\lambda}(\bm{\hat u_j^n})=\lim_{m\rightarrow\infty}\bm{d}(\bm{\lambda}_j^{(m)},\bm{\hat{u}_j^n})$.
To obtain a finite number of iterations for the iteration in cell $j$, a stopping criterion 
\begin{align}\label{eq:tauCrit}
\sum_{i=0}^p\left\Vert \nabla_{\bm{\lambda_i}}L(\bm{\lambda}_j^{(m)};\bm{\hat{u}}_j^n) \right\Vert < \tau
\end{align}
is used.

We now write down the entire scheme: To obtain a more compact notation, we define
\begin{align}\label{eq:momentIterationFunction}
\bm{c}\left(\bm{\lambda}_{\ell},\bm{\lambda}_c,\bm{\lambda}_r\right):= \langle \bm u_{s}(\bm{\lambda}_c^T\bm{\varphi})\bm{\varphi}^T\rangle_Q^T - \frac{\Delta t}{\Delta x}\left(\langle \bm g(\bm u_{s}(\bm{\lambda}_c^T\bm{\varphi}),\bm u_{s}(\bm{\lambda}_r^T\bm{\varphi}))\bm{\varphi}^T\rangle_Q^T-\langle \bm g(\bm u_{s}(\bm{\lambda}_{\ell}^T\bm{\varphi}),\bm u_{s}(\bm{\lambda}_c^T\bm{\varphi}))\bm{\varphi}^T\rangle_Q^T\right).
\end{align}
The moment iteration is then given by
\begin{align}\label{eq:momentIteration}
\bm{\hat u}_j^{n+1} = \bm{c}\left(\bm{\hat\lambda}(\bm{\hat u}_{j-1}^n),\bm{\hat\lambda}(\bm{\hat u}_{j}^n),\bm{\hat\lambda}(\bm{\hat u}_{j+1}^n)\right),
\end{align}
where the map from the moment vector to the dual variables, i.e. $\bm{\lambda}(\bm{\hat u}_{j}^n)$, is obtained by iterating
\begin{align}\label{eq:dualIteration}
\bm{\lambda}_j^{(m+1)} = \bm{d}(\bm{\lambda}_{j}^{(m)};\bm{\hat u}_j^{n}).
\end{align}
until condition \eqref{eq:tauCrit} is fulfilled. This gives Algorithm \ref{alg:IPM}.

\begin{algorithm}[H]
\begin{algorithmic}[1]
\For{$j=0$ to $N_x+1$}
\State $\bm{u}_j^0 \leftarrow \frac{1}{\Delta x} \int_{x_{j-1/ 2}}^{x_{j+1/ 2}} \langle u_{\text{IC}}(x, \cdot) \bm{\varphi} \rangle_Q dx$
\EndFor
\For{$n=0$ to $N_t$}
\For{$j=0$ to $N_x+1$}
\State $\bm{\lambda}_j^{(0)} \leftarrow \bm{\hat \lambda}_j^{n}$
\While{\eqref{eq:tauCrit} is violated}
\State $\bm{\lambda}_j^{(m+1)} \leftarrow \bm{d}(\bm{\lambda}_{j}^{(m)};\bm{\hat u}_j^{n})$
\State $m \leftarrow m+1$
\EndWhile
\State $\bm{\hat \lambda}_j^{n+1} \leftarrow \bm{\lambda}_j^{(m)}$
\EndFor
\For{$j=1$ to $N_x$}
\State $\bm{\hat u}_j^{n+1} \leftarrow \bm{c}(\bm{\hat \lambda}_{j-1}^{n+1},\bm{\hat \lambda}_j^{n+1},\bm{\hat \lambda}_{j+1}^{n+1})$
\EndFor
\EndFor
\end{algorithmic}
\caption{IPM algorithm}
\label{alg:IPM}
\end{algorithm}

\subsection{Costs of evaluating the numerical flux}
\label{sec:costNumFlux}

A straight-forward implementation is ensured by the choice of the numerical flux \eqref{eq:numericalFluxIPM}. This choice of the numerical flux is common in the field of transport theory, where it is called the \textit{kinetic flux}. By simply taking moments of a given deterministic flux, the software can easily be extended to various physical problems whenever an implementation of $\bm g = \bm g(\bm u_\ell, \bm u_r)$ is available. However, the need to approximate the flux by an integral seems to be numerically expensive, especially when considering that the stochastic-Galerkin method often allows an analytic computation of all arising integrals, see for example \cite{hu2015stochastic}. In the following, we discuss this issue and show the advantages of our numerical flux choice by considering the stochastic Burgers' equation
\begin{align*}
\partial_t u + \partial_x \frac{u^2}{2} &= 0,\\
u(t=0,x,\xi) &= u_{IC}(x,\xi).
\end{align*}
The scalar random variable $\xi$ is uniformly distributed in the interval $[-1,1]$, hence the gPC basis functions $\bm\varphi=(\varphi_0,\cdots,\varphi_M)^T$ are the Legendre polynomials. Choosing the SG ansatz \eqref{eq:SGClosure} and testing with the gPC basis functions yields the SG moment system
\begin{align*}
\partial_t \hat u_i + \partial_x \frac12\sum_{n,m = 0}^M \hat u_n \hat u_m \langle \varphi_n\varphi_m\varphi_i \rangle = 0.
\end{align*}
Defining the matrices $\bm C_i := \langle \bm\varphi\bm\varphi^T\varphi_i\rangle\in\mathbb{R}^{N\times N}$ gives
\begin{align*}
\partial_t \bm{\hat u} + \partial_x \bm F(\bm{\hat u}) = 0
\end{align*}
with $F_i(\bm{\hat u}) = \frac12\bm{\hat u}^T\bm C_i\bm{\hat u}$. Note that $\bm{C}_i$ can be computed analytically, hence choosing a Lax-Friedrichs flux
\begin{align}\label{eq:numFluxAnalytic}
G_i^{(LF)}(\bm{\hat u}_{\ell},\bm{\hat u}_{r}) =\frac{1}{4}\left(\bm{\hat u}_{\ell}^T \bm{C}_i \bm{\hat u}_{\ell}+\bm{\hat u}_{r}^T \bm{C}_i \bm{\hat u}_{r}\right) - \frac{\Delta x}{2\Delta t}(\bm{\hat u}_{r}-\bm{\hat u}_{\ell})_i
\end{align}
requires no integral evaluations. Recall, that the numerical flux choice made in this work gives
\begin{align}\label{eq:numericalFluxIPMBurgers}
 \bm{G}(\bm{\hat u}_{\ell},\bm{\hat u}_{r}) = \sum_{k=1}^Q w_k g(\mathcal{U}(\bm{\hat u}_{\ell};\xi_k),\mathcal{U}(\bm{\hat u}_{r};\xi_k))\bm{\varphi}(\xi_k)f_{\Xi}(\xi_k),
\end{align}
where $\mathcal{U}$ is the SG ansatz \eqref{eq:SGClosure}. When the chosen deterministic flux $g$ is Lax-Friedrichs, the order of the polynomials inside the sum is $3M=3(N-1)$. Choosing a Gauss-Lobatto quadrature rule, $Q = \frac32 N -1$ quadrature points suffice for an exact computation of the numerical flux. Indeed, with this choice of quadrature points, the numerical fluxes \eqref{eq:numFluxAnalytic} and \eqref{eq:numericalFluxIPMBurgers} are equivalent. 
Counting the number of operations, one observes that our choice of the numerical flux \eqref{eq:numericalFluxIPMBurgers} is significantly cheaper: When computing and storing the values in a matrix $\bm A\in\mathbb{R}^{Q\times N}$ with entries $a_{ki} = \varphi_i(\xi_k)$ before running the program, the numerical flux \eqref{eq:numericalFluxIPMBurgers} can be split into two parts. First, we determine the SG solution at all quadrature points, i.e. we compute $\bm{u}^{(\ell)} := \bm A \bm{\hat u}_{\ell}$ and $\bm{u}^{(r)} := \bm A \bm{\hat u}_{r}$ which requires $O(N\cdot Q)$ operations. These solution values are then used to compute the numerical flux
\begin{align*}
G_i(\bm{\hat u}_{\ell},\bm{\hat u}_{r}) &= \sum_{k=1}^Q w_k g(u^{(\ell)}_k,u^{(r)}_k)a_{ki}f_{\Xi}(\xi_k),
\end{align*}
which again requires $O(N\cdot Q)$ operations, i.e. the costs are $O(N^2)$. The evaluation of \eqref{eq:numFluxAnalytic} however requires $O(N^3)$ operations.