\section{Summary and outlook}
\label{sec:summary_outlook}
In this work, we proposed acceleration techniques, which can be applied to the intrusive framework of IPM and SG: We use a one-shot technique to iterate moments and their corresponding dual variables to their steady state simultaneously. By not fully converging the dual iteration, we can reduce computational costs. Additionally, we make use of adaptivity and propose to keep a low maximal truncation order for most of the iteration to the steady state solution. Since complicated structures in the uncertain dimension only appear on a small portion of the spacial mesh, we are able to heavily reduce computational costs. The effects of the proposed techniques have been demonstrated by comparing results obtained with IPM as well as SG against SC. In our test cases, the intrusive methods yield the same error level as SC for a reduced runtime, especially since intrusive methods require less unknowns to achieve a certain accuracy due to aliasing errors. In higher-dimensional problems, this effect is amplified since the number of unknowns to achieve a certain total degree is asymptotically smaller than the number of quadrature points in a tensorized or sparse grid. Furthermore, we could observe that the required residual at which the solution of intrusive methods reaches a steady state is smaller than for SC. Additionally, the ability to adaptively change the truncation order helps intrusive methods to compete with SC in terms of computational runtime.

In future work, we aim at further accelerating the IPM method by using non-exact Hessian approximations. Similar to the one-shot idea of not fully converging the dual problem, it seems to be plausible to not spend too much time on computing the Hessian when the moments are not close to a steady state. Hessian approximations that can be interesting are BFGS and sparse BFGS \cite[Chapter~6.1]{nocedal2006numerical}, which construct the Hessian from previously computed gradients. Note that this strategy will increase the used memory, since old Hessians or gradients from a certain number of old time steps need to be saved in every spacial cell.
Even though the non-intrusive nature of stochastic-Collocation or Monte Carlo methods allows an easy implementation, it can be important to intrusively modify the code in order to fully exploit all acceleration potentials. Synchronizing the time updates of the solution at different quadrature points yields an increased control over the solution during the computation, which can for example be uses to employ adaptive methods. In this case one can switch to a fine quadrature level in a certain spatial cell by for example computing moments with the given coarse set of collocation points. From these moments one can compute an IPM reconstruction, which one can then evaluate at a finer quadrature set. Another example of breaking up the non-intrusive nature of Monte Carlo methods can be found in \cite{poette2019gpc}, where the generation of random samples is combined with the sampling after collisions to increase efficiency. Furthermore, we aim at applying the proposed acceleration techniques to SG methods with hyperbolicity limiters \cite{wu2017stochastic,schlachter2018hyperbolicity}.