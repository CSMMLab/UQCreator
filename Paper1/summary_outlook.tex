\section{Summary and outlook}
\label{sec:summary_outlook}

In future work, we aim at further accelerating the IPM method by using non-exact Hessian approximations. Similar to the one-shot idea of not fully converging the dual problem, it seems to be plausible to not spend too much time on computing the Hessian when the moments are not close to steady state. Hessian approximations that can be interesting are BFGS and sparse BFGS, which construct the Hessian from previously computed gradients. Note that this strategy will increase the used memory, since old Hessians or gradients from a certain number of old time steps need to be saved in every spacial cell.
Even though the non-intrusive nature of stochastic-Collocation or Monte Carlo methods allows an easy implementation, it can be important to intrusively modify the code in order to fully exploit all acceleration potentials. Synchronizing the time updates of the solution at different quadrature points yields an increased control over the solution during the computation, which can for example be uses to employ adaptive methods. In this case one can switch to a fine quadrature level in a certain spatial cell by for example computing moments with the given coarse set of collocation points. From these moments one can compute an IPM reconstruction, which one can then evaluate at a finer quadrature set. Another example of breaking up the non-intrusive nature of Monte Carlo methods can be found in \cite{poette2019gpc}, where the generation of random samples is combined with the sampling after collisions to increase efficiency.